
\documentclass[../main.tex]{subfiles}
\begin{document}

% ===================================================================

\section{Conclusions and Future Work}
\label{sec:conclusions}

We presented a reliable proof-reconstruction in type theory
for the propositional fragment of the \Metis prover. We provided for
each \Metis inference rule a formal description in type theory.
This formalizations is mainly exposed in
Section~\ref{ssec:emulating-inferences}.

We built the \Athena translator tool written in \Haskell
that generates \Agda proof-terms of \Metis derivations.
\Agda files generated by this translator imports
\Agda formalizations of the \Metis reasoning~\cite{AgdaProp,AgdaMetis}.
% We tested successful using this tool and \Agda as we
% described in Section~\ref{sec:proof-reconstruction} over a subset of
% propositional problems from~\cite{Prieto-Cubides2017}.

The reconstruction approach in this study was designed to use
only syntactical aspects of the logic.
This decision was in the beginning a drawback
since it demands more detailed proofs, a
description of every transformation or deduction step performed by
the prover, which is rarely included in the output of these programs,
see Section~\ref{ssec:emulating-inferences}.
Nevertheless, we chose that syntactical treatment instead of using
semantics to extend this work towards the support of first-order logic
or other non-classical logics.
For predicative logic, recall satisfiability is undecidable and
its syntactical aspect plays a important role to reconstruct proofs.

One of the main contribution of this study was
increasing the trustworthiness of the automatic prover \Metis.
Justifying a proof by a theorem prover
has a real significant impact for these automatic tools.
The reverse engineering task to understand the prover reasoning
can reveal important issues or bugs
in many parts of these systems (\eg, preprocessing, reasoning, or
deduction modules). During this research, we had the opportunity
to contribute to \Metis by reporting some bugs\footnote{Issues No.~2,
No.~4. and commit \name{8a3f11e} in the \Metis' official repository
\url{https://github.com/gilith/metis}}.
Fortunately, all these problems were fixed quickly by
\citeauthor{hurd2003first} in the \Metis 20170810 release.

\subsection*{Future work}

Further research directions include, but are not limited to:

\begin{itemize}
\item extend the proof-reconstruction presented in this paper to
  \begin{itemize}
    \item support the proposition fragment with equality of \Metis.
    \item support other \ATPs for propositional logic like \name{EProver} or \name{Z3}. This development can be carried out by following the \name{EProver}
description on Kanso's Ph.D. thesis~\cite{Kanso2012}.
    \item support \Metis first-order proofs.
  \end{itemize}

\item use a shallow embedding instead the deep embedding
implemented in~\cite{AgdaProp} to reconstruct proofs.
\item improve some functions in Section~\ref{ssec:emulating-inferences}
\begin{itemize}
  \item by investigating the consequences of removing the \clausify
inference rule by the \canonicalize rule.
  \item by increasing the coverage of the \simplify rule. Since is fairly
complex its implementation in the \Metis source code, some cases could be omitted in Section~\ref{sssec:simplify}.
\end{itemize}

% For instance, we can

% and many cases could be omitted.
% Other improvements are related with computational efficiency of
% rules that strongly depend on reordering tasks,
% Section~\ref{sssec:resolve}.
% We believe some rules like \canonicalize can get some help by
% performing an analysis with \Athena.
\end{itemize}
\end{document}
