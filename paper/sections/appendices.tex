
\documentclass[../main.tex]{subfiles}
\begin{document}

\newpage

\begin{subappendices}

\renewcommand{\thesection}{\Alph{section}}%
% or try \arabic{section}

\section{Classical Propositional Logic}
\label{ssec:CPL}

\paragraph*{Propositions.}
We define the propositional logic as usual.  A proposition is an
expression formed by indivisible propositional atoms (e.g., symbols
$φ₀, φ₁, \dots$) and the logic constants: \emph{falsum} $(⊥)$,
\emph{verum} ($⊤$), the binary connectives ($∧, ∨, ⇒, ⇔)$, and the
negation $(¬)$. We use an inductive definition of the set of
propositions, denoted \Prop, as the mechanism to produce any
propositional formula~\cite{VanDalen1994}.

\begin{itemize}
\item $⊥, ⊤ \in \Prop$.
\item $ϕ_i \in \Prop$ for all $i \in \mathbb{N}$.
\item If $φ, ψ\in \Prop$, then $(φ\ \square\ ψ) \in \Prop$,
where the symbol ($\square$) is one of the binary connectives mentioned above.
\item If $φ \in \Prop$, then $(¬ φ) \in \Prop$.
\end{itemize}

\paragraph*{Inference rules.}
An inference rule is a deduction step of the form of a set of
propositional \emph{premises} that produces propositions as the
conclusion.  A derivation is a

is a mechanism to obtain the conclusion $φ$ from
the application of some deduction steps. Such deduction step has

As we show in Fig.~\ref{fig:derivation}, the premises are always above the line, the conclusion
is below it, and the name of the inference or the theorem used is
on the left or on the right of the line.
% As a notation convention, if the name of a rule used in the derivation step is typed using verbatim font face, for example, \verb!resolve!, it means the inference rule belongs to our implementation~\cite{AgdaProp, AgdaMetis} in the Proof Assistant of a inference rule presented
% with a similar name.

\begin{figure}
\begin{equation*}
\begin{bprooftree}
  \AxiomC{$φ ∧ ψ$}
  \RightLabel{∧-proj₂}
  \UnaryInfC{$ψ$}
  \AxiomC{$φ ∧ ψ$}
  \RightLabel{∧-proj₁}
  \UnaryInfC{$φ$}
  \RightLabel{∧-intro}
  \BinaryInfC{$ψ ∧ φ$}
\end{bprooftree}
\end{equation*}
\caption{Derivation example.}
\label{fig:derivation}
\end{figure}

\paragraph*{Theorems.}
A sequent $Γ ⊢ φ$ represents a theorem, but we
define it as a relation between a set of propositions premises $Γ$,
and $φ$ as the conclusion of the sequent.
The symbol $⊢$ is called turnstile.
If $Γ = ∅$, we write $⊢ φ$, and we say that $φ$ is a theorem.
A theorem, $Γ ⊢ φ$ , means that there is a derivation with
conclusion $φ$ and with all (uncanceled) hypotheses in $Γ$.

\paragraph*{Natural deductions.}
The natural deduction introduced by Gentzen in 1930s defines a formal system
for derivations that uses a set of inference rules as the only way to obtain a
conclusion. We use this system to refer to formal proofs as natural deduction
proofs and conversely.

In this document, we extend the \CPL formal system
Fig.~\ref{fig:CPL-inference-rules} described on
paper~\cite{Altenkirch2015} to include inferences rules for the biconditional
connective, $PEM$ and $RAA$ postulates, among others.


\paragraph*{Proof trees.}
We use trees to represent derivations and natural deduction proofs
(see for instance, the proof tree in a \Metis' derivation in
Fig.~\ref{fig:metis-example}).  In a proof tree, the root is the
conclusion of the entire derivation. The nodes labeled with the name
of the inference rule contains the result of applying that rule to
their parents. Lastly, the leaves in the tree become the premises, not
all necessary canceled.


\section{Customized \TSTP syntax}
\label{app:tstp-syntax}


\begin{itemize}
  \item We use \verb!inf! instead of \verb!inference!
  \item We shorten names generated automatically by \Metis, \eg,
\verb!sg0! instead of \verb!subgoal_0! or \verb!norm0_0!
insead of \verb!normalize_0_0!.
  \item We remove the \verb!plain! role everywhere.
  \item We remove empty fields in the inference information
  \item If the inference rule does not need arguments except its parent nodes, we remove the field of useful information.
  \item Use ⊤, ⊥, ¬, ∧, ∨, ⇒, ⇔ symbols to express the formulas.
  \item When the purpose to show a \TSTP derivation does not include
  some parts of the derivation we use instead the ellipsis (\verb!...!).
\end{itemize}


\begin{figure}
\begin{verbatim}
  fof(premise, axiom, p).
  fof(goal, conjecture, p).
  fof(subgoal_0, plain, p, inference(strip, [], [goal])).
  fof(negate_0_0, plain, ~ p, inference(negate, [], [subgoal_0])).
  fof(normalize_0_0, plain, ∼ p,
    inference(canonicalize, [], [negate_0_0])).
  fof(normalize_0_1, plain, p,
    inference(canonicalize, [], [premise])).
  fof(normalize_0_2, plain, $false,
    inference(simplify, [], [normalize_0_0, normalize_0_1]))
  cnf(refute_0_0, plain, $false,
    inference(canonicalize, [], [normalize_0_2])).
\end{verbatim}
\caption{\texttt{Metis}' \texttt{TSTP} derivation for the
problem $p\vdash p$.}
\label{fig:metis-proof-tstp}
\end{figure}

\begin{figure}
\begin{verbatim}
  fof(premise, axiom, p).
  fof(goal, conjecture, p).
  fof(sg0, p, inf(strip, [goal])).
  fof(neg0, ¬ p, inf(negate, [sg0])).
  fof(norm0, ¬ p, inf(canonicalize, [neg0])).
  fof(norm1, p, inf(canonicalize, [premise])).
  fof(norm2, ⊥, inf(simplify, [norm0, norm1]))
  cnf(refute0, ⊥, inf(canonicalize, [norm2])).
\end{verbatim}
\caption{\texttt{Metis}' \texttt{TSTP} derivation using a customized syntax}
\label{fig:metis-proof-tstp-customized}
\end{figure}

\section{Polarity for propositions}
\label{app:polarity-for-propositions}

\begin{equation}
\label{def:polarity}
  \begin{aligned}
  &\hspace{.495mm}\fpolarity : \Prop \to \abbre{Polarity}\\
    &\begin{array}{lll}
      \fpolarity &(φ₁ ∧ φ₂) &= ⊕\\
      \fpolarity &(φ₁ ∨ φ₂) &= ⊖\\
      \fpolarity &(φ₁ ⇒ φ₂) &= ⊖\\
      \fpolarity &(¬ φ)     &=
        \begin{cases}
        ⊕, &\text{ if }\fpolarity~φ=⊖;\\
        ⊖, &\text{ if }\fpolarity~φ=⊕;
        \end{cases}\\
      \fpolarity &φ     &=⊕
    \end{array}
  \end{aligned}
\end{equation}


\section{Another proof case for the \strip inference rule}
\label{app:strip-proof-case}

\begin{itemize}
\item[∙] Case $φ ≡ φ₁ ⇒ φ₂$.
\begin{equation*}
  \begin{bprooftree}
  \AxiomC{}
  \RightLabel{assume}
  \UnaryInfC{$Γ , φ₁ ⊢ φ₁$}
  \AxiomC{$Γ ⊢ \fstrip₁~(φ₁ ⇒ φ₂)~(\suc~n)$}
  \RightLabel{by~\eqref{eq:strip-fixed}}
  \UnaryInfC{$Γ ⊢ \fuh~(φ₁ ⇒ \fstrip₁~φ₂~n)$}
  \RightLabel{Lemma~\ref{lem:inv-uh-lem}}
  \UnaryInfC{$Γ ⊢ φ₁ ⇒ \fstrip₁~φ₂~n$}
  \RightLabel{weaken}
  \UnaryInfC{$Γ , φ₁ ⊢ φ₁ ⇒ \fstrip₁~φ₂~n$}
  \RightLabel{⇒-elim}
  \BinaryInfC{$Γ , φ₁ ⊢ \fstrip₁~φ₂~n$}
  \RightLabel{by~ind.~hyp.}
  \UnaryInfC{$Γ , φ₁ ⊢ φ₂$}
  \RightLabel{⇒-intro}
  \UnaryInfC{$Γ ⊢ φ₁ ⇒ φ₂$}
  \end{bprooftree}
\end{equation*}
\end{itemize}

% \section{A Small Example}
% \label{appendix}

% \subsection{\CPL Problem}

% \begin{equation*}
% \{\ (p ∨ q) ∧ (p ∨ r)\ \} \vdash p ∨ (q ∧ r)
% \end{equation*}

% \subsection{\TPTP Problem}

% \begin{verbatim}
%   fof(a1, axiom, ((p | q) & (p | r))).
%   fof(goal, conjecture, (p | (q & r))).
% \end{verbatim}

% \subsection{\Metis \TSTP Derivation}
% \verbatiminput{sections/data/problem.tstp}

% \subsection{\Agda Proof-term}
% \verbatiminput{sections/data/problem.agda}


\end{subappendices}

\end{document}
