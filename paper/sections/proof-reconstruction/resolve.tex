\documentclass[../../main.tex]{subfiles}
\begin{document}

\subsubsection{Resolve.}
\label{sssec:resolve}

Logic equivalence between propositions is a major issue to justify
prover reasoning steps. Since we left out semantics to treat only the
syntax aspects of propositional logic, our approach shows logic
equivalence by converting propositions to their conjunctive normal
form, and reordering those and the inner disjunctions to match them.

\begin{mydefinition}

A \emph{literal} is an propositional variable (positive literal) or a
negation of an propositional variable (negative literal).

\end{mydefinition}

\begin{notation}
We use \Lit as synonym of \Prop type to refer literals.
\end{notation}

\begin{mydefinition}

The \emph{negative normal form} of a formula is the logical
equivalent version of it in which negations appear only in the
literals and the formula does not contain any implications.

\end{mydefinition}

\begin{mydefinition}

The \emph{conjunctive normal form} of a formula also called clausal
normal form is the logical equivalent version expressed as a
conjunction of clauses where a \emph{clause} is the disjunction of
zero or more literals.

\end{mydefinition}

Below, we provide Lemma~\ref{lem:reorder-or},
Lemma~\ref{lem:reorder-and}, Lemma~\ref{lem:reorder-and-or} to perform such reordering tasks, the omitted proofs can be found
in~\cite{AgdaMetis}.

First, we define the $\fassoc_{⬓}$ function in~\eqref{eq:fassoc} to
convert a disjunction or a conjunction into its right-associative
form. The square symbol (⬓) can be the conjunction symbol or the
disjunction symbol. We use $\fassoc_{∧}$ to convert conjunctions and
$\fassoc_{∨}$ for disjunctions.

\begin{equation}
\label{eq:fassoc}
  \begin{aligned}
    &\hspace{.495mm}\fassoc_{⬓} : \Prop \to \Prop\\
    &\begin{array}{lll}
    \fassoc_{⬓} &((φ₁~⬓~φ₂)~⬓~φ₃) &= \fassoc_{⬓}~(φ₁~⬓~(φ₂~⬓~φ₃))\\
    \fassoc_{⬓} &(φ₁~⬓~φ₂)        &= φ₁~⬓~\fassoc_{⬓}~φ₂\\
    \fassoc_{⬓} &φ                &= φ.
    \end{array}
  \end{aligned}
\end{equation}


\begin{mainlemma}
\label{lem:rassoc}
  If $Γ ⊢ φ$ then $Γ ⊢ \fassoc_{⬓}~φ$.
\end{mainlemma}

\begin{myremark}

In \TPTP syntax, the formulas are in left-associative form by
default. Despite of that convention, \Metis assumes the formulas to
be in right-associative form by default. This is a matter to take
into account for the proof-reconstruction.

\end{myremark}

The $\fbuild_{∨}$ function defined in~\eqref{eq:build-or} intends to
construct a disjunction from another disjunction, specifically, this
function will try to rearrange the disjuncts in the source formula to
be the target disjunction formula.

\begin{mainlemma}
\label{lem:build-or}
  If $Γ ⊢ φ$ and $ψ : \Target$ then $Γ ⊢ \fbuild_{∨}~φ~ψ$, where
\begin{equation}
  \begin{split}
  \label{eq:build-or}
  &\fbuild_{∨} : \Source → \Target → \Prop\\
  &\fbuild_{∨}~φ~ψ~ =%\\
  % &\hspace{3mm}
  \begin{cases}
  ψ, &\text{ if } φ ≡ ψ;\\
  ψ, &\text{ if } ψ ≡ ψ₁ ∨ ψ₂\text{ and } ψᵢ ≡ \fbuild_{∨}~φ~ψᵢ\text{ for some }i=1, 2;\\
  φ, &\text{ otherwise.}
  \end{cases}
  \end{split}
\end{equation}
\end{mainlemma}

From now on, we assume all propositions to be right-associative
unless otherwise stated.

The $\ffactor$ function in~\eqref{eq:factor-definition}
simplifies a special case of disjunction, the repeated disjuncts
(\eg, $\ffactor~(φ ∨ φ)~=~φ$).
Notice that other cases like $φ~∨~(ψ~∨~φ)$ do not reduce to $ψ~∨~φ$.
We use this function in Lemma~\ref{lem:sbuild-or}.

\begin{mainlemma}
\label{lem:factor}
 If $Γ ⊢ φ$ then $Γ ⊢ \ffactor~φ$, where
\begin{align}
\begin{split}
  \label{eq:factor-definition}
  &\ffactor : \Prop → \Prop\\
  &\ffactor~φ~ =
  \begin{cases}
    φ₁,  &\text{ if }φ ≡ φ₁ ∨ φ₂ \text{ and } φ₁ ≡ \ffactor~φ₂;\\
    φ,   &\text{ otherwise.}
  \end{cases}
\end{split}
\end{align}
\end{mainlemma}

To construct a disjunction~$ψ$ from a disjunction~$φ$, we have used
ideas from the description in~\cite{bohme2010} to prove equality
between nested disjunctions.  We define the $\fsbuild_{∨}$ function
in~\eqref{eq:strong-build-or} that uses every disjunct from the
source formula, $φ$, to build up the target disjunction $ψ$.

\begin{mainlemma}
\label{lem:sbuild-or}
If $Γ ⊢ φ$ and $ψ : \Target$ then $Γ ⊢ \fsbuild_{∨}~φ~ψ$, where
\begin{equation}
\label{eq:strong-build-or}
 \begin{aligned}
     &\hspace{.495mm}\fsbuild_{∨} : \Source → \Target → \Prop\\
    &\begin{array}{lll}
    \fsbuild_{∨}&(φ₁ ∨ φ₂)~ψ &=\ffactor~(\fbuild_{∨}~φ₁~ψ ∨ \fbuild_{∨}~φ₂~ψ)\\
    \fsbuild_{∨}&φ~ψ &= \fbuild_{∨}~φ~ψ.
     \end{array}
\end{aligned}
\end{equation}
\end{mainlemma}

\begin{myexamplenum}
We build the disjunction $(p ∨ q) ∨ r$ from the disjunction
$r ∨ (q ∨ p)$ using Lemma~\ref{lem:sbuild-or}.

\begin{equation*}
(\mathcal{D})\hspace{3mm}
  \begin{bprooftree}
  \AxiomC{$Γ ⊢ q$}
  \RightLabel{∨-intro₂}
  \UnaryInfC{$Γ ⊢ p ∨ q$}
  \RightLabel{∨-intro₁}
  \UnaryInfC{$Γ ⊢ (p ∨ q) ∨ r$}

  \AxiomC{$Γ ⊢ p$}
  \RightLabel{∨-intro₁}
  \UnaryInfC{$Γ ⊢ p ∨ q$}
  \RightLabel{∨-intro₁}
  \UnaryInfC{$Γ ⊢ (p ∨ q) ∨ r$}

  \RightLabel{∨-elim}
  \BinaryInfC{$Γ, q ∨ p ⊢ (p ∨ q) ∨ r$}
  \end{bprooftree}
\end{equation*}

\begin{equation*}
  \begin{bprooftree}
  \AxiomC{$Γ ⊢ r$}
  \RightLabel{∨-intro₂}
  \UnaryInfC{$Γ ⊢ (p ∨ q) ∨ r$}

  \AxiomC{$\mathcal{D}$}
  \UnaryInfC{$Γ, q ∨ p ⊢ (p ∨ q) ∨ r$}

  \RightLabel{∨-elim}
  \BinaryInfC{$Γ, r ∨ (q ∨ p) ⊢ (p ∨ q) ∨ r$}

  \RightLabel{⇒-intro}
  \UnaryInfC{$Γ ⊢ r ∨ (q ∨ p) ⇒ (p ∨ q) ∨ r$}

  \end{bprooftree}
\end{equation*}
\end{myexamplenum}

\begin{myremark}

Notice that using $\fsbuild_{∨}$ we can build not only a disjunction
with the same disjuncts of the source formula but also a complete
different disjunction by adding new disjuncts to the source formula
via introduction rules for disjunctions.

\end{myremark}

The following lemma aims to reorder nested disjunctions
by forcing the formula to be in right-associative form in order to
apply Lemma~\ref{lem:sbuild-or}.

\begin{mainlemma}
  \label{lem:reorder-or}
  If $Γ ⊢ φ$ and $ψ: \Target$ then $Γ ⊢ \freorder_{∨}~φ~ψ~$, where
  \begin{align}
    \begin{split}
    \label{eq:reorder-or}
    &\freorder_{∨} : \Source \to \Target \to \Prop\\
    &\freorder_{∨}~φ~ψ~= \fsbuild_{∨}~(\fassoc_{∨}~φ)~ψ.
    \end{split}
  \end{align}
\end{mainlemma}

\begin{proof}
Use Lemma~\ref{lem:rassoc} and Lemma~\ref{lem:sbuild-or}.
\end{proof}

Now, we define the $\freorder_{∧}$ function in~\eqref{eq:reorder-and}
to reorder nested conjunctions. This will help us in the end of this
section to reorder conjunctive normal forms.

\begin{mainlemma}
  \label{lem:reorder-and}
  If $Γ ⊢ φ$ and $ψ : \Target$ then $Γ ⊢ \freorder_{∧}~φ~ψ$, where
    \begin{align}
      \begin{split}
      \label{eq:reorder-and}
        &\freorder_{∧} : \Source \to \Target \to \Prop\\
        &\freorder_{∧}~φ~ψ =
        \begin{cases}
          φ, &\text{ if }φ ≡ ψ;\\
          ψ₁ ∧ ψ₂, &\text{ if } ψ ≡ ψ₁ ∧ ψ₂\text{, }ψ₁ ≡ \freorder_{∧}~φ~ψ₁;\\
                   &\text{ and } ψ₂ ≡ \freorder_{∧}~φ~ψ₂;\\
          φ,       &\text{ if } ψ ≡ ψ₁ ∧ ψ₂;\\
          \fconjunct~φ~ψ, &\text{ otherwise.}
        \end{cases}
      \end{split}
  \end{align}
\end{mainlemma}

\begin{myexamplenum}
\begin{equation}
\label{ex:reorder-example}
\begin{aligned}
\begin{array}{llll}
  \freorder_{∧} &(p ∧ q ∧ r)   &(r ∧ q ∧ p)     ~&=~(r ∧ q ∧ p),\\
  \freorder_{∧} &(p ∧ q ∧ r)   &(r ∧ r ∧ p)     ~&=~(r ∧ q ∧ p),\\
  \freorder_{∧} &(p ∧ q ∧ r)   &(k ∧ q ∧ p)     ~&=~(p ∧ q ∧ r),\\
  \freorder_{∧} &((p ∨ q) ∧ r) &((r ∧ (q ∨ p)) ~&=~((p ∨ q) ∧ r).
\end{array}
\end{aligned}
\end{equation}
\end{myexamplenum}

In the last example in~\eqref{ex:reorder-example}, we could not build
the conjunction $r ∧ (q ∨ p)$ since $p ∨ q$ is not syntactical equal
to $q ∨ p$. We solve this issue in Lemma~\ref{lem:reorder-and-or} by
using the $\fconjunctor$ function defined in~\eqref{eq:conjunct-or}.
The purpose of this function consists of extracting a disjunction
from a conjunction, but without matter the order of the inner disjunctions.

\begin{mainlemma}
  \label{lem:conjunct-or}
  If $Γ ⊢ φ$ and $ψ : \Target$ then $Γ ⊢ \fconjunctor~φ~ψ$, where
  \begin{equation}
    \begin{split}
    \label{eq:conjunct-or}
      &\fconjunctor : \Source → \Target \to \Prop\\
      &\fconjunctor~φ~ψ =
      \begin{cases}
        ψ, &\text{ if }φ ≡ ψ;\\
        ψ, &\text{ if } ψ ≡ \freorder_{∨}~φ~ψ;\\
        ψ, &\text{ if }ψ ≡ ψ₁ ∧ ψ₂,\ ψ₁ ≡ \fconjunctor~φ~ψ₁,\\
           &\text{ and } ψ₂ ≡ \freorder_{∨}~φ~ψ₂;\\
        ψ, &\text{ if }φ ≡ φ₁ ∧ φ₂,\ ψ ≡ \fconjunctor~φ₁~ψ;\\
        ψ, &\text{ if }φ ≡ φ₁ ∧ φ₂,\ ψ ≡ \fconjunctor~φ₂~ψ;\\
        φ, &\text{ otherwise.}
      \end{cases}
    \end{split}
  \end{equation}
\end{mainlemma}

We are able now to reorder conjunctive normal forms using the
$\freorder_{∧∨}$ function defined in~\eqref{eq:reorder-and-or}
by using the previous lemma.

\begin{mainlemma}
  \label{lem:reorder-and-or}
  If $Γ ⊢ φ$ and $ψ : \Target$ then $Γ ⊢ \freorder_{∧∨}~φ~ψ$, where
   \begin{align}
      \begin{split}
      \label{eq:reorder-and-or}
      &\freorder_{∧∨} : \Source \to \Target \to \Prop\\
      &\freorder_{∧∨}~φ~ψ=
        \begin{cases}
          ψ, &\text{ if } φ ≡ ψ;\\
          ψ, &\text{ if } ψ ≡ ψ₁ ∧ ψ₂,\ ψ₁ ≡ \freorder_{∧∨}~φ~ψ₁,\\
             &\text{ and } ψ₂ ≡ \freorder_{∧∨}~φ~ψ₂;\\
          φ, &\text{ if } ψ ≡ ψ₁ ∧ ψ₂;\\
          \fconjunctor~φ~ψ, &\text{ otherwise.}
        \end{cases}
      \end{split}
  \end{align}
\end{mainlemma}

Now, we are ready to reconstruct the \resolve rule using
Lemma~\ref{lem:reorder-and-or}. As we see in the following, the
\resolve rule is the \Metis version of the resolution theorem. This
rule takes into account, two propositions that contain a positive
literal $ℓ$ and its negation $¬ ℓ$ respectively. Then, it produces the
\emph{resolvent}, a disjunction of two propositions: the first
proposition after removing the literal $ℓ$ and the second proposition
after removing its negation $¬ ℓ$.

The positive literal~$ℓ$ must occur in the formula from the first
derivation and the negative literal~$¬ ℓ$ must occur in the formula
from the second derivation, see the pattern of the \emph{resolve}
rule in Fig.~\ref{fig:metis-inferences}.

\begin{myexamplenum}\hspace{10cm}
\label{ex:resolve-tstp}
\begin{verbatim}
  cnf(r₄, ¬ r ∨ p ∨ q, ...
  cnf(r₅, p ∨ q ∨ r, ...
  cnf(r₆, p ∨ q, inf(resolve, r, [r₅, r₄])).
\end{verbatim}

In the excerpt above, we apply resolution to the first two formulas,
$¬ r ∨ p ∨ q$ and $p ∨ q ∨ r$.
The last line tells us the literal used
for resolution is $r$. Syntactically speaking, we can not derive
neither the conclusion $p ∨ q$ in \verb!r₆! nor apply the resolution
theorem with \verb!r₄! and \verb!r₅! since the formulas do not fit
the pattern required.

If the scenario would have other like replacing \verb!r₅! by
\begin{verbatim}
  cnf(r₅, r ∨ p ∨ q, ...
\end{verbatim}
The \resolve rule have could derive $(p ∨ q) ∨ (p ∨ q)$, but again, that is not the expected result.
\end{myexamplenum}

Therefore, we perform a sequence of rearrangements inside the
involved formulas to match with the expected pattern by the
\emph{resolve} inference rule in Fig.~\ref{fig:metis-inferences}.

Using reordering after applying a customized version of the
resolution theorem defined in~\eqref{eq:rsol} we get the expected
result.

\begin{mainlemma}
  \label{lem:rsol}
  If $Γ ⊢ φ$ then $Γ ⊢ \frsol~φ$, where
  \begin{equation}
    \label{eq:rsol}
    \begin{aligned}
    &\frsol : \Prop \to \Prop\\
    &\begin{array}{ll}
      \frsol~φ &=
        \begin{cases}
          φ₂,      &\text{ if }φ ≡ (φ₁ ∨ φ₂) ∧ (¬ φ₁ ∨ φ₂);\\
          φ₂ ∨ φ₄, &\text{ if }φ ≡ (φ₁ ∨ φ₂) ∧ (¬ φ₁ ∨ φ₄);\\
          φ, &\text{ otherwise.}
        \end{cases}
      \end{array}
      \end{aligned}
\end{equation}
\end{mainlemma}

\begin{mainth}
  \label{thm:resolve}
  Let $ℓ$ be a literal, $ℓ : \Lit$, and $ψ : \Target$. If $Γ ⊢ φ₁$ and
  $Γ~⊢~φ₂$ then $Γ~⊢~\fresolve~φ₁~φ₂~ℓ~ψ$, where
  \begin{equation}
  \begin{split}
  \label{eq:resolve}
    &\fresolve : \Source \to \Source \to \Lit \to \Target \to \Prop\\
    &\fresolve~φ₁~φ₂~ℓ~ψ =
      \frsol~(\freorder_{∨}~φ₁~(ℓ ∨ ψ) ∧ \freorder_{∨}~φ₂~(¬ ℓ ∨ ψ)).
  \end{split}
  \end{equation}
\end{mainth}

\Needspace{3\baselineskip}
\begin{proof}
  \begin{equation*}
  \begin{bprooftree}
    \AxiomC{$Γ ⊢ φ₁$}
    \RightLabel{Lemma~\ref{lem:reorder-and-or}}
    \UnaryInfC{$Γ ⊢ \freorder_{∨}~φ₁~(ℓ ∨ ψ)$}
    \RightLabel{Lemma~\ref{lem:reorder-and-or}}
    \AxiomC{$Γ ⊢ φ₂$}
    \UnaryInfC{$Γ ⊢ \freorder_{∨}~φ₂~(¬ ℓ ∨ ψ)$}
    \RightLabel{∧-intro}
    \BinaryInfC{$Γ ⊢ \freorder_{∨}~φ₁~(ℓ ∨ ψ) ∧ \freorder_{∨}~φ₂~(¬ ℓ ∨ ψ)$}
    \RightLabel{Lemma~\ref{lem:rsol}}
    \UnaryInfC{$Γ ⊢ \frsol~(\freorder_{∨}~φ₁~(ℓ ∨ ψ) ∧
     \freorder_{∨}~φ₂~(¬ ℓ ∨ ψ))$}
    \RightLabel{by (\ref{eq:resolve})}
    \UnaryInfC{$Γ ⊢ \fresolve~φ₁~φ₂~ℓ~ψ$}
  \end{bprooftree}
  \end{equation*}
\end{proof}

\begin{myexamplenum}

Continuing with the problem presented in
Example~\ref{ex:resolve-tstp}, we can use Theorem \ref{thm:resolve} to derive $Γ~⊢~p~∨~q$.

\begin{equation*}
  \begin{bprooftree}
  \AxiomC{$Γ ⊢ p ∨ q ∨ r $}
  \AxiomC{$Γ ⊢ ¬ r ∨ p ∨ q$}
  \RightLabel{Theorem \ref{thm:resolve}}
  \BinaryInfC{$Γ ⊢ \fresolve~(p ∨ q ∨ r)~(¬ r ∨ p ∨ q)~r~(p ∨ q)$}
  \RightLabel{by (\ref{eq:resolve})}
  \UnaryInfC{$Γ ⊢ \frsol~(\freorder_{∨}~(p ∨ q ∨ r)~(r ∨ (p ∨ q))~∧~\freorder_{∨}~(¬ r ∨ p ∨ q)~(¬ r ∨ (p ∨ q))$}
  \RightLabel{by (\ref{eq:reorder-or})}
  \UnaryInfC{$Γ ⊢ \frsol~((r ∨ (p ∨ q)) ∧ (¬ r ∨ (p ∨ q)))$}
  \RightLabel{by (\ref{eq:rsol})}
  \UnaryInfC{$Γ ⊢ p ∨ q$}
  \end{bprooftree}
\end{equation*}
\end{myexamplenum}

\end{document}
