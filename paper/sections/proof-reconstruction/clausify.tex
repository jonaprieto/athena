\documentclass[../../main.tex]{subfiles}
\begin{document}


\subsubsection{Clausification.}
\label{sssec:clausification}

The \clausify rule transforms a
propositional formula into its clausal normal form, a conjunction
of clauses. A \emph{clause} is the disjunction of zero or more
literals and a \emph{literal} is an atom (positive literal) or a
negation of an atom (negative literal).
It is important to notice that this kind of conversion between one formula
to its clausal normal form is not unique, and \Metis has a customized
approach to perform that transformation. That is the reason we perform some
reordering for the \CNF of the source formula to fulfill this issue.

In the following \Metis \TSTP derivation, we see an example where
\clausify expands the formula \texttt{norm0} by using distributive
laws presented in conjunctive normal form conversion.

\begin{verbatim}
fof(a1, axiom, p => (q & r)).
...
fof(norm0, (~ p | (q & r)), inf(canonicalize, [a1])).
fof(norm1, ((~ p | q) & (~ p | r)), inf(clausify, [norm0])).
...
\end{verbatim}

\begin{mainth}
\label{thm:clausify}
  If $Γ ⊢ φ$ and $ψ : \Prop$ then $Γ ⊢ \fclausify~φ~ψ$, where,

  \begin{equation*}
  \begin{aligned}
  &\hspace{.495mm}\fclausify : \Prop → \Prop → \Prop\\
  &\begin{array}{llll}
  \fclausify~φ~ψ &=
         \begin{cases}
        ψ, &\text{ if }φ≡ψ\\
        \freorder_{∧∨}~(\fcnf~φ~ψ), &\text{ otherwise.}
      \end{cases}
  \end{array}
  \end{aligned}
  \end{equation*}
\end{mainth}

\begin{proof}
If $φ ≡ ψ$, $Γ ⊢ \fclausify~φ~ψ$ normalizes to $Γ ⊢ ψ$. The conclusion follows by applying the $\fsubst$ lemma.\\[2mm]
Otherwise, we use Theorem~\ref{lem:reorder-and-or} and Theorem~\ref{lem:cnf}.
\end{proof}

\begin{remark}
The \clausify rule is often preceded by the \canonicalize rule.
Both rules perform \emph{Clausification} that introduces
into the problem axioms or definitions in the domain of the solution.
The clausification algorithms is mainly described on paper by
\citeauthor{Sutcliffe1996}~\cite{Sutcliffe1996}.
\end{remark}

\end{document}
