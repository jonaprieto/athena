\documentclass[../../main.tex]{subfiles}
\begin{document}


\subsubsection{Clausification.}
\label{sssec:clausification}

The \clausify rule transforms a
propositional formula into its clausal normal form, a conjunction
of clauses. A \emph{clause} is the disjunction of zero or more
literals and a \emph{literal} is an atom (positive literal) or a
negation of an atom (negative literal).
It is important to notice that this kind of conversion
between one formula to its clausal normal form is not unique, and \Metis has a
customized approach to perform that transformation. That is the reason we
perform some reordering for the CNF of the source formula to fulfill this
issue.

In the following \Metis \TSTP derivation, we see an example where
\clausify expands the formula \texttt{normalize\_0\_0} by using distributive
laws presented in conjunctive normal form conversion.

\begin{verbatim}
fof(a1, axiom, (p => (q & r))).
...
fof(normalize_0_0, plain, (~ p | (q & r)),
    inference(canonicalize, [], [a1])).
fof(normalize_0_1, plain, ((~ p | q) & (~ p | r)),
    inference(clausify, [], [normalize_0_0])).
\end{verbatim}

\begin{definition}[clausify]
  \label{def:clausify}
 \begin{align*}
   \begin{split}
      &\fclausify : \Prop → \Prop → \Prop\\
      &\fclausify~φ~ψ =
      \begin{cases}
        ψ, &\text{ if }φ≡ψ\\
        \freorder_{∧∨}~(\fcnf~φ~ψ), &\text{ otherwise.}
      \end{cases}
      \end{split}
  \end{align*}
\end{definition}

\begin{theorem}[thm-clausify]
\label{thm:thm-clausify}
  If $Γ ⊢ φ$ and $ψ : \Prop$ then $Γ ⊢ \fclausify~φ~ψ$.
\end{theorem}

\begin{proof} If $φ ≡ ψ$, this lemmas follows after applying $\fsubst$
lemma. Otherwise, we use Theorem~\ref{thm:thm-reorder-and-or} and Theorem~\ref{thm:thm-cnf}.
\end{proof}

\begin{remark}
The \clausify rule is often preceded by the \canonicalize rule.
Both rules perform \emph{Clausification} that introduces
into the problem axioms or definitions in the domain of the solution.
The Clausification algorithms is mainly described on paper by
\citeauthor{Sutcliffe1996}~\cite{Sutcliffe1996}. We left for future work
investigate the consequences of removing $clausify$ inference by
strengthen $canonicalize$ rule in Def.~\ref{def:canonicalize}
\end{remark}

\end{document}
