\documentclass[../../main.tex]{subfiles}
\begin{document}

\subsubsection{Simplify.}
\label{sssec:simplify}

The \simplify rule is an inference that performs simplification of
definitions and tautologies. This rule transverses a list of previous
derivations by applying Lemma~\ref{lem:canon-or},
Lemma~\ref{lem:canon-and}, among others.
This rule works to find a contradiction in the first place, or a new
formula (often smaller than its input formulas) to use later in the
derivation.

We observe based on the analysis of different cases in the \TSTP
derivations that \simplify can be modeled by a function with three
arguments: two source formulas and the target formula.

Since the main purpose of the \simplify rule is simplification of
formulas, we have defined the $\freduce$ function to help removing
the negation of a given literal $ℓ$ from a input formula.

\begin{equation}
\label{def:remove-literal}
  \begin{aligned}
  &\hspace{.495mm}\freduce : \Prop → \Lit → \Prop\hspace*{3.5cm}\\
  &\begin{array}{lll}
\freduce &(φ₁~∧~φ₂)~&ℓ~= \fsimplify_{∧}~(\freduce~φ₁~ℓ~∧~\freduce~φ₂~ℓ)\\
\freduce &(φ₁~∨~φ₂)~&ℓ~= \fsimplify_{∨}~(\freduce~φ₁~ℓ~∨~\freduce~φ₂~ℓ)\\[2mm]
\freduce &φ~&ℓ~=\begin{cases}
  ⊥,  &\text{ if }φ\text{ is a literal and }ℓ ≡ \fnnf(¬ φ);\\
  φ,  &\text{ otherwise.}
  \end{cases}
   \end{array}
  \end{aligned}
\end{equation}

\begin{mainlemma}
\label{lem:reduce-literal}
Let $ℓ$ be a literal and $φ : \Prop$. If $Γ ⊢ φ$ and $Γ ⊢ ℓ$ then
$Γ~⊢~\freduce~φ~ℓ$.
\end{mainlemma}

\begin{proof} This proof is by induction on the structure of $φ$.\\

∙~Case $φ ≡ φ₁ ∧ φ₂$.

\begin{center}
\begin{scprooftree}{1}
\AxiomC{$Γ ⊢ φ₁ ∧ φ₂$}
\RightLabel{∧-proj₁}
\UnaryInfC{$Γ ⊢ φ₁$}
\AxiomC{$Γ ⊢ ℓ$}
\RightLabel{by~ind.~hyp.}
\BinaryInfC{$Γ ⊢ \freduce~φ₁~ℓ$}
\AxiomC{$Γ ⊢ φ₁ ∧ φ₂$}
\RightLabel{∧-proj₂}
\UnaryInfC{$Γ ⊢ φ₂$}
\AxiomC{$Γ ⊢ ℓ$}
\RightLabel{by~ind.~hyp.}
\BinaryInfC{$Γ ⊢ \freduce~φ₂~ℓ$}
\RightLabel{∧-intro}
\BinaryInfC{$Γ ⊢ \freduce~φ₁~ℓ~∧~\freduce~φ₂~ℓ$}
\RightLabel{Lemma~\ref{lem:canon-and}}
\UnaryInfC{$Γ ⊢ \fsimplify_{∧}~(\freduce~φ₁~ℓ~∧~\freduce~φ₂~ℓ)$}
\end{scprooftree}
\end{center}

∙~Case $φ ≡ φ₁ ∨ φ₂$.

\begin{center}
\begin{scprooftree}{1}
\AxiomC{$Γ , φ₁ ⊢ φ₁$}
\AxiomC{$Γ ⊢ ℓ$}
\RightLabel{weaken}
\UnaryInfC{$Γ , φ₁ ⊢ ℓ$}
\RightLabel{by~ind.~hyp.}
\BinaryInfC{$Γ , φ₁ ⊢ \freduce~φ₁~ℓ$}
\RightLabel{∨-intro₁}
\UnaryInfC{$Γ , φ₁ ⊢ \freduce~φ₁~ℓ ∨ \freduce~φ₂~ℓ$}
%
\AxiomC{$Γ , φ₂ ⊢ φ₂$}
\AxiomC{$Γ ⊢ ℓ$}
\RightLabel{weaken}
\UnaryInfC{$Γ , φ₂ ⊢ ℓ$}
\RightLabel{by~ind.~hyp.}
\BinaryInfC{$Γ , φ₂ ⊢ \freduce~φ₂~ℓ$}
\RightLabel{∨-intro₂}
\UnaryInfC{$Γ , φ₂ ⊢ \freduce~φ₁~ℓ ∨ \freduce~φ₂~ℓ$}
\LeftLabel{$(\mathcal{D})$\hspace{3mm}}
\RightLabel{∨-elim}
\BinaryInfC{$Γ , φ₁ ∨ φ₂ ⊢ \freduce~φ₁~ℓ ∨ \freduce~φ₂~ℓ$}
\RightLabel{⇒-intro}
\UnaryInfC{$Γ ⊢  φ₁ ∨ φ₂ ⇒ (\freduce~φ₁~ℓ ∨ \freduce~φ₂~ℓ)$}
\end{scprooftree}
\end{center}
\medskip

\def\defaultHypSeparation{\hskip.4in}
\begin{center}
\begin{scprooftree}{1}
\AxiomC{$\mathcal{D}$}
\RightLabel{⇒-elim}
\AxiomC{$Γ ⊢ φ₁ ∨ φ₂$}
\BinaryInfC{$Γ ⊢ \freduce~φ₁~ℓ ∨ \freduce~φ₂~ℓ$}
\RightLabel{Lemma~\ref{lem:canon-or}}
\UnaryInfC{$Γ ⊢ \fsimplify_{∨}~(\freduce~φ₁~ℓ ∨ \freduce~φ₂~ℓ)$}
\end{scprooftree}
\end{center}

\def\defaultHypSeparation{\hskip.2in}

∙~Case $φ$ is a literal and $ℓ ≡ \fnnf~(¬ φ)$.

\begin{center}
\begin{scprooftree}{1}
\AxiomC{$Γ ⊢ ℓ$}
\AxiomC{$ℓ ≡ \fnnf~(¬ φ)$}
\RightLabel{subst}
\BinaryInfC{$Γ ⊢ \fnnf~(¬ φ)$}
\RightLabel{Lemma~\ref{lem:nnf-inv}}
\UnaryInfC{$Γ ⊢ ¬ φ$}
\AxiomC{$Γ ⊢ φ$}
\RightLabel{¬-elim}
\BinaryInfC{$Γ ⊢ ⊥$}
\end{scprooftree}
\end{center}

∙ Otherwise use the same hypothesis $Γ ⊢ φ$.
\end{proof}

The \fsimplify function is defined in~\eqref{eq:simplify}. If some
input formula is equal to the target formula, we derive that formula.
If some formula is $⊥$, we derive the target formula by using the
$⊥-elim$ inference rule. Otherwise, the simplification functions take
place if the second input formula is a conjunction or a disjunction.
Otherwise, if the second input formula in the sources is a literal, we
use the $\freduce$ function and Lemma~\ref{lem:reduce-literal}.

\begin{equation}
\label{eq:simplify}
  \begin{aligned}
  &\hspace{.495mm}\fsimplify : \Source → \Source → \Target → \Prop\hspace*{3.5cm}\\
  & \fsimplify~φ₁~φ₂~ψ =\\
  &\hspace{3mm}
  \begin{cases}
  ψ, &\text{ if }φᵢ ≡ ⊥\text{ for some }i = 1, 2;\\
  ψ, &\text{ if }φᵢ ≡ ψ\text{ for some }i = 1, 2;\\
  \fsimplify_{∧}~(\fsimplify~φ₁~φ₂₁~ψ)~φ₂₂~ψ,
  &\text{ if }φ₂ ≡ φ₂₁ ∧ φ₂₂;\\
   \fsimplify_{∨}~(\fsimplify~φ₁~φ₂₁~ψ~∨~\fsimplify~φ₁~φ₂₂~ψ)
  &\text{ if }φ₂ ≡ φ₂₁ ∨ φ₂₂;\\
  \freduce~φ₁~φ₂, &\text{ if }φ₂\text{ is a literal};\\
  φ₁,  &\text{ otherwise.}
  \end{cases}
  \end{aligned}
\end{equation}

\begin{mainlemma}
  \label{lem:binary-simplify}
  Let $ψ : \Target$. If $Γ ⊢ φ₁$ and $Γ ⊢ φ₂$ then
  $Γ ⊢ \fsimplify~φ₁~φ₂$.
\end{mainlemma}

\begin{proof}
This proof is by induction on the structure of $φ$. \\

∙~Case $φᵢ ≡ ψ$ for some $i = 1, 2$. If $φᵢ ≡ ψ$ then by subst lemma since
$Γ ⊢ φᵢ$ we derive $Γ ⊢ ψ$.\\

∙~Case $φᵢ ≡ ⊥$ for some $i = 1, 2$.
\begin{equation*}
\begin{bprooftree}
\AxiomC{$Γ ⊢ φᵢ$}
\AxiomC{$φᵢ ≡ ⊥$}
\RightLabel{subst}
\BinaryInfC{$Γ ⊢ ⊥$}
\RightLabel{⊥-elim}
\UnaryInfC{$Γ ⊢ ψ$}
\end{bprooftree}
\end{equation*}

∙~Case $φ₂ ≡ φ₂₁ ∧ φ₂₂$.

\begin{equation*}
\begin{bprooftree}
\AxiomC{$Γ ⊢ φ₁$}
\AxiomC{$Γ ⊢ φ₂₁ ∧ φ₂₂$}
\RightLabel{∧-proj₁}
\UnaryInfC{$Γ ⊢ φ₂₁$}
\RightLabel{by~ind.~hyp.}
\BinaryInfC{$Γ ⊢ \fsimplify~φ₁~φ₂₁~ψ$}
\AxiomC{$Γ ⊢ φ₂₁ ∧ φ₂₂$}
\RightLabel{∧-proj₂}
\UnaryInfC{$Γ ⊢ φ₂₂$}
\RightLabel{by~ind.~hyp.}
\BinaryInfC{$Γ ⊢ \fsimplify~(\fsimplify~φ₁~φ₂₁~ψ)~φ₂₂~ψ$}
\end{bprooftree}
\end{equation*}

∙~Case $φ₂ ≡ φ₂₁ ∨ φ₂₂$.

\def\defaultHypSeparation{\hskip.2in}

\begin{center}
\begin{scprooftree}{1}
\AxiomC{$Γ ⊢ φ₁$}
\RightLabel{weaken}
\UnaryInfC{$Γ , φ₂₁ ⊢ φ₁$}
\AxiomC{$Γ , φ₂₁ ⊢ φ₂₁$}
\RightLabel{by~ind.~hyp.}
\BinaryInfC{$Γ , φ₂₁ ⊢ \fsimplify~φ₁~φ₂₁~ψ$}
\LeftLabel{$(\mathcal{D}_{1})$\hspace{5mm}}
\RightLabel{∨-intro₁}
\UnaryInfC{$Γ , φ₂₁ ⊢ \fsimplify~φ₁~φ₂₁~ψ ∨ \fsimplify~φ₁~φ₂₁~ψ$}
\end{scprooftree}
\end{center}
\medskip
\begin{center}
\begin{scprooftree}{1}
\AxiomC{$Γ ⊢ φ₁$}
\RightLabel{weaken}
\UnaryInfC{$Γ , φ₂₂ ⊢ φ₁$}
\AxiomC{$Γ , φ₂₂ ⊢ φ₂₂$}
\RightLabel{by~ind.~hyp.}
\BinaryInfC{$Γ , φ₂₂ ⊢ \fsimplify~φ₁~φ₂₂~ψ$}
\LeftLabel{$(\mathcal{D}_{2})$\hspace{5mm}}
\RightLabel{∨-intro₂}
\UnaryInfC{$Γ , φ₂₂ ⊢ \fsimplify~φ₁~φ₂₂~ψ ∨ \fsimplify~φ₁~φ₂₂~ψ$}
\end{scprooftree}
\end{center}
\medskip
\begin{center}
\begin{scprooftree}{1}
\AxiomC{$\mathcal{D}_{1}$}
% \UnaryInfC{$Γ , φ₂₁ ⊢ \fsimplify~φ₁~φ₂₁~ψ ∨ \fsimplify~φ₁~φ₂₁~ψ$}
\AxiomC{$\mathcal{D}_{2}$}
% \UnaryInfC{$Γ , φ₂₂ ⊢ \fsimplify~φ₁~φ₂₂~ψ ∨ \fsimplify~φ₁~φ₂₂~ψ$}
\RightLabel{∨-elim}
\BinaryInfC{$Γ , φ₂₁ ∨ φ₂₂ ⊢ \fsimplify~φ₁~φ₂₁~ψ ∨ \fsimplify~φ₁~φ₂₂~ψ$}
\RightLabel{⇒-intro}
\LeftLabel{$(\mathcal{D}_{3})$\hspace{5mm}}
\UnaryInfC{$Γ ⊢  φ₂₁ ∨ φ₂₂ ⇒ (\fsimplify~φ₁~φ₂₁~ψ ∨ \fsimplify~φ₁~φ₂₂~ψ)$}
\end{scprooftree}
\end{center}

\medskip
\begin{center}
\begin{scprooftree}{1}
\AxiomC{$\mathcal{D}_{3}$}
\UnaryInfC{$Γ ⊢  φ₂₁ ∨ φ₂₂ ⇒ (\fsimplify~φ₁~φ₂₁~ψ ∨ \fsimplify~φ₁~φ₂₂~ψ)$}
\RightLabel{⇒-elim}
\AxiomC{$Γ ⊢ φ₂₁ ∨ φ₂₂$}
\BinaryInfC{$Γ ⊢ \fsimplify~φ₁~φ₂₁~ψ ∨ \fsimplify~φ₂~φ₂₂~ψ$}
\RightLabel{Lemma~\ref{lem:canon-or}}
\UnaryInfC{$Γ ⊢ \fsimplify_{∨}~(\fsimplify~φ₁~φ₂₁~ψ ∨ \fsimplify~φ₂~φ₂₂~ψ)$}
\end{scprooftree}
\end{center}
\medskip

∙~Case $φ₂$ is a literal. Use Lemma~\ref{lem:reduce-literal} with $φ = φ₁$ and
$ℓ = φ₂$.
\end{proof}

\begin{mainth}
  \label{thm:simplify}
  Let $ψ : \Target$. Let $φᵢ : \Source$ such that $Γ ⊢ φᵢ$ for
  $i = 1, \cdots, n$ and $n \geq 2$.
  Then $Γ ⊢ \fsimplify~γ_{n-1}~φ_{n}~ψ$ where
  $γ₁ = φ₁$, and $γᵢ ≡ \fsimplify~γ_{i-1}~φᵢ~ψ$.
\end{mainth}

\begin{proof} We prove this theorem by induction on $n$.\\
∙~Case $n = 2$. Use Lemma~\ref{lem:binary-simplify}.\\
∙~Case $n > 2$. Suppose this theorem is valid for $n$, that is,
for $i = 1, \cdots, n$, $Γ ⊢ \fsimplify~γ_{n-1}~φ_{n}~ψ$. Let us prove it for $n+1$.\\

\begin{equation*}
\begin{bprooftree}
\AxiomC{$Γ ⊢ \fsimplify~γ_{n-1}~φ_{n}~ψ$}
\AxiomC{$γ_{n} ≡ \fsimplify~γ_{n-1}~φ_{n}~ψ$}
\RightLabel{subst}
\BinaryInfC{$Γ ⊢ γ_{n}$}
\AxiomC{$Γ ⊢ φ_{n+1}$}
\RightLabel{by~ind.~hyp.}
\BinaryInfC{$Γ ⊢ \fsimplify~γ_{n}~φ_{n+1}~ψ$}
\end{bprooftree}
\end{equation*}
\end{proof}

\begin{myremark}

Besides the fact that $\List~\Prop~\to~\Prop$ is the type that most
fit with the \simplify rule, we choose a different option. In the
translation from \TSTP to \Agda, we take the list of derivations and
we apply the rule by using a left folding (the \verb!foldl! function
in functional programming) with the $\fsimplify$ function over the
list of $φ₁, φ₂, \cdots, φₙ$ that avoids us to define a new theorem
type to support \List \Prop type in the conclusion side.

\end{myremark}

\begin{myexamplenum}
Let us review the following \TSTP excerpt where \simplify was used twice.

\begin{verbatim}
  fof(n₀, (¬ p ∨ q) ∧ ¬ r ∧ ¬ q ∧ (p ∨ (¬ s ∨ r)), ...
  fof(n₁, p ∨ (¬ s ∨ r), inf(conjunct, n₀)).
  fof(n₂, ¬ p ∨ q, inf(conjunct, n₀)).
  fof(n₃, ¬ q, inf(conjunct, n₀)).
  fof(n₄, ¬ p, inf(simplify, [n₂, n₃])).
  fof(n₅, ¬ r, inf(conjunct, n₀)).
  fof(n₆, ⊥, inf(simplify, [n₁, n₄, n₅])).
\end{verbatim}

\begin{enumerate}
\item The \simplify rule derives $¬ p$ in \verb!n₄!
from \verb!n₂! and \verb!n₃! derivations.

$$\fsimplify~(¬ p ∨ q)~(¬ q)~(¬ p) = ¬ p.$$
% \begin{verbatim}
% fof(norm2, ¬ p ∨ q, inf(conjunct, [norm0])).
% fof(norm3, ¬ q, inf(conjunct, [norm0])).
% fof(norm4, ¬ p, inf(simplify, [norm2, norm3])).
% \end{verbatim}
\item To derive ⊥ in \verb!n₆! we use
Theorem~\ref{thm:simplify} and we get the following proof.

\begin{equation*}
\begin{bprooftree}
\AxiomC{$Γ ⊢ p ∨ (¬ s ∧ r)$}
\AxiomC{$Γ ⊢ ¬ p$}
\RightLabel{Theorem~\ref{thm:simplify}}
\BinaryInfC{$Γ ⊢ ¬ s ∧ r$}
\AxiomC{$Γ ⊢ ¬ r$}
\RightLabel{Theorem~\ref{thm:simplify}}
\BinaryInfC{$Γ ⊢ ⊥$}
\end{bprooftree}
\end{equation*}

% \begin{verbatim}
% fof(n1, g ∨ (¬ s ∧ r), inf(conjunct, [n0])).
% fof(n4, ¬ p, inf(simplify, [n2, n3])).
% fof(n5, ¬ r, inf(conjunct, [n0])).
% fof(n6, ⊥, inf(simplify, [n1, n4, n5])).
% \end{verbatim}
\end{enumerate}
\end{myexamplenum}


% subsubsection simplify (end)
% -------------------------------------------------------------------

\end{document}
