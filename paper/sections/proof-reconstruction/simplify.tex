\documentclass[../../main.tex]{subfiles}
\begin{document}

\subsubsection{Simplification.}
\label{sssec:simplify}

The \simplify rule is an inference that performs simplification of
definitions and tautologies. This rule
transverses a list of previous derivations by applying different theorems
(e.g., \emph{modus pones}, \emph{modus tollens}, or \emph{disjunctive
syllogism}) to get a contradiction in the first place, or a smaller
convenient formula

% To emulate this rule, we use theorems from Section~\ref{sssec:canonicalize},
% Section~\ref{sssec:conjunct}.
The following \TSTP excerpt is an example of the \simplify rule.

% \begin{figure}
\begin{verbatim}
fof(normalize_0_1, plain, (g | (~ a & k)),
    inference(conjunct, [], [normalize_0_0])).
fof(normalize_0_2, plain, (~ g | q),
    inference(conjunct, [], [normalize_0_0])).
fof(normalize_0_3, plain, (~ q),
  inference(conjunct, [], [normalize_0_0])).
fof(normalize_0_4, plain, (~ g),
    inference(simplify, [], [normalize_0_2, normalize_0_3])).
fof(normalize_0_5, plain, (~ k),
inference(conjunct, [], [normalize_0_0])).
fof(normalize_0_6, plain, ($false),
    inference(simplify, [],
              [normalize_0_1, normalize_0_4, normalize_0_5])).
\end{verbatim}
% \caption{Example of \simplify rule in a \TSTP derivation of \Metis.}
% \end{figure}

%TODO check canonicalize can reduce a conjunction into one smaller.
\begin{definition}[simplify]
\label{def:simplify}
For $i,\, j = 1, 2$, $i\neq j$ and $γ : \Prop$,
\begin{align*}
    \begin{split}
      &\fsimplify : \Prop → \Prop → \Prop → \Prop\\
      &\fsimplify(φ₁, φ₂, ψ) =%\\
      % &\hspace{4mm}
      \begin{cases}
        ψ,          &\text{ if }φᵢ ≡ ψ\\
        ⊥,          &\text{ if }φᵢ ≡ ⊥\\
        φⱼ,         &\text{ if }φᵢ ≡ ⊤\\
        γ,          &\text{ if }φᵢ ≡ φⱼ ⇒ γ\\
        γ,          &\text{ if }φᵢ ≡ γ ⇒ \fnnf(¬ φⱼ)\\
        γ,          &\text{ if }φᵢ ≡ \fnnf(¬ φⱼ) ∨ γ\\
        ⊥,          &\text{ if } \fnnf(¬ φⱼ) ≡ \fconjunct(φᵢ, \fnnf(¬ φⱼ))\\
      % φ          &\text{ if }\fconjunct(nnf(¬ φᵢ), φⱼ) ≡  φⱼ\\
        ⊥,          &\text{ if } ¬ φⱼ ≡ \fcanonicalize(φᵢ, ¬ φⱼ) \\
        φ₁,         &\text{ otherwise.}\\
      \end{cases}%\hspace{10cm}
    \end{split}
\end{align*}
\end{definition}

\begin{remark}
Besides the fact that $\List\ \Prop \to \Prop$ is the type that coincides
with the \simplify rule in \TSTP derivations, we choose a different
option. In the translation from \TSTP to \Agda, we take the list of
derivations and use a classical left folding ($\rm{foldl}$) with
$\fsimplify$ function over the list of $φ₁, φ₂, \cdots, φₙ$
that avoids us to define a new sequent type to
support \List \Prop type in the conclusion side.
\end{remark}

\begin{theorem}[thm-simplify] % (fold)
  \label{thm:thm-simplify}
For $i=1,\, 2, \cdots, n$, $Γ ⊢ φᵢ$ and $ψ : \Prop$ then
if $n > 2$, $Γ ⊢ \fsimplify(γ_{n-2}, φ_{n}, ψ)$ where $γ_{i} = \fsimplify(φ_{i}, φ_{i+1}, ψ)$
else $Γ ⊢ \fsimplify(φ₁, φ₂, ψ)$.
\end{theorem}

% \begin{sketchproof}
% If $Γ ⊢ φᵢ$ for $i=1, \cdots, n$, we prove this theorem by induction on $n$.
% Let us show one case.
% \begin{itemize}
%   \item If $n ==2$ and $(canonicalize(φ₁, ¬ φ₂) ≡ ¬ φ₂)$, we need to prove
%   $Γ ⊢ simplify(φ₁, φ₂, ψ)$.
% \begin{equation*}
%   \begin{bprooftree}
%     \AxiomC{$Γ ⊢ φ₁$}
%     \RightLabel{thm-canonicalize}
%     \UnaryInfC{$Γ ⊢ canonicalize(φ₁, ¬φ₂)$}
%     \AxiomC{$¬ φ₂ ≡ canonicalize(φ₁, ¬ φ₂)$}
%     \RightLabel{subst}
%     \BinaryInfC{$Γ ⊢ ¬ φ₂ $}
%   \end{bprooftree}
%   \begin{bprooftree}
%   \AxiomC{$Γ ⊢ ¬ φ₂ $}
%   \AxiomC{$Γ ⊢ φ₂$}
%   \RightLabel{¬-elim}
%   \BinaryInfC{$Γ ⊢ ⊥$}
%   \RightLabel{by def.}
%   \UnaryInfC{$Γ ⊢ simplify(φ₁, φ₂, ψ)$}
%   \end{bprooftree}
% \end{equation*}

% \end{itemize}
% \end{sketchproof}

% subsubsection simplify (end)
% -------------------------------------------------------------------

\end{document}
