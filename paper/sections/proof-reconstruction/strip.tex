\documentclass[../../main.tex]{subfiles}
\begin{document}

\subsubsection{Stripping a goal.}
\label{sssec:strip-a-goal}

To prove a goal, \Metis splits the goal into
disjoint cases. This process produces a list of new subgoals, the
conjunction of these subgoals implies the goal. Then, a proof of the
goal becomes, in smaller proofs, one refutation for each subgoal.

\begin{myexamplenum}
The subgoals associated to a goal are introduced in the \TSTP derivation
with the \strip inference rule.

\begin{verbatim}
  fof(goal, conjecture, (p ∧ r) ∧ q)).
  fof(sg0, p, inf(strip, [goal])).
  fof(sg1, p ⇒ r, inf(strip, [goal])).
  fof(sg2, (p ∧ r) ⇒ q, inf(strip, [goal])).
\end{verbatim}

In this example, the conjecture $(p ∧ r) ∧ q$ is stripped into
tree subgoals: $p$, $p ⇒ r$ and $(p ∧ r) ⇒ q$.

\begin{equation}
\label{eq:strip-example}
(p ∧ (p ⇒ r) ∧ ((p ∧ r) ⇒ q)) ⇒ ((p ∧ r) ∧ q).
\end{equation}

\Metis proves each subgoal in the same order above from left to right
in~\eqref{eq:strip-example}.
So far, very little attention has been paid to the role of the \strip rule
in \TSTP derivations since \Metis does not make explicit the way
it uses the subgoals to prove the conjecture.
\end{myexamplenum}

We prove the correctness of the \strip inference rule in
Theorem~\ref{thm:strip}. To show that theorem, we need to prove
Lemma~\ref{lem:inv-uh-lem} and Lemma~\ref{lem:lem-inv-strip}.

\begin{mainlemma}
  \label{lem:inv-uh-lem}
Let $\text{n} : \Nat$ be the complexity measure of the $\fuh_0$ function
in~\eqref{eq:uh-definition}.
If $Γ ⊢ \fuh₁~φ~n$ then $Γ ⊢ φ$ where $\fuh₁$ is the function defined
in~\eqref{eq:uh-structured}.
\end{mainlemma}

\begin{proof}
The proof is by induction on the cases defined by the outcome of the
$\fuh_{1}$ function.
\begin{itemize}
  \item If $n = 0$, by definition in~\eqref{eq:uh-structured}
        we conclude $Γ ⊢ φ$.
  \item For $n \geq 1$, we use induction on the structure of the first
        argument.
\vskip 2mm
\begin{itemize}
\item Case $φ ≡ φ₁ ⇒ (φ₂ ⇒ φ₃)$.
\begin{equation*}
  \begin{bprooftree}
  \AxiomC{$Γ ⊢ \fuh_1~(φ₁ ⇒ (φ₂ ⇒ φ₃))~(\suc~n)$}
  \RightLabel{by~(\ref{eq:uh-structured})}
  \UnaryInfC{$Γ ⊢ \fuh_1~((φ₁ ∧ φ₂) ⇒ φ₃)~n$}
  \RightLabel{by~ind.~hyp.}
  \UnaryInfC{Γ ⊢ (φ₁ ∧ φ₂) ⇒ φ₃}
  \RightLabel{∧⇒\textsf{-to-}⇒⇒}
  \UnaryInfC{Γ ⊢ φ₁ ⇒ (φ₂ ⇒ φ₃)}
  \end{bprooftree}
\end{equation*}
using the following theorem proved in~\cite{AgdaProp},
  \begin{equation*}
  \texttt{∧⇒\textsf{-to-}⇒⇒}\ :\  Γ ⊢ (φ₁ ∧ φ₂) ⇒ φ₃ → Γ ⊢ φ₁ ⇒ (φ₂ ⇒ φ₃).
  \end{equation*}

\item Case $φ ≡ φ₁ ⇒ (φ₂ ∧ φ₃)$.
\begin{equation*}
  (\mathcal{D}_1)
  \begin{bprooftree}
    \AxiomC{$Γ ⊢ \fuh_1~(φ₁ ⇒ (φ₂ ∧ φ₃))~(\suc~n)$}
    \RightLabel{by~(\ref{eq:uh-structured})}
    \UnaryInfC{$Γ ⊢ \fuh_1~(φ₁ ⇒ φ₂)~n ∧ \fuh_1~(φ₁ ⇒ φ₃)~n$}
    \RightLabel{∧-proj₁}
    \UnaryInfC{$Γ ⊢ \fuh_1~(φ₁ ⇒ φ₂)~n$}
    \RightLabel{by~ind.~hyp.}
    \UnaryInfC{$Γ ⊢ φ₁ ⇒ φ₂$}
  \end{bprooftree}
\end{equation*}

\begin{equation*}
  (\mathcal{D}_2)
  \begin{bprooftree}
    \AxiomC{$Γ ⊢ \fuh_1~(φ₁ ⇒ (φ₂ ∧ φ₃))~(\suc~n)$}
    \RightLabel{by~(\ref{eq:uh-structured})}
    \UnaryInfC{$Γ ⊢ \fuh_1~(φ₁ ⇒ φ₂)~n ∧ \fuh_1~(φ₁ ⇒ φ₃)~n$}
    \RightLabel{∧-proj₂}
    \UnaryInfC{$Γ ⊢ \fuh_1~(φ₁ ⇒ φ₃)~n$}
    \RightLabel{by~ind.~hyp.}
    \UnaryInfC{$Γ ⊢ φ₁ ⇒ φ₃$}
    \end{bprooftree}
\end{equation*}

Now, using \texttt{⇒∧⇒\textsf{-to-}⇒∧} theorem from~\cite{AgdaProp},
\begin{equation*}
  \texttt{⇒∧⇒\textsf{-to-}⇒∧}\ :\ Γ ⊢ (φ₁ ⇒ φ₂) ∧ (φ₁ ⇒ φ₃) → Γ ⊢ φ₁ ⇒ (φ₂ ∧ φ₃),
\end{equation*}
\begin{equation*}
  \begin{bprooftree}
  \AxiomC{$\mathcal{D}_1$}
  \AxiomC{$\mathcal{D}_2$}
  \RightLabel{∧-intro}
  \BinaryInfC{$Γ ⊢ (φ₁ ⇒ φ₂) ∧ (φ₁ ⇒ φ₃)$}
  \RightLabel{\tt ⇒∧⇒\textsf{-to-}⇒∧}
  \UnaryInfC{Γ ⊢ φ₁ ⇒ (φ₂ ∧ φ₃)}
  \end{bprooftree}
\end{equation*}
\item Other cases are proved in a similar way.
\end{itemize}
\end{itemize}
\end{proof}

The $\fstrip₀$ function yields the conjunction of subgoals that implies the
goal of the problem in the \Metis \TSTP derivations but is not a structurally
recursive function.  Instead, we present its structural recursive version
in~\eqref{eq:strip-fixed}, the $\fstrip₁$ function after applying the bounded
technique described in Section~\ref{ssec:structural-recursion}. We define the
$\fstrip₁$ function based on the definition from the \Metis source code.

\begin{equation}
\label{eq:strip-fixed}
\begin{aligned}
&\hspace{.495mm}\fstrip₁ : \Prop → \Nat → \Prop\\
&\begin{array}{llll}
\fstrip₁ &(φ₁ ∧ φ₂)     &(\suc~n) &= \fuh~(\fstrip₁~φ₁~n) ∧ \fuh~(φ₁ ⇒ \fstrip₁~φ₂~n)\\
\fstrip₁ &(φ₁ ∨ φ₂)     &(\suc~n) &= \fuh~((¬ φ₁) ⇒ \fstrip₁~φ₂~n)\\
\fstrip₁ &(φ₁ ⇒ φ₂)     &(\suc~n) &= \fuh~(φ₁ ⇒ \fstrip₁~φ₂~n)\\
\fstrip₁ &(¬ (φ₁ ∧ φ₂)) &(\suc~n) &= \fuh~(φ₁ ⇒ \fstrip₁~(¬ φ₂)~n)\\
\fstrip₁ &(¬ (φ₁ ∨ φ₂)) &(\suc~n) &= \fuh~(\fstrip₁~(¬ φ₁)~n) ∧ \fuh~((¬ φ₁) ⇒ \fstrip₁~(¬ φ₂)~n)\\
\fstrip₁ &(¬ (φ₁ ⇒ φ₂)) &(\suc~n) &= \fuh~(\fstrip₁~φ₁~n) ∧ \fuh~(φ₁ ⇒ \fstrip₁~(¬ φ₂)~n)\\
\fstrip₁ &(¬ (¬ φ₁))      &(\suc~n) &= \fuh~(\fstrip₁~φ₁~n)\\
\fstrip₁ &(¬ ⊥)         &(\suc~n) &= ⊤\\
\fstrip₁ &(¬ ⊤)         &(\suc~n) &= ⊥\\
\fstrip₁ &φ             &n        &= φ.
\end{array}
\end{aligned}
\end{equation}

In a similar way as we define $\fuh_{cm}$ in~\eqref{eq:uh-complexity}, we define
$\fstrip_{cm}$ as the complexity measure for the $\fstrip₀$ function.
Then we define the \fstrip function is defined as follows.
\begin{align}
  \label{eq:strip}
  \begin{split}
  &\fstrip : \Prop → \Prop\\
  &\fstrip~φ~ = \fstrip_{1}~φ~(\fstrip_{cm}~φ).
  \end{split}
\end{align}

\begin{mainlemma}
\label{lem:lem-inv-strip}
Let $\text{n} : \Nat$ be the complexity measure of the \fstrip function defined
in~\eqref{eq:strip-fixed}.
If $Γ~⊢~\fstrip₁~φ~n$ then $Γ~⊢~φ$.
\end{mainlemma}

\begin{proof}
The proof is by induction on the structure of the
formula~$φ$ by following the equations in~\eqref{eq:strip-fixed}.
We present a straightforward case with double negation, a case with
conjunction connective, and last, the case with a negated disjunction.
We refer the reader to~\cite{AgdaMetis} for the complete proof in \Agda.

∙~Case $φ ≡ ¬ (¬ φ₁) $.
\begin{equation*}
  \begin{bprooftree}
 \AxiomC{$Γ ⊢ \fstrip₁~(¬(¬φ₁))~(\suc~n)$}
  \RightLabel{by~(\ref{eq:strip-fixed})}
  \UnaryInfC{$Γ ⊢ \fuh~(\fstrip₁~φ₁~n)$}
  \RightLabel{Lemma~\ref{lem:inv-uh-lem}}
  \UnaryInfC{$Γ ⊢ \fstrip₁~φ₁~n$}
  \RightLabel{by~ind.~hyp.}
  \UnaryInfC{$Γ ⊢ φ₁$}
  \end{bprooftree}
\end{equation*}

∙~Case $φ ≡ φ₁ ∧ φ₂$. We prove $Γ ⊢ φ₁$ and $Γ ⊢ φ₂$.
From the conjunction of φ₁ and φ₂, the expected result follows.

\begin{equation*}
(\mathcal{D})\hspace{3mm}
  \begin{bprooftree}
  \AxiomC{$Γ ⊢ \fstrip₁~(φ₁ ∧ φ₂)~(\suc\,\textit{n})$}
  \RightLabel{by~(\ref{eq:strip-fixed})}
  \UnaryInfC{$Γ ⊢ \fuh~(\fstrip₁~φ₁~n) ∧ \fuh~(φ₁ ⇒ \fstrip₁~φ₂~n)$}
  \RightLabel{∧-proj₁}
  \UnaryInfC{$Γ ⊢ \fuh~(\fstrip₁~φ₁~n)$}
  \RightLabel{Lemma~\ref{lem:inv-uh-lem}}
  \UnaryInfC{$Γ ⊢ \fstrip₁~φ₁~n$}
  \RightLabel{by~ind.~hyp.}
  \UnaryInfC{$Γ ⊢ φ₁$}
    \end{bprooftree}
\end{equation*}

\begin{equation*}
  \begin{bprooftree}
  \AxiomC{$\mathcal{D}$}
  \UnaryInfC{Γ ⊢ φ₁}
  \AxiomC{$Γ ⊢ \fstrip₁~(φ₁ ∧ φ₂)~(\suc~n)$}
  \RightLabel{by~(\ref{eq:strip-fixed})}
  \UnaryInfC{$Γ ⊢ \fuh~(\fstrip₁~φ₁~n) ∧ \fuh~(φ₁ ⇒ \fstrip₁~φ₂~n)$}
  \RightLabel{$∧$-proj$₂$}
  \UnaryInfC{$Γ ⊢ \fuh~(φ₁ ⇒ \fstrip₁~φ₂~n)$}
  \RightLabel{Lemma~\ref{lem:inv-uh-lem}}
  \UnaryInfC{$Γ ⊢ φ₁ ⇒ \fstrip₁~φ₂~n$}
  \RightLabel{$⇒$-elim}
  \BinaryInfC{$Γ ⊢ \fstrip₁~φ₂~n$}
  \RightLabel{by~ind.~hyp.}
  \UnaryInfC{$Γ ⊢ φ₂$}
  \end{bprooftree}
\end{equation*}

∙~Case $φ ≡ ¬ (φ₁ ∨ φ₂)$.
We prove $Γ ⊢ ¬ φ₁$ and $Γ ⊢ ¬ φ₂$.
From the conjunction of $¬ φ₁$ and $¬ φ₂$ by applying De Morgan Law
and the result follows.

\begin{equation*}
(\mathcal{D})\hspace{3mm}
\begin{bprooftree}
\AxiomC{$Γ ⊢ \fstrip₁~(¬ (φ₁ ∨ φ₂))~(\suc~n)$}
\RightLabel{by~(\ref{eq:strip-fixed})}
\UnaryInfC{$Γ ⊢ \fuh~(\fstrip₁~(¬ φ₁)~n) ∧ \fuh~((¬ φ₁) ⇒ \fstrip₁~(¬ φ₂)~n)$}
\RightLabel{∧-proj₁}
\UnaryInfC{$Γ ⊢ \fuh~(\fstrip₁~(¬ φ₁)~n)$}
\RightLabel{Lemma~\ref{lem:inv-uh-lem}}
\UnaryInfC{$Γ ⊢ \fstrip₁~(¬ φ₁)~n$}
\RightLabel{by~ind.~hyp.}
  \UnaryInfC{$Γ ⊢ ¬φ₁$}
\end{bprooftree}
\end{equation*}
\begin{equation*}
  \begin{bprooftree}
  \AxiomC{$\mathcal{D}$}
  \UnaryInfC{$Γ ⊢ ¬ φ₁$}
  \AxiomC{$Γ ⊢ \fstrip₁~(¬ (φ₁ ∨ φ₂))~(\suc~n)$}
  \RightLabel{by~(\ref{eq:strip-fixed})}
  \UnaryInfC{$Γ ⊢ \fuh~(\fstrip₁~(¬ φ₁)~n) ∧ \fuh~((¬ φ₁) ⇒ \fstrip₁~(¬ φ₂)~n)$}
  \RightLabel{∧-proj₂}
  \UnaryInfC{$Γ ⊢ \fuh~((¬ φ₁) ⇒ \fstrip₁~(¬ φ₂)~n)$}
  \RightLabel{Lemma~\ref{lem:inv-uh-lem}}
  \UnaryInfC{$Γ ⊢ (¬ φ₁) ⇒ \fstrip₁~(¬ φ₂)~n$}
  \RightLabel{⇒-elim}
  \BinaryInfC{$Γ ⊢ \fstrip₁~(¬ φ₂)~n$}
  \RightLabel{by~ind.~hyp.}
  \UnaryInfC{$Γ ⊢ ¬ φ₂$}
  \end{bprooftree}
\end{equation*}
∙~  Other cases are proved in a similar way, see Appendix~\ref{app:strip-proof-case}.

\end{proof} % the proof continue...

The following theorem is convenient to substitute equals by equals in
a theorem. Recall the equality (≡) is symmetric and transitive as well.
We use these properties without any mention.

\begin{mainlemma}
  \label{lem:subst}
  Substitution theorem.
\begin{equation*}
  \label{eq:substitution-theorem}
  \begin{bprooftree}
  \AxiomC{$Γ ⊢ φ$}   \AxiomC{$ψ ≡ φ$}
  \RightLabel{\fsubst}
  \BinaryInfC{$Γ ⊢ ψ$}
  \end{bprooftree}
\end{equation*}
\end{mainlemma}

We can now formulate the result that justifies the stripping strategy
of \Metis to prove goals.
For the sake of brevity, we state the following theorem for the
\strip function when the goal has two subgoals. In other cases,
we extend the theorem in the natural way.

\begin{mainth}
\label{thm:strip}
Let $\text{n} : \Nat$ be the complexity measure of the strip function defined
in~\eqref{eq:strip-fixed}.
Let $s₁$ and $s₂$ be the subgoals of the goal $φ$, that is,
$$\fstrip₁~φ~n~≡~s₁∧~s₂.$$
If $Γ ⊢ s₁$ and $Γ ⊢ s₂$ then $Γ ⊢ φ$.
\end{mainth}

\begin{proof}
\begin{equation*}
  \begin{bprooftree}
  \AxiomC{$ Γ ⊢ s_1 $}
  \AxiomC{$ Γ ⊢ s_2 $}
  \RightLabel{∧-intro}
  \BinaryInfC{$Γ ⊢ s_1\wedge s_2$}
  \AxiomC{$\fstrip₁~φ~n ≡ s_1\wedge s_2 $}
  \RightLabel{\fsubst}
  \BinaryInfC{$Γ ⊢ \fstrip₁~φ~n$}
  \RightLabel{Lemma~\ref{lem:lem-inv-strip}}
  \UnaryInfC{$Γ ⊢ φ$}
\end{bprooftree}
\end{equation*}
\end{proof}

% -------------------------------------------------------------------

Since \Metis proves a conjecture by refutation,
to prove each subgoal, \Metis assumes the negation of it
by using the \negate rule after the \strip inference application
that introduce such a subgoal.

\begin{myexamplenum}
In the following \TSTP derivation, note that the three subgoals
\verb!sg0!, \verb!sg1! and \verb!sg2! are assumed by the \negate
rule in \verb!neg0!, \verb!neg1! and  \verb!neg2! respectively.

\begin{verbatim}
  fof(goal, conjecture, (p ∧ r ∧ q)).
  fof(sg0, p, inf(strip,[goal])).
  fof(sg1, p ⇒ r, inf(strip,[goal])).
  fof(sg2, (p ∧ r) ⇒ q, inf(strip,[goal])).
  fof(neg0, ¬ p, inf(negate,[sg0])).
  ...
  fof(neg1, ¬ (p ⇒ r), inf(negate,[sg1])).
  ...
  fof(neg2, ¬ ((p ∧ r) ⇒ q), inf(negate,[sg2])).
\end{verbatim}
\end{myexamplenum}

\end{document}
