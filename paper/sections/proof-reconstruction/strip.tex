\documentclass[../../main.tex]{subfiles}
\begin{document}

\subsubsection{Stripping a goal.}
\label{sssec:strip-a-goal}

To prove a goal, \Metis splits the goal into
disjoint cases. This process produces a list of new subgoals, the
conjunction of these subgoals implies the goal. Then, a proof of the
goal becomes, in smaller proofs, one refutation for each subgoal.
These subgoals are introduced in the \TSTP derivation with the \strip
inference rule as we show in the following excerpt.

\begin{verbatim}
fof(goal, conjecture, ((p & r) & q)).
fof(subgoal_0, plain, p,
    inference(strip, [], [goal])).
fof(subgoal_1, plain, p => r,
    inference(strip, [], [goal])).
fof(subgoal_2, plain, (p & r) => q,
    inference(strip, [], [goal])).
\end{verbatim}

To prove the correctness of the process mentioned above,
we need to establish some lemmas before introduce our main goal of this
part, Theorem~\ref{thm:thm-strip}.


\begin{mainlemma}
  \label{lem:inv-uh-lem}
  If $Γ ⊢ \fuh\ φ$ then $Γ ⊢ φ$.
\end{mainlemma}

\begin{sketchproof}
Use induction on the cases defined by the outcome of the
$\fuh$ function in Eq.~(\ref{eq:uh-structured}).
\begin{itemize}
  \item If $n = 0$, by definition we conclude $Γ ⊢ φ$.
  \item If $n = 1$, we apply the following theorem proved in~\cite{AgdaProp},
    \begin{equation*}
    \texttt{∧⇒\rm{-to-}⇒⇒}\ :\  Γ ⊢ (φ₁ ∧ φ₂) ⇒ φ₃ → Γ ⊢ φ₁ ⇒ (φ₂ ⇒ φ₃).
    \end{equation*}
  \item For $n > 1$, we use induction on the structure of the second
        argument.
\vskip 2mm
\begin{itemize}

\item Case $φ ≡ φ₁ ⇒ (φ₂ ⇒ φ₃)$.
\begin{equation*}
  \begin{bprooftree}
  \AxiomC{$Γ ⊢ \fuh(φ₁ ⇒ (φ₂ ⇒ φ₃), n)$}
  \RightLabel{by def.}
  \UnaryInfC{$Γ ⊢ \fuh(φ₁ ⇒ (φ₂ ⇒ φ₃), n-1)$}
  \RightLabel{lem-inv-uh}
  \UnaryInfC{Γ ⊢ (φ₁ ∧ φ₂) ⇒ φ₃}
  \RightLabel{∧⇒\rm{-to-}⇒⇒.}
  \UnaryInfC{Γ ⊢ φ₁ ⇒ (φ₂ ⇒ φ₃)}
  \end{bprooftree}
\end{equation*}

\item Case $φ ≡ φ₁ ⇒ (φ₂ ∧ φ₃)$.
\begin{equation*}
  \scalebox{0.95}{
  ($\mathcal{D}_1$)
  \begin{bprooftree}
    \AxiomC{$Γ ⊢ \fuh(φ₁ ⇒ φ₂, n) ∧ \fuh(φ₁ ⇒ φ₃, n)$}
    \RightLabel{∧-proj₁}
    \UnaryInfC{$Γ ⊢ \fuh(φ₁ ⇒ φ₂, n)$}
    \RightLabel{by def.}
    \UnaryInfC{$Γ ⊢ \fuh(φ₁ ⇒ φ₂, n-1)$}
    \RightLabel{lem-inv-uh,}
    \UnaryInfC{$Γ ⊢ φ₁ ⇒ φ₂$}
  \end{bprooftree}
  }
\end{equation*}

\begin{equation*}
  (\mathcal{D}_2)
  \begin{bprooftree}
    \AxiomC{$Γ ⊢ \fuh(φ₁ ⇒ φ₂) ∧ \fuh(φ₁ ⇒ φ₃)$}
    \RightLabel{∧-proj₂}
    \UnaryInfC{$Γ ⊢ \fuh(φ₁ ⇒ φ₃, n)$}
    \RightLabel{by def.}
    \UnaryInfC{$Γ ⊢ \fuh(φ₁ ⇒ φ₃, n-1)$}
    \RightLabel{lem-inv-uh.}
    \UnaryInfC{$Γ ⊢ φ₁ ⇒ φ₃$}
    \end{bprooftree}
\end{equation*}

Finally, using the theorem \texttt{⇒∧⇒\rm{-to-}⇒∧} from~\cite{AgdaProp},
\begin{equation*}
  \texttt{⇒∧⇒\rm{-to-}⇒∧}\ :\ Γ ⊢ (φ₁ ⇒ φ₂) ∧ (φ₁ ⇒ φ₃) → Γ ⊢ φ₁ ⇒ (φ₂ ∧ φ₃),
\end{equation*}
\begin{equation*}
  \begin{bprooftree}
  \AxiomC{$\mathcal{D}_1$}
  \AxiomC{$\mathcal{D}_2$}
  \RightLabel{∧-intro}
  \BinaryInfC{$Γ ⊢ (φ₁ ⇒ φ₂) ∧ (φ₁ ⇒ φ₃)$}
  \RightLabel{\tt ⇒∧⇒\rm{-to-}⇒∧.}
  \UnaryInfC{Γ ⊢ φ₁ ⇒ (φ₂ ∧ φ₃)}
  \end{bprooftree}
\end{equation*}
\item Other cases are proved in a similar way.
\end{itemize}
\end{itemize}
\end{sketchproof}

We now define the $\fstrip$ function that yields the conjunction of
subgoals that implies the goal of the problem in the \Metis \TSTP derivations.

\begin{definition}[strip]
\label{def:strip}
\label{eq:strip}

\begin{equation}
\begin{aligned}
&\hspace{.495mm}\fstrip : \Prop → \Nat → \Prop\\
&\begin{array}{lll}
\fstrip &(φ₁ ∧ φ₂)     ~(\suc~n) &= \fuh~(\fstrip~φ₁~n) ∧ \fuh~(φ₁ ⇒ \fstrip~φ₂~n)\\
\fstrip &(φ₁ ∨ φ₂)     ~(\suc~n) &= \fuh~(¬ φ₁ ⇒ \fstrip~φ₂~n)\\
\fstrip &(φ₁ ⇒ φ₂)     ~(\suc~n) &= \fuh~(φ₁ ⇒ \fstrip~φ₂~n)\\
\fstrip &(¬ (φ₁ ∧ φ₂)) ~(\suc~n) &= \fuh~(φ₁ ⇒ \fstrip~¬ φ₂~n)\\
\fstrip &(¬ (φ₁ ∨ φ₂)) ~(\suc~n) &= \fuh~(\fstrip~¬ φ₁~n) ∧ \fuh~(¬ φ₁ ⇒ \fstrip~¬ φ₂~n)\\
\fstrip &(¬ (φ₁ ⇒ φ₂)) ~(\suc~n) &= \fuh~(\fstrip~φ₁~n) ∧ \fuh~(φ₁ ⇒ \fstrip~¬ φ₂~n)\\
\fstrip &(¬ ¬ φ₁)      ~(\suc~n) &= \fuh~(\fstrip~φ₁~n)\\
\fstrip &(¬ ⊥)         ~(\suc~n) &= ⊤\\
\fstrip &(¬ ⊤)         ~(\suc~n) &= ⊥\\
\fstrip &φ             ~n &= φ
\end{array}
\end{aligned}
\end{equation}
\end{definition}

% The definition above has termination problems since it is not
% a structural recursion. Therefore, we have reformulated the equation
% above to solve such an issue following the technique
% described in Section~\ref{ssec:structural-recursion}.
% We refer to $\fstrip$ as the function after applying the process mentioned
% above here and below unless otherwise stated.

\begin{lemma}[lem-inv-strip]
If $Γ ⊢ \fstrip~φ$ then $Γ ⊢ φ$.
\label{lem:lem-inv-strip}
\end{lemma}

\begin{sketchproof}
The proof is by induction on the structure of the
formula~$φ$ by following the cases in Eq.~(\ref{eq:strip}).
We present some cases but for a complete proof, we refer the reader to~\cite{AgdaMetis}.

\begin{itemize}
\item Case $φ ≡ ¬ ¬ φ₁$.
\begin{equation*}
  \begin{bprooftree}
 \AxiomC{$Γ ⊢ \fstrip~(¬¬φ₁)~(\suc~n)$}
  \RightLabel{by def.}
  \UnaryInfC{$Γ ⊢ \fuh~(\fstrip~φ₁~n)$}
  \RightLabel{lem-inv-uh}
  \UnaryInfC{$Γ ⊢ \fstrip~φ₁~n$}
  \RightLabel{by hip.}
  \UnaryInfC{$Γ ⊢ φ₁$}
  \end{bprooftree}
\end{equation*}

\item Case $φ ≡ φ₁ ∧ φ₂$. We get a proof for each conjunct and using the
introduction rule for conjunction connective, the result follows.

\begin{equation*}
(\mathcal{D})\hspace{3mm}
  \begin{bprooftree}
  \AxiomC{$Γ ⊢ \fstrip~(φ₁ ∧ φ₂)~(\suc~n)$}
  \RightLabel{by def.}
  \UnaryInfC{$Γ ⊢ \fuh~(\fstrip~φ₁~n) ∧ \fuh~(φ₁ ⇒ \fstrip~φ₂~n)$}
  \RightLabel{∧-proj₁}
  \UnaryInfC{$Γ ⊢ \fuh~(\fstrip~φ₁~n)$}
  \RightLabel{lem-inv-uh}
  \UnaryInfC{$Γ ⊢ \fstrip~φ₁~n$}
  \RightLabel{by hip.}
  \UnaryInfC{$Γ ⊢ φ₁$}
    \end{bprooftree}
\end{equation*}

\begin{equation*}
  \begin{bprooftree}
  \AxiomC{$\mathcal{D}$}
  \UnaryInfC{Γ ⊢ φ₁}
  \AxiomC{$Γ ⊢ \fstrip~(φ)~(\suc~n)$}
  \RightLabel{by def.}
  \UnaryInfC{$Γ ⊢ \fuh~(\fstrip~φ₁~n) ∧ \fuh~(φ₁ ⇒ \fstrip~φ₂~n)$}
  \RightLabel{$∧$-proj$₂$}
  \UnaryInfC{$Γ ⊢ \fuh~(φ₁ ⇒ \fstrip~φ₂~n)$}
  \RightLabel{lem-inv-uh}
  \UnaryInfC{$Γ ⊢ φ₁ ⇒ \fstrip~φ₂~n$}
  \RightLabel{$⇒$-elim}
  \BinaryInfC{$Γ ⊢ \fstrip~φ₂~n$}
  \RightLabel{by hip.}
  \UnaryInfC{$Γ ⊢ φ₂$}
  \end{bprooftree}
\end{equation*}

\item Case $φ ≡ φ₁ ⇒ φ₂$.
\begin{equation*}
  \begin{bprooftree}
  \AxiomC{}
  \RightLabel{assume φ₁}
  \UnaryInfC{$Γ , φ₁ ⊢ φ₁$}
  \AxiomC{$Γ ⊢ \fstrip~(φ)~(\suc~n)$}
  \RightLabel{by def.}
  \UnaryInfC{$Γ ⊢ \fuh~(φ₁ ⇒ \fstrip~φ₂~n)$}
  \RightLabel{lem-inv-uh}
  \UnaryInfC{$Γ ⊢ φ₁ ⇒ \fstrip~φ₂~n$}
  \RightLabel{weaken φ₁}
  \UnaryInfC{$Γ , φ₁ ⊢ φ₁ ⇒ \fstrip~φ₂~n$}
  \RightLabel{⇒-elim}
  \BinaryInfC{$Γ , φ₁ ⊢ \fstrip~φ₂~n$}
  \RightLabel{by hip}
  \UnaryInfC{$Γ , φ₁ ⊢ φ₂$}
  \RightLabel{⇒-intro.}
  \UnaryInfC{$Γ ⊢ φ₁ ⇒ φ₂$}
  \end{bprooftree}
\end{equation*}

\item Case $φ ≡ ¬ (φ₁ ∨ φ₂)$. We show that $Γ ⊢ ¬ φ₁$ and $Γ ⊢ ¬ φ₂$.
The result follows by introducing the conjunction and applying
De Morgan Laws.

\begin{equation*}
(\mathcal{D})\hspace{3mm}
\begin{bprooftree}
\AxiomC{$Γ ⊢ \fstrip~(φ)~(\suc~n)$}
\RightLabel{by def.}
\UnaryInfC{$Γ ⊢ \fuh~(\fstrip~¬ φ₁~n) ∧ \fuh~(¬ φ₁ ⇒ \fstrip~¬ φ₂~n)$}
\RightLabel{∧-proj₁}
\UnaryInfC{$Γ ⊢ \fuh~(\fstrip~¬ φ₁~n)$}
\RightLabel{lem-inv-uh,}
\UnaryInfC{$Γ ⊢ \fstrip~¬ φ₁~n$}
\RightLabel{by hip.}
  \UnaryInfC{$Γ ⊢ ¬φ₁$}
\end{bprooftree}
\end{equation*}
\end{itemize}
\end{sketchproof}

\begin{equation*}
  \begin{bprooftree}
  \AxiomC{$\mathcal{D}$}
  \UnaryInfC{$Γ ⊢ ¬ φ₁$}
  \AxiomC{$Γ ⊢ \fstrip~(φ)~(\suc~n)$}
  \RightLabel{by def.}
  \UnaryInfC{$Γ ⊢ \fuh~(\fstrip~¬ φ₁~n) ∧ \fuh~(¬ φ₁ ⇒ \fstrip~¬ φ₂~n)$}
  \RightLabel{∧-proj₂}
  \UnaryInfC{$Γ ⊢ \fuh~(¬ φ₁ ⇒ \fstrip~¬ φ₂~n)$}
  \RightLabel{lem-inv-uh}
  \UnaryInfC{$Γ ⊢ ¬ φ₁ ⇒ \fstrip~¬ φ₂~n$}
  \RightLabel{⇒-elim}
  \BinaryInfC{$Γ ⊢ \fstrip~¬ φ₂~n$}
  \RightLabel{by hip.}
  \UnaryInfC{$Γ ⊢ ¬ φ₂$}
  \end{bprooftree}
\end{equation*}

\begin{lemma}[subst]
  \label{lem:subst}
  Substitution theorem.
\begin{equation*}
  \label{eq:substitution-theorem}
  \begin{bprooftree}
  \AxiomC{$Γ ⊢ φ$}   \AxiomC{$ψ ≡ φ$}
  \RightLabel{subst.}
  \BinaryInfC{$Γ ⊢ ψ$}
  \end{bprooftree}
\end{equation*}
\end{lemma}

We can now formulate the result that justifies the stripping strategy
of \Metis to prove goals.

\begin{mainth}[thm-strip]
\label{thm:thm-strip}
 If $s_1$ and $s_2$ are the subgoals of the goal $φ$, $Γ ⊢ s_1$, and $Γ ⊢ s_2$ then $Γ ⊢ φ$.
\end{mainth}

\begin{proof}
\begin{equation*}
  \begin{bprooftree}
  \AxiomC{$ Γ ⊢ s_1 $}
  \RightLabel{∧-intro}
  \UnaryInfC{$Γ ⊢ s_1\wedge s_2$}
  \AxiomC{$\fstrip~φ~n ≡ s_1\wedge s_2 $}
  \RightLabel{subst}
  \BinaryInfC{$Γ ⊢ \fstrip~φ~n$}
  \RightLabel{lem-inv-strip.}
  \UnaryInfC{$Γ ⊢ φ$}
\end{bprooftree}
\end{equation*}
\end{proof}

% -------------------------------------------------------------------

\end{document}
