\documentclass[../../main.tex]{subfiles}
\begin{document}

\subsubsection{Stripping a goal.}
\label{sssec:strip-a-goal}

To prove a goal, \Metis splits the goal into
disjoint cases. This process produces a list of new subgoals, the
conjunction of these subgoals implies the goal. Then, a proof of the
goal becomes, in smaller proofs, one refutation for each subgoal.
These subgoals are introduced in the \TSTP derivation with the \strip
inference rule as we show in the following excerpt.

\begin{verbatim}
fof(goal, conjecture, (p & r) & q)).
fof(sg0, p, inf(strip, [goal])).
fof(sg1, p => r, inf(strip, [goal])).
fof(sg2, (p & r) => q, inf(strip, [goal])).
\end{verbatim}

To prove the correctness of the process mentioned above,
we need to establish some lemmas before introduce our main goal of this
part, Theorem~\ref{thm:strip}.

\begin{mainlemma}
  \label{lem:inv-uh-lem}
Let $\text{n} : \Nat$ be the complexity measure of the \rm{\fuh₀} function
in~(\ref{eq:strip-unfixed}).
If $Γ ⊢ \fuh₁~φ~n$ then $Γ ⊢ φ$.
\end{mainlemma}

\begin{proof}
Use induction on the cases defined by the outcome of the
$\fuh$ function in~(\ref{eq:uh-structured}).
\begin{itemize}
  \item If $n = 0$, by definition we conclude $Γ ⊢ φ$.
  \item If $n = 1$, we apply the following theorem proved in~\cite{AgdaProp},
    \begin{equation*}
    \texttt{∧⇒\rm{-to-}⇒⇒}\ :\  Γ ⊢ (φ₁ ∧ φ₂) ⇒ φ₃ → Γ ⊢ φ₁ ⇒ (φ₂ ⇒ φ₃).
    \end{equation*}
  \item For $n > 1$, we use induction on the structure of the second
        argument.
\vskip 2mm
\begin{itemize}

\item Case $φ ≡ φ₁ ⇒ (φ₂ ⇒ φ₃)$.
\begin{equation*}
  \begin{bprooftree}
  \AxiomC{$Γ ⊢ \fuh~(φ₁ ⇒ (φ₂ ⇒ φ₃))~(\suc~n)$}
  \RightLabel{by def.}
  \UnaryInfC{$Γ ⊢ \fuh~((φ₁ ∧ φ₂) ⇒ φ₃)~n$}
  \RightLabel{by hip.}
  \UnaryInfC{Γ ⊢ (φ₁ ∧ φ₂) ⇒ φ₃}
  \RightLabel{∧⇒\rm{-to-}⇒⇒}
  \UnaryInfC{Γ ⊢ φ₁ ⇒ (φ₂ ⇒ φ₃)}
  \end{bprooftree}
\end{equation*}

\item Case $φ ≡ φ₁ ⇒ (φ₂ ∧ φ₃)$.
\begin{equation*}
  \scalebox{0.95}{
  ($\mathcal{D}_1$)
  \begin{bprooftree}
    \AxiomC{$Γ ⊢ \fuh~(φ₁ ⇒ φ₂)~n ∧ \fuh~(φ₁ ⇒ φ₃)~n$}
    \RightLabel{∧-proj₁}
    \UnaryInfC{$Γ ⊢ \fuh~(φ₁ ⇒ φ₂)~n$}
    \RightLabel{by hip.}
    \UnaryInfC{$Γ ⊢ φ₁ ⇒ φ₂$}
  \end{bprooftree}
  }
\end{equation*}

\begin{equation*}
  (\mathcal{D}_2)
  \begin{bprooftree}
    \AxiomC{$Γ ⊢ \fuh~(φ₁ ⇒ φ₂)~n ∧ \fuh~(φ₁ ⇒ φ₃)~n$}
    \RightLabel{∧-proj₂}
    \UnaryInfC{$Γ ⊢ \fuh~(φ₁ ⇒ φ₃)~n$}
    \RightLabel{by hip.}
    \UnaryInfC{$Γ ⊢ φ₁ ⇒ φ₃$}
    \end{bprooftree}
\end{equation*}

Finally, using the theorem \texttt{⇒∧⇒\rm{-to-}⇒∧} from~\cite{AgdaProp},
\begin{equation*}
  \texttt{⇒∧⇒\rm{-to-}⇒∧}\ :\ Γ ⊢ (φ₁ ⇒ φ₂) ∧ (φ₁ ⇒ φ₃) → Γ ⊢ φ₁ ⇒ (φ₂ ∧ φ₃),
\end{equation*}
\begin{equation*}
  \begin{bprooftree}
  \AxiomC{$\mathcal{D}_1$}
  \AxiomC{$\mathcal{D}_2$}
  \RightLabel{∧-intro}
  \BinaryInfC{$Γ ⊢ (φ₁ ⇒ φ₂) ∧ (φ₁ ⇒ φ₃)$}
  \RightLabel{\tt ⇒∧⇒\rm{-to-}⇒∧}
  \UnaryInfC{Γ ⊢ φ₁ ⇒ (φ₂ ∧ φ₃)}
  \end{bprooftree}
\end{equation*}
\item Other cases are proved in a similar way.
\end{itemize}
\end{itemize}
\end{proof}

We now define the $\fstripp$ function in~(\ref{eq:strip-unfixed}) that
yields the conjunction of subgoals that implies the goal of the problem in
the \Metis \TSTP derivations.

\begin{equation}
\label{eq:strip-unfixed}
\begin{aligned}
&\hspace{.495mm}\fstripp : \Prop → \Prop\\
&\begin{array}{lll}
\fstripp &(φ₁ ∧ φ₂)     &= \fuh~(\fstripp~φ₁) ∧ \fuh~(φ₁ ⇒ \fstripp~φ₂)\\
\fstripp &(φ₁ ∨ φ₂)     &= \fuh~(¬ φ₁ ⇒ \fstripp~φ₂)\\
\fstripp &(φ₁ ⇒ φ₂)     &= \fuh~(φ₁ ⇒ \fstripp~φ₂)\\
\fstripp &(¬ (φ₁ ∧ φ₂)) &= \fuh~(φ₁ ⇒ \fstripp~¬ φ₂)\\
\fstripp &(¬ (φ₁ ∨ φ₂)) &= \fuh~(\fstripp~¬ φ₁) ∧ \fuh~(¬ φ₁ ⇒ \fstripp~¬ φ₂)\\
\fstripp &(¬ (φ₁ ⇒ φ₂)) &= \fuh~(\fstripp~φ₁) ∧ \fuh~(φ₁ ⇒ \fstripp~¬ φ₂)\\
\fstripp &(¬ ¬ φ₁)      &= \fuh~(\fstripp~φ₁)\\
\fstripp &(¬ ⊥)         &= ⊤\\
\fstripp &(¬ ⊤)         &= ⊥\\
\fstripp &φ             &= φ
\end{array}
\end{aligned}
\end{equation}

The function defined above has termination problems since it is not
a structural recursion. Therefore, we have reformulated it
in~(\ref{eq:strip-fixed}) in order to solve such an issue following
the technique described in Section~\ref{ssec:structural-recursion}.

%  1137  git add .
%  1138  git commit -am "[ paper ] strip proof."
%  1139  git pull origin paper
%  1140  git push origin paper
\begin{equation}
\label{eq:strip-fixed}
\begin{aligned}
&\hspace{.495mm}\fstrip : \Prop → \Nat → \Prop\\
&\begin{array}{llll}
\fstrip &(φ₁ ∧ φ₂)     &(\suc~n) &= \fuh~(\fstrip~φ₁~n) ∧ \fuh~(φ₁ ⇒ \fstrip~φ₂~n)\\
\fstrip &(φ₁ ∨ φ₂)     &(\suc~n) &= \fuh~(¬ φ₁ ⇒ \fstrip~φ₂~n)\\
\fstrip &(φ₁ ⇒ φ₂)     &(\suc~n) &= \fuh~(φ₁ ⇒ \fstrip~φ₂~n)\\
\fstrip &(¬ (φ₁ ∧ φ₂)) &(\suc~n) &= \fuh~(φ₁ ⇒ \fstrip~¬ φ₂~n)\\
\fstrip &(¬ (φ₁ ∨ φ₂)) &(\suc~n) &= \fuh~(\fstrip~¬ φ₁~n) ∧ \fuh~(¬ φ₁ ⇒ \fstrip~¬ φ₂~n)\\
\fstrip &(¬ (φ₁ ⇒ φ₂)) &(\suc~n) &= \fuh~(\fstrip~φ₁~n) ∧ \fuh~(φ₁ ⇒ \fstrip~¬ φ₂~n)\\
\fstrip &(¬ ¬ φ₁)      &(\suc~n) &= \fuh~(\fstrip~φ₁~n)\\
\fstrip &(¬ ⊥)         &(\suc~n) &= ⊤\\
\fstrip &(¬ ⊤)         &(\suc~n) &= ⊥\\
\fstrip &φ             &\text{n} &= φ
\end{array}
\end{aligned}
\end{equation}

\begin{mainlemma}
\label{lem:lem-inv-strip}
Let $\text{n} : \Nat$ be the complexity measure of the strip function
in~(\ref{eq:strip-unfixed}).
If $Γ ⊢ \fstrip~φ~n$ then $Γ ⊢ φ$.
\end{mainlemma}

\begin{proof}
The proof is by induction on the structure of the
formula~$φ$ by following the cases in~(\ref{eq:strip-fixed}).
We present a straightforward case with double negation, a case with
conjunction connective, and last, the case with a negated disjunction.
We refer the reader to~\cite{AgdaMetis} for the complete proof in \Agda.

$\bullet$~Case $φ ≡ ¬ ¬ φ₁$.
\begin{equation*}
  \begin{bprooftree}
 \AxiomC{$Γ ⊢ \fstrip~(¬¬φ₁)~(\suc~n)$}
  \RightLabel{by~(\ref{eq:uh-fixed})}
  \UnaryInfC{$Γ ⊢ \fuh~(\fstrip~φ₁~n)$}
  \RightLabel{Lemma~\ref{lem:inv-uh-lem}}
  \UnaryInfC{$Γ ⊢ \fstrip~φ₁~n$}
  \RightLabel{by hip.}
  \UnaryInfC{$Γ ⊢ φ₁$}
  \end{bprooftree}
\end{equation*}

$\bullet$~Case $φ ≡ φ₁ ∧ φ₂$. We get a proof for each conjunct and using the
introduction rule for conjunction connective, the expected result follows.

\begin{equation*}
(\mathcal{D})\hspace{3mm}
  \begin{bprooftree}
  \AxiomC{$Γ ⊢ \fstrip~(φ₁ ∧ φ₂)~(\suc~n)$}
  \RightLabel{by~(\ref{eq:strip-fixed})}
  \UnaryInfC{$Γ ⊢ \fuh~(\fstrip~φ₁~n) ∧ \fuh~(φ₁ ⇒ \fstrip~φ₂~n)$}
  \RightLabel{∧-proj₁}
  \UnaryInfC{$Γ ⊢ \fuh~(\fstrip~φ₁~n)$}
  \RightLabel{Lemma~\ref{lem:inv-uh-lem}}
  \UnaryInfC{$Γ ⊢ \fstrip~φ₁~n$}
  \RightLabel{by hip.}
  \UnaryInfC{$Γ ⊢ φ₁$}
    \end{bprooftree}
\end{equation*}

\begin{equation*}
  \begin{bprooftree}
  \AxiomC{$\mathcal{D}$}
  \UnaryInfC{Γ ⊢ φ₁}
  \AxiomC{$Γ ⊢ \fstrip~(φ₁ ∧ φ₂)~(\suc~n)$}
  \RightLabel{by~(\ref{eq:strip-fixed})}
  \UnaryInfC{$Γ ⊢ \fuh~(\fstrip~φ₁~n) ∧ \fuh~(φ₁ ⇒ \fstrip~φ₂~n)$}
  \RightLabel{$∧$-proj$₂$}
  \UnaryInfC{$Γ ⊢ \fuh~(φ₁ ⇒ \fstrip~φ₂~n)$}
  \RightLabel{Lemma~\ref{lem:inv-uh-lem}}
  \UnaryInfC{$Γ ⊢ φ₁ ⇒ \fstrip~φ₂~n$}
  \RightLabel{$⇒$-elim}
  \BinaryInfC{$Γ ⊢ \fstrip~φ₂~n$}
  \RightLabel{by hip.}
  \UnaryInfC{$Γ ⊢ φ₂$}
  \end{bprooftree}
\end{equation*}

$\bullet$~Case $φ ≡ ¬ (φ₁ ∨ φ₂)$.
We show the theorems $Γ ⊢ ¬ φ₁$ and $Γ ⊢ ¬ φ₂$.
With the conjunction of $¬ φ₁$ and $¬ φ₂$ by applying De Morgan Law, the
result follows.

\begin{equation*}
(\mathcal{D})\hspace{3mm}
\begin{bprooftree}
\AxiomC{$Γ ⊢ \fstrip~(¬ (φ₁ ∨ φ₂))~(\suc~n)$}
\RightLabel{by~(\ref{eq:strip-fixed})}
\UnaryInfC{$Γ ⊢ \fuh~(\fstrip~¬ φ₁~n) ∧ \fuh~(¬ φ₁ ⇒ \fstrip~¬ φ₂~n)$}
\RightLabel{∧-proj₁}
\UnaryInfC{$Γ ⊢ \fuh~(\fstrip~¬ φ₁~n)$}
\RightLabel{Lemma~\ref{lem:inv-uh-lem}}
\UnaryInfC{$Γ ⊢ \fstrip~¬ φ₁~n$}
\RightLabel{by hip.}
  \UnaryInfC{$Γ ⊢ ¬φ₁$}
\end{bprooftree}
\end{equation*}
\begin{equation*}
  \begin{bprooftree}
  \AxiomC{$\mathcal{D}$}
  \UnaryInfC{$Γ ⊢ ¬ φ₁$}
  \AxiomC{$Γ ⊢ \fstrip~(¬ (φ₁ ∨ φ₂))~(\suc~n)$}
  \RightLabel{by~(\ref{eq:strip-fixed})}
  \UnaryInfC{$Γ ⊢ \fuh~(\fstrip~¬ φ₁~n) ∧ \fuh~(¬ φ₁ ⇒ \fstrip~¬ φ₂~n)$}
  \RightLabel{∧-proj₂}
  \UnaryInfC{$Γ ⊢ \fuh~(¬ φ₁ ⇒ \fstrip~¬ φ₂~n)$}
  \RightLabel{Lemma~\ref{lem:inv-uh-lem}}
  \UnaryInfC{$Γ ⊢ ¬ φ₁ ⇒ \fstrip~¬ φ₂~n$}
  \RightLabel{⇒-elim}
  \BinaryInfC{$Γ ⊢ \fstrip~¬ φ₂~n$}
  \RightLabel{by hip.}
  \UnaryInfC{$Γ ⊢ ¬ φ₂$}
  \end{bprooftree}
\end{equation*}
\end{proof} % the proof continue...

MISSING TEXT

\begin{mainlemma}[subst]
  \label{lem:subst}
  Substitution theorem.
\begin{equation*}
  \label{eq:substitution-theorem}
  \begin{bprooftree}
  \AxiomC{$Γ ⊢ φ$}   \AxiomC{$ψ ≡ φ$}
  \RightLabel{subst}
  \BinaryInfC{$Γ ⊢ ψ$}
  \end{bprooftree}
\end{equation*}
\end{mainlemma}

We can now formulate the result that justifies the stripping strategy
of \Metis to prove goals.
For the sake of brevity, we state the following theorem for the
\strip function when the goal has only two subgoals. In other cases,
we extend that theorem in the natural way.

\begin{mainth}
\label{thm:strip}
If $s_1$ and $s_2$ are the subgoals of the goal $φ$, that is,
$\fstrip~φ~n~≡~s_1\wedge~s_2$, if $Γ ⊢ s_1$ and $Γ ⊢ s_2$ then $Γ ⊢ φ$.
\end{mainth}

\begin{proof}
\begin{equation*}
  \begin{bprooftree}
  \AxiomC{$ Γ ⊢ s_1 $}
  \AxiomC{$ Γ ⊢ s_2 $}
  \RightLabel{∧-intro}
  \BinaryInfC{$Γ ⊢ s_1\wedge s_2$}
  \AxiomC{$\fstrip~φ~n ≡ s_1\wedge s_2 $}
  \RightLabel{subst}
  \BinaryInfC{$Γ ⊢ \fstrip~φ~n$}
  \RightLabel{Lemma~\ref{lem:lem-inv-strip}}
  \UnaryInfC{$Γ ⊢ φ$}
\end{bprooftree}
\end{equation*}
\end{proof}


% -------------------------------------------------------------------

\end{document}
