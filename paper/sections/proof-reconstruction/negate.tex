\documentclass[../../main.tex]{subfiles}
\begin{document}

\subsubsection{Negating subgoals.}
\label{sssec:negate}

Each proof of a subgoal is a refutation, thereby each proof assumes
the negation of its subgoal.
The \negate rule assumes the negation of such subgoals. Then, this rules
is always derived from a \strip inference application as we can see in
the following \TSTP excerpt.

\begin{myexample}
\negate is a rule that only puts the negation symbol
in front of the formula to assume it.

\begin{verbatim}
fof(sg0, (~ p => q), inf(strip, [goal])).
fof(neg0, ~ (~ p => q), inf(negate, [sg0])).
\end{verbatim}

\end{myexample}

To emulate this rule, we only need to introduce the negation
of the subgoal by using $assume$ rule.

\end{document}
