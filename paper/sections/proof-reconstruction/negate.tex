\documentclass[../../main.tex]{subfiles}
\begin{document}

\subsubsection{Negating subgoals.}
\label{sssec:negate}

Each proof of a subgoal is a refutation, thereby each proof assumes
the negation of its subgoal. The \negate rule
assumes the negation of such subgoals, and
appears after applying the \strip inference to the goal.

\begin{verbatim}
fof(goal, conjecture, (p & q)).
fof(subgoal_0, plain, (p), inference(strip, [], [goal])).
fof(subgoal_1, plain, (p => q), inference(strip, [], [goal])).
fof(negate_0_0, plain, (~ p), inference(negate, [], [subgoal_0])).
\end{verbatim}

To emulate this rule, we introduce the negation
of the subgoal by using $assume$ rule from the inferences in
Fig.~\ref{fig:CPL-inference-rules}.


\end{document}
