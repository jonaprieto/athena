\documentclass[../../main.tex]{subfiles}
\begin{document}


\subsubsection{Canonicalize.}
\label{sssec:canonicalize}

% The normalization module in \Metis is the responsible for
% \canonicalize, \clausify, \conjunct and \simplify.

One of the main purposes of the \canonicalize rule consists of adding the axioms to the problem. In this process, some transformations are carried
out by convert a \verb!fof! formula to their clausal normal form.
The process mainly consists of removing redundancies (tautologies or
definitions) and converting the proposition to its normalized
negative normal form, conjunctive normal form, or
definitional normal form.

\begin{definition} %[\abbre{NProp}]
  \NProp is the type for a formula in normalized negative normal form
  in which negations appear only in the literals and the expression is
  in terms only of ($⊥$, $⊤$, $¬$, $∧$, $∨$) connectives.
  We define the \NProp type in a similar way as we did for \Prop type.
\end{definition}

A sequent $Γ ⟝ φ$ represents a theorem where
$Γ$ is a set of \Prop propositions premises, and $φ : \NProp$ is the
conclusion of the theorem. The $⟝$ theorems have the same subset of rules
for the connectives of \NProp type in Fig.~\ref{fig:CPL-inference-rules}.

% -------------------------------------------------------------------
\subsubsection{Redundancy.}
\label{sssec:redundancy}

There are redundancies in a formula when some of the following theorems can apply inside the formula.

  \[
    \begin{bprooftree}
      \AxiomC{$Γ ⊢ φ ∧ ⊥$}
      \UnaryInfC{$Γ ⊢ ⊥$}
    \end{bprooftree}
    \qquad
    \begin{bprooftree}
      \AxiomC{$Γ ⊢ φ ∧ ⊤$}
      \UnaryInfC{$Γ ⊢ φ$}
    \end{bprooftree}
    \qquad
    \begin{bprooftree}
      \AxiomC{$Γ ⊢ φ ∧ ¬ φ$}
      \UnaryInfC{$Γ ⊢ ⊥$}
    \end{bprooftree}
    \qquad
    \begin{bprooftree}
      \AxiomC{$Γ ⊢ φ ∧ φ$}
      \UnaryInfC{$Γ ⊢ φ$}
    \end{bprooftree}
  \]

  \[
    \begin{bprooftree}
      \AxiomC{$Γ ⊢ φ ∨ ⊥$}
      \UnaryInfC{$Γ ⊢ φ$}
    \end{bprooftree}
    \qquad
    \begin{bprooftree}
      \AxiomC{$Γ ⊢ φ ∨ ⊤$}
      \UnaryInfC{$Γ ⊢ ⊤$}
    \end{bprooftree}
    \qquad
    \begin{bprooftree}
      \AxiomC{$Γ ⊢ φ ∨ ¬ φ$}
      \UnaryInfC{$Γ ⊢ ⊤$}
    \end{bprooftree}
    \qquad
    \begin{bprooftree}
      \AxiomC{$Γ ⊢ φ ∨ φ$}
      \UnaryInfC{$Γ ⊢ φ$}
    \end{bprooftree}
  \]

\begin{remark}
Here and below, we assume all formulas to be in right-associative form in
the input of the functions.  To perform such conversion, we can apply
the $\rm{rdisj}$ and $\rm{rconj}$ functions depending on the main connective Eq.~(\ref{def:rdisj}).
\end{remark}

In a disjunction, $φ₁ ∨ φ₂ ∨ \cdots ∨ φₙ$, we say $ψ ∈_{∨} φ$,
if there is some $i = 1, \cdots, n$ such that $ψ ≡ φᵢ$.
Note that $ψ ∈_{∨} φ$ is another name for the equality
$ψ ≡ \fconjunct_{∨}(φ, ψ)$.

  \begin{align*}
  \label{def:rm-or}
    \begin{split}
      &\frm_{∨} :  \NProp \to \NProp\\
      &\frm_{∨}~φ =
      \begin{cases}
        \frm_{∨}~φ₂,    &\text{ if }φ ≡ φ₁ ∨ φ₂\text{ and }φ₁ ∈_{∨} φ₂\\
        φ₁~∨~rm_{∨}~φ₂, &\text{ if }φ ≡ φ₁ ∨ φ₂\\
        φ,  &\text{ otherwise.}
      \end{cases}
    \end{split}
  \end{align*}

\begin{mainlemma}
  \label{lem:rm-or}
  If $Γ ⟝ φ$ then $Γ ⟝ \frm_{∨}(φ)$.
\end{mainlemma}

Now, we have removed redundancies in the disjunctions by applying
$\frm_{∨}$ function, we can define a similar function to work with
conjunctions, \ie, $\frm_{∧}$.

In a conjunction, $φ = φ₁ ∧ φ₂ ∧ \cdots ∧ φₙ$, we say
$ψ ∈_{∧} φ$, if there is some $i = 1, \cdots, n$ such that $ψ = φᵢ$.
We define $ψ ∈_{∧} φ$ as the equality $ψ ≡ \fconjunct(φ, ψ)$.

\label{eq:rm-and}
  \begin{align*}
    \begin{split}
    &\frm_{∧} : \NProp \to \NProp\\
    &\frm_{∧}~φ =
    \begin{cases}
      rm_{∧}~φ₂,      &\text{ if }φ ≡ φ₁ ∧ φ₂\text{ and }φ₁ ∈_{∧} φ₂\\
      φ₁ ∧ rm_{∧}~φ₂, &\text{ if }φ ≡ φ₁ ∧ φ₂\\
      φ,               &\text{ otherwise.}
    \end{cases}
    \end{split}
  \end{align*}

\begin{mainlemma}
  \label{lem:rm-and}
  If $Γ ⟝ φ$ then $Γ ⟝ \frm_{∧}(φ)$.
\end{mainlemma}

\begin{example}
\begin{equation*}
  \begin{bprooftree}
  \AxiomC{$Γ ⟝ φ ∨ φ$}
  \RightLabel{Lemma~\ref{lem:rm-or}}
  \UnaryInfC{$Γ ⟝ φ$}
  \end{bprooftree}
  \qquad
  \begin{bprooftree}
  \AxiomC{$Γ ⟝ φ ∧ φ$}
  \RightLabel{Lemma~\ref{lem:rm-and}}
  \UnaryInfC{$Γ ⟝ φ$}
  \end{bprooftree}
\end{equation*}
\end{example}

Continuing with the rest of tautologies from
Fig.~\ref{fig:conjunctive-disjunctive-simpl}, we define the
$\rm{ndisj}_{∨}$ function and $\rm{ndisj}_{∨}$ function that
take into account the following cases.

\begin{equation*}
\begin{bprooftree}
  \AxiomC{$Γ ⟝ φ ∨ ¬~φ$}
  \RightLabel{lem-ndisj$_{∨}$,}
  \UnaryInfC{$Γ ⟝ ⊤$}
\end{bprooftree}
\begin{bprooftree}
  \AxiomC{$Γ ⟝ φ ∨ ⊤$}
  \RightLabel{lem-ndisj$_{∨}$,}
  \UnaryInfC{$Γ ⟝ ⊤$}
\end{bprooftree}
\begin{bprooftree}
  \AxiomC{$Γ ⟝ φ ∨ ⊥$}
  \RightLabel{lem-ndisj$_{∨}$.}
  \UnaryInfC{$Γ ⟝ φ$}
\end{bprooftree}
\end{equation*}

\begin{equation*}
\begin{bprooftree}
  \AxiomC{$Γ ⟝ φ ∧ ¬~φ$}
  \RightLabel{lem-ndisj$_{∧}$,}
  \UnaryInfC{$Γ ⟝ ⊥$}
\end{bprooftree}
\begin{bprooftree}
  \AxiomC{$Γ ⟝ φ ∧ ⊤$}
  \RightLabel{lem-ndisj$_{∧}$,}
  \UnaryInfC{$Γ ⟝ φ$}
\end{bprooftree}
\begin{bprooftree}
  \AxiomC{$Γ ⟝ φ ∧ ⊥$}
  \RightLabel{lem-ndisj$_{∧}$.}
  \UnaryInfC{$Γ ⟝ ⊥$}
\end{bprooftree}
\end{equation*}

\begin{definition}[ndisj$_{∨}$]
  \label{def:ndisj-or}
  \begin{align*}
    \begin{split}
    &\rm{ndisj}_{∨} : \NProp \to \NProp\\
    &\rm{ndisj}_{∨}(φ) =
      \begin{cases}
        ⊤, &\text{ if }φ ≡ ¬ φ₁ ∨ φ₂,\ φ₁ ∈_{∨} φ₂\\
        ⊤, &\text{ if }φ ≡ ¬ φ₁ ∨ φ₂,\ \rm{ndisj}_{∨}(φ₂) ≡ ⊤\\
        ¬ φ₁ ∨ \rm{ndisj}_{∨}(φ₂), &\text{ if }φ ≡ ¬ φ₁ ∨ φ₂\\
        ⊤, &\text{ if }φ ≡ φ₁ ∨ φ₂,\ ¬φ₁ ∈_{∨} φ₂ \\
        ⊤, &\text{ if }φ ≡ φ₁ ∨ φ₂,\ \rm{ndisj}_{∨}(φ₂) ≡ ⊤\\
        φ₁ ∨ \rm{ndisj}_{∨}(φ₂),  &\text{ if }φ ≡ φ₁ ∨ φ₂\\
        φ, &\text{ otherwise.}
      \end{cases}
    \end{split}
  \end{align*}
\end{definition}

\begin{mainlemma} % (fold)
  \label{lem:lem_ndisj-or}
  If $Γ ⟝ φ$ then $Γ ⟝ \rm{ndisj}_{∨}(φ)$.
\end{mainlemma}

\begin{definition}[ndisj$_{∧}$]
  \label{def:ndisj-and}
  \begin{align*}
    \begin{split}
    &\rm{ndisj}_{∧} : \NProp \to \NProp\\
    &\rm{ndisj}_{∧}(φ) =
      \begin{cases}
        ⊥, &\text{ if }φ ≡ ¬ φ₁ ∧ φ₂,\ φ₁ ∈_{∧} φ₂\\
        ⊥, &\text{ if }φ ≡ ¬ φ₁ ∧ φ₂,\ \rm{ndisj}_{∧}(φ₂) ≡ ⊥\\
        ¬ φ₁ ∧ \rm{ndisj}_{∧}(φ₂), &\text{ if }φ ≡ ¬ φ₁ ∧ φ₂\\
        ⊥, &\text{ if }φ ≡ φ₁ ∧ φ₂,\ φ₁ ∈_{∧} φ₂\\
        ⊥, &\text{ if }φ ≡ φ₁ ∧ φ₂,\ \rm{ndisj}_{∧}(φ₂) ≡ ⊥\\
        φ₁ ∧ \rm{ndisj}_{∧}(φ₂), &\text{ if }φ ≡ φ₁ ∧ φ₂\\
        φ, &\text{ otherwise.}
      \end{cases}
    \end{split}
  \end{align*}
\end{definition}

\begin{mainlemma} % (fold)
  \label{lem:lem_ndisj-and}
  If $Γ ⟝ φ$ then $Γ ⟝ \rm{ndisj}_{∧}(φ)$.
\end{mainlemma}

% subsubsection redundancy (end)
% -------------------------------------------------------------------

We use the functions defined above to remove redundancies in conjunctions
with $\fcanon_{∧}$ function, and $\fcanon_{∨}$ function to remove redundancies
in disjunctions.

\begin{definition}[canon$_{∧}$]
\label{def:canon-and}
\begin{align*}
    \begin{split}
      &\fcanon_{∧} : \NProp \to \NProp\\
      &\fcanon_{∧}(φ) =
        \begin{cases}
          φⱼ, &\text{ if } φ ≡ φ₁ ∧ φ₂,\ φᵢ ≡ ⊤,\text{ for } i,j = 1,2, i≠j\\
          ⊥,  &\text{ if } φ ≡ φ₁ ∧ φ₂,\ φᵢ ≡ ⊥,\text{ for } i,j = 1,2\\
          \frm_∧(\rm{ndisj}_∧(φ)), &\text{ if } φ ≡ φ₁ ∧ φ₂\\
          φ,                &\text{ otherwise.}
        \end{cases}
    \end{split}
\end{align*}
\end{definition}

\begin{mainlemma}
  \label{lem:lem_canon-and}
  If $Γ ⟝ φ$ then $Γ ⟝ \fcanon_{∧}(φ)$.
\end{mainlemma}

\begin{definition}[canon$_{∨}$]
\label{def:canon-or}
\begin{align*}
    \begin{split}
      &\fcanon_{∨} : \NProp \to \NProp\\
      &\fcanon_{∨}(φ) =
        \begin{cases}
        φⱼ, &\text{ if } φ ≡ φ₁ ∨ φ₂,\ φᵢ ≡ ⊥,\text{ for } i,j = 1,2, i≠j\\
        ⊤,  &\text{ if } φ ≡ φ₁ ∨ φ₂,\ φᵢ ≡ ⊤,\text{ for } i,j = 1,2\\
        \frm_∨(\rm{ndisj}_∨(φ)), &\text{ if } φ ≡ φ₁ ∧ φ₂\\
        φ, &\text{ otherwise.}
        \end{cases}
    \end{split}
\end{align*}
\end{definition}

\begin{mainlemma}
  \label{lem:lem_canon-or}
  If $Γ ⟝ φ$ then $Γ ⟝ \fcanon_{∨}(φ)$.
\end{mainlemma}

% \begin{remark}
% A similar treatment defines the $\fcanon_{⊻} : \NProp \to \NProp$.
% This function removes redundancies related with the exclusive
% disjunction connective.
% % Its theorem, thm-canon$_{⊻}$ is presented in~\cite{AgdaMetis}.
% We left out this description for the brevity of this
% document and refer the interested reader to \cite{AgdaMetis} for
% more details.
% \end{remark}

Now, we can translate formulas from \Prop to \NProp by defining
the $\fnnf₀$ function. This function is based on the $\fnnf$ function
described in \citeauthor{Bezem2002}~\cite{Bezem2002}. Their $\fnnf$ function
translates formulas to their negative normal form. In the
reference, the authors avoid a termination problem with a similar
function by using, as an extra argument, the polarity of the formula.
We follow the same idea in Eq.~(\ref{def:nnf-zero}).
Therefore, we define the \abbre{Polarity} type with
two constructors: $⊕$ to denote positive polarity, and $⊖$ to denote
negative polarity of a formula. The polarity of a formula is defined in
Eq.~(\ref{def:polarity}).

% fun polarity True = true
%   | polarity False = false
%   | polarity (Literal (_,(pol,_))) = not pol
%   | polarity (And _) = true
%   | polarity (Or _) = false
%   | polarity (Xor (_,_,pol,_)) = pol
%   | polarity (Exists _) = true
%   | polarity (Forall _) = false;
\begin{equation}
\label{def:polarity}
  \begin{aligned}
  &\hspace{.495mm}\fpolarity : \Prop \to \abbre{Polarity}\\
    &\begin{array}{lll}
      \fpolarity &(φ₁ ∧ φ₂) &= ⊕\\
      \fpolarity &(φ₁ ∨ φ₂) &= ⊖\\
      \fpolarity &(φ₁ ⇒ φ₂) &= ⊖\\
      \fpolarity &(¬ φ)     &=
        \begin{cases}
        ⊕, &\text{ if }\fpolarity~φ=⊖\\
        ⊖, &\text{ if }\fpolarity~φ=⊕
        \end{cases}\\
      \fpolarity &φ     &=⊕
    \end{array}
  \end{aligned}
\end{equation}

\begin{equation}
\label{def:nnf-zero}
  \begin{aligned}
  &\hspace{.495mm}\fnnf₀ : \abbre{Polarity} \to \Prop \to \NProp\\
    &\begin{array}{lll}
      \fnnf₀~⊕~(φ₁ ∧ φ₂) &=\fcanon_{∧}~(\fnnf₀~⊕~φ₁ ∧ \fnnf₀~⊕~φ₂)\\
      \fnnf₀~⊕~(φ₁ ∨ φ₂) &=\fcanon_{∨}~(\fnnf₀~⊕~φ₁ ∨ \fnnf₀~⊕~φ₂)\\
      \fnnf₀~⊕~(φ₁ ⇒ φ₂) &=\fcanon_{∨}~(\fnnf₀~⊖~φ₁ ∨ \fnnf₀~⊕~φ₂)\\
      \fnnf₀~⊕~(¬ φ₁)    &=\fnnf₀~⊖~φ₁                              \\
      \fnnf₀~⊕~φ         &=φ        \\
      \fnnf₀~⊖~(φ₁ ∧ φ₂) &=\fcanon_{∨}(\fnnf₀~⊖,~φ₁ ∨ \fnnf₀~⊖~φ₂)\\
      \fnnf₀~⊖~(φ₁ ∨ φ₂) &=\fcanon_{∧}(\fnnf₀~⊖,~φ₁ ∧ \fnnf₀~⊖~φ₂)\\
      \fnnf₀~⊖~(φ₁ ⇒ φ₂) &=\fcanon_{∧}(\fnnf₀~⊖,~φ₂ ∧ \fnnf₀~⊖~φ₁)\\
      \fnnf₀~⊖~(¬ φ₁)    &=\fnnf₀~⊕~φ₁\\
      \fnnf₀~⊖~⊤         &=⊥\\
      \fnnf₀~⊖~⊥         &=⊤\\
      \fnnf₀~⊖~φ         &=φ
    \end{array}
  \end{aligned}
\end{equation}

\begin{mainlemma}
  \label{lem:lem-nnf}
  If $Γ ⊢ φ$ then $Γ ⟝ \fnnf~φ$ where,
  \begin{align*}
   \begin{split}
     &\fnnf : \NProp \to \NProp\\
     &\fnnf~φ = \fnnf₀(⊕, φ).
   \end{split}
  \end{align*}
\end{mainlemma}

\begin{mainlemma}
  \label{lem:lem-dist-or}
  $Γ ⟝ φ$ then $Γ ⟝ \fdist_{∨}(φ)$, where,
  \begin{align*}
    \begin{split}
    &\fdist_{∨} : \NProp \to \NProp\\
    &\fdist_{∨}(φ) =
      \begin{cases}
        \fdist_{∨}(φ₁ ∨ φ₂) ∧ \fdist_{∨}(φ₂ ∨ φ₃), &\text{ if }φ ≡ (φ₁ ∧ φ₂) ∨ φ₃\\
        \fdist_{∨}(φ₁ ∨ φ₃) ∧ \fdist_{∨}(φ₁ ∨ φ₃), &\text{ if }φ ≡ φ₁ ∨ (φ₂ ∧ φ₃)\\
        φ,                                     &\text{ otherwise.}
      \end{cases}
    \end{split}
  \end{align*}
\end{mainlemma}

\begin{mainlemma}
  \label{lem:lem-dist}
  $Γ ⟝ φ$ then $Γ ⟝ \fdist(φ)$, where,
  \begin{align*}
      \begin{split}
      &\fdist : \NProp \to \NProp\\
      &\fdist(φ) =
        \begin{cases}
          \fdist(φ₁) ∧ \fdist(φ₂),
            &\text{ if }φ ≡ φ₁ ∧ φ₂\\
          \fdist_{∨}(\fdist(φ₁) ∨ \fdist(φ₂)),
            &\text{ if }φ ≡ φ₁ ∨ φ₂\\
          φ, &\text{ otherwise.}
        \end{cases}
      \end{split}
  \end{align*}
\end{mainlemma}

\begin{sketchproof} If $φ ≡ φ₁ ∨ φ₂$, we get for $i = 1, 2$,
\begin{equation*}
  \begin{bprooftree}
    \RightLabel{assume}
    \AxiomC{$Γ, φ_{i} ⊢ φ_{i}$}
    \RightLabel{by hip.}
    \UnaryInfC{$Γ ⊢ \fdist(φ_{i})$}
    \RightLabel{∨-intro$_{1,2}$}
    \UnaryInfC{$Γ, φ_{i} ⊢ \fdist(φ₁) ∨ \fdist(φ₂)$}
    \RightLabel{lemma~\ref{lem:lem-dist-or}}
    \UnaryInfC{$Γ, φ_{i} ⊢ \fdist_{∨}(\fdist(φ₁) ∨ \fdist(φ₂))$}
    \end{bprooftree}
\end{equation*}
Then, by using the ∨-elim rule in the last derivation above, we
derive $Γ, φ₁ ∨ φ₂ ⟝ \fdist_{∨}(\fdist(φ₁) ∨ \fdist(φ₂))$ and the lemma
follows.
\end{sketchproof}

To convert normalized propositions to classical propositions, we
make the following definition, completing the translation between
these two types.

\begin{definition}[form]
\begin{align*}
    \begin{split}
      &\fform : \NProp \to \Prop\\
      &\fform(φ) =
      \begin{cases}
        \fform(φ₁) ∧ \fform(φ₂), &\text{ if }φ ≡ φ₁ ∧ φ₂\\
        \fform(φ₁) ∨ \fform(φ₂), &\text{ if }φ ≡ φ₁ ∨ φ₂\\
        \fform_{⊻}(φ₂),        &\text{ if }φ ≡ φ₁ ⊻ φ₂\\
        φ, &\text{ otherwise.}\\
      \end{cases}
    \end{split}
\end{align*}
\end{definition}

\begin{mainlemma}
  \label{lem:lem-form}
   If $Γ ⟝ φ$ then $Γ ⊢ \fform(φ)$.
\end{mainlemma}

As a result, we can get a conjunctive normal form by applying the
following theorem.

\begin{mainth}
\label{thm:thm-cnf}
  If $Γ ⊢ φ$ then $Γ ⊢ \rm{cnf}(φ)$, where,
  \begin{align*}
    \begin{split}
    &\rm{cnf} : \Prop \to \Prop\\
    &\rm{cnf} \varphi = \fform (\fdist (\fnnf  \varphi))
    \end{split}
  \end{align*}
\end{mainth}

\begin{proof}
  Composition of the Lemmas \ref{lem:lem-form}, \ref{lem:lem-dist}
  and \ref{lem:lem-nnf}.
\end{proof}

Since all the previous transformations came from  equivalences in  the
classical propositional logic, the following theorem plays an important role
in the next section.

\begin{mainth}
\label{thm:invCnfThm}
  If $Γ ⊢ \rm{cnf}(φ)$ then $Γ ⊢ φ$.
\end{mainth}

The \canonicalize rule is an overloaded inference rule which performs
normalization of the proposition input.
This rule converts a \texttt{fof} %TODO
formula in clausal form to a \CNF clause but also it performs some
simplifications in the process (e.g., removing ⊤ and ⊥ in the formula).

Since this rule mostly consists of dealing with \CNF clauses, to
justify its reasoning, our strategy mainly consists of checking the
equality of normalized negative form between the source and the
target formula. If it fails, we try to reorder the conjunctive
normal form of the source formula to match with
the conjunctive normal form of the target formula.

\begin{definition}[canonicalize]
\begin{equation}
  \label{def:canonicalize}
  \begin{aligned}
  &\hspace{.495mm}\fcanonicalize : \Prop \to \Prop \to \Prop\\
  &\begin{array}{lll}
        \fcanonicalize~φ~φ &= ψ \\
        \fcanonicalize~φ~\fform(\fnnf~φ) &= \fform(\fnnf~φ) \\
        ψ, \text{ if }\fcnf~ψ ≡ \freorder_{∧∨}(\fcnf~φ, \fcnf~ψ)\\
        \fcanonicalize~φ~ψ &= φ
      \end{array}
    \end{aligned}
  \end{equation}
\end{definition}

\begin{mainth} % (fold)
  \label{thm:canonicalizeThm}
  \hspace{4mm}\\
  If $Γ ⊢ φ$ and $ψ : \Prop$ then $Γ ⊢ \fcanonicalize(φ, ψ)$.
\end{mainth}

\begin{proof}\hspace{3mm}
\begin{itemize}
\item If $φ ≡ ψ$ by substitution theorem we conclude $Γ ⊢ ψ$.
\item If $ψ ≡ \fform(\fnnf~φ)$,
\begin{equation*}
  \begin{bprooftree}
    \AxiomC{$Γ ⊢ φ$}
    \RightLabel{lem-nnf}
    \UnaryInfC{$Γ ⊢₂~\fnnf~φ$}
    \RightLabel{lem-form}
    \UnaryInfC{$Γ ⊢ \fform(\fnnf~φ)$}
    \AxiomC{$ψ ≡ \fform(\fnnf~φ)$}
    \RightLabel{subst.}
    \BinaryInfC{$Γ ⊢ ψ$}
  \end{bprooftree}
\end{equation*}

\item If $γ = \fcnf~ψ ≡ \freorder_{∧∨}~(\fcnf~φ)~(\fcnf~ψ)$,
  \begin{equation*}
    \begin{bprooftree}
      \AxiomC{$Γ ⊢ φ$}
      \RightLabel{thm-cnf}
      \UnaryInfC{$Γ ⊢ \fcnf~φ$}
      \RightLabel{thm-reorder$_{∧∨}$}
      \UnaryInfC{$Γ ⊢ \freorder_{∧∨}(\fcnf~φ, \fcnf~ψ)$}
      \AxiomC{$γ$}
      \RightLabel{subst}
      \BinaryInfC{$Γ ⊢ \fcnf~ψ$}
      \RightLabel{thm-inv-cnf.}
      \UnaryInfC{$Γ ⊢ ψ$}
    \end{bprooftree}
  \end{equation*}
\end{itemize}
\end{proof}

\end{document}
