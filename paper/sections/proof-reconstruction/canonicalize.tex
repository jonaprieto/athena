\documentclass[../../main.tex]{subfiles}
\begin{document}


\subsubsection{Canonicalize.}
\label{sssec:canonicalize}

The \canonicalize rule stands for adding an axiom to the problem.
As far as we know this rule is always used to introduce the subgoals
in their refutations proofs. We also observed how this rule is used to
assume in the refutations, axioms, definitions or hypothesis.

To achieve this , some transformations are carried out by converting a
\verb!fof! formula to its clausal normal form, a conjunction of clauses, and a
\emph{clause} is the disjunction of zero or more literals. We use clausal normal
form as another synonym of conjunctive normal form. The \canonicalize rule
remove inside the formula redundancies (tautologies or definitions) as well. The
\canonicalize jointly with the \clausify rule perform \emph{clausification}. The
clausification process is widely described by
\citeauthor{Sutcliffe1996}~\cite{Sutcliffe1996}.

To reconstruct the \canonicalize rule, we adapted some ideas from the \Metis
source code. The presentation of this reconstruction is as follows. We firstly
describe functions to remove redundancies inside the formula. After, we present
the negative normal form conversion in~Lemma~\ref{lem:nnf} and the conjunctive
normal form in Lemma~\ref{lem:cnf}. At the end of this section, we state
Theorem~\ref{thm:canonicalize} to reconstruct the \canonicalize rule.

Now, we say that there are redundancies in a formula when some of the theorems
in~Fig.~\ref{fig:redundancies} can be applied inside the formula.

\begin{figure}
  \[
    \begin{bprooftree}
      \AxiomC{$Γ ⟝ φ ∨ ⊥$}
      \UnaryInfC{$Γ ⟝ φ$}
    \end{bprooftree}
    \qquad
    \begin{bprooftree}
      \AxiomC{$Γ ⟝ φ ∨ ⊤$}
      \UnaryInfC{$Γ ⟝ ⊤$}
    \end{bprooftree}
    \qquad
    \begin{bprooftree}
      \AxiomC{$Γ ⟝ φ ∨ ¬ φ$}
      \UnaryInfC{$Γ ⟝ ⊤$}
    \end{bprooftree}
    \qquad
    \begin{bprooftree}
      \AxiomC{$Γ ⟝ φ ∨ φ$}
      \UnaryInfC{$Γ ⟝ φ$}
    \end{bprooftree}
  \]

  \[
    \begin{bprooftree}
      \AxiomC{$Γ ⟝ φ ∧ ⊥$}
      \UnaryInfC{$Γ ⟝ ⊥$}
    \end{bprooftree}
    \qquad
    \begin{bprooftree}
      \AxiomC{$Γ ⟝ φ ∧ ⊤$}
      \UnaryInfC{$Γ ⟝ φ$}
    \end{bprooftree}
    \qquad
    \begin{bprooftree}
      \AxiomC{$Γ ⟝ φ ∧ ¬ φ$}
      \UnaryInfC{$Γ ⟝ ⊥$}
    \end{bprooftree}
    \qquad
    \begin{bprooftree}
      \AxiomC{$Γ ⟝ φ ∧ φ$}
      \UnaryInfC{$Γ ⟝ φ$}
    \end{bprooftree}
  \]
  \caption{Theorems to use inside a formula to remove redundancies.}
\label{fig:redundancies}
\end{figure}


\begin{notation}
In a disjunction, $φ ≡ φ₁ ∨ φ₂ ∨ \cdots ∨ φₙ$, we say $ψ ∈_{∨} φ$,
if there is some $i = 1, \cdots, n$ such that $ψ ≡ φᵢ$.
Note that $ψ ∈_{∨} φ$ is another name for the equality
$ψ ≡ \fconjunctor~φ~ψ$.
\end{notation}

Lemma~\ref{lem:rm-or} serves to remove repeated disjuncts in disjunctions.
We assume the formulas to be right-associative unless otherwise stated.

\begin{mainlemma}
  \label{lem:rm-or}
  If $Γ ⟝ φ$ then $Γ ⟝ \frm_{∨}~φ$ where,
  \begin{align*}
  \label{def:rm-or}
    \begin{split}
      &\frm_{∨} :  \NProp \to \NProp\\
      &\frm_{∨}~φ =
      \begin{cases}
        \frm_{∨}~φ₂,    &\text{ if }φ ≡ φ₁ ∨ φ₂\text{ and }φ₁ ∈_{∨} φ₂\\
        φ₁~∨~rm_{∨}~φ₂, &\text{ if }φ ≡ φ₁ ∨ φ₂\\
        φ,  &\text{ otherwise.}
      \end{cases}
    \end{split}
  \end{align*}
\end{mainlemma}

\begin{myexample}

The formula $φ ∨ (ψ ∨ (φ ∨ φ)) ∨ φ$ has redundancies. To remove such
redundancies with Lemma~\ref{lem:rm-or}, we first use Lemma~\ref{lem:rassoc} to
get the right-associative version of the formula. Now, using the $\frm_{∨}$
function, we get the formula $ψ ∨ φ$ logical equivalent to
$φ ∨ (ψ ∨ (φ ∨ φ)) ∨ φ$.

\begin{equation}
\label{eq:lemma-rm-or-example}
  \begin{bprooftree}
  \AxiomC{$Γ ⟝ φ ∨ (ψ ∨ (φ ∨ φ)) ∨ φ $}
  \RightLabel{Lemma~\ref{lem:rassoc}}
  \UnaryInfC{$Γ ⟝ \fassoc_{∨}~(φ ∨ (ψ ∨ (φ ∨ φ)) ∨ φ) $}
  \RightLabel{by (\ref{eq:fassoc})}
  \UnaryInfC{$Γ ⟝ φ ∨ (ψ ∨ (φ ∨ (φ ∨ φ)))$}
  \RightLabel{Lemma~\ref{lem:rm-or}}
  \UnaryInfC{$Γ ⟝ ψ ∨ φ$}
  \end{bprooftree}
  \end{equation}
\end{myexample}

Now, we can remove  redundancies associated with disjunction in
Fig.~\ref{fig:or-redundancies} by applying Lemma~\ref{lem:canon-or}.

\begin{figure}
\label{fig:or-redundancies}
\begin{equation*}
\begin{bprooftree}
  \AxiomC{$Γ ⟝ φ ∨ ¬~φ$}
  \UnaryInfC{$Γ ⟝ ⊤$}
\end{bprooftree}\qquad
\begin{bprooftree}
  \AxiomC{$Γ ⟝ φ ∨ ⊤$}
  \UnaryInfC{$Γ ⟝ ⊤$}
\end{bprooftree}\qquad
\begin{bprooftree}
  \AxiomC{$Γ ⟝ φ ∨ ⊥$}
  \UnaryInfC{$Γ ⟝ φ$}
\end{bprooftree}\qquad
\begin{bprooftree}
  \AxiomC{$Γ ⟝ φ ∨ φ$}
  \UnaryInfC{$Γ ⟝ φ$}
\end{bprooftree}
\end{equation*}
\caption{Redundancies in disjunctions.}
\end{figure}

\begin{mainlemma}
  \label{lem:canon-or}
  If $Γ ⟝ φ$ then $Γ ⟝ \fcanon_{∨}~φ$ where,

\begin{equation*}
\label{eq:canon-or}
\begin{aligned}
 &\hspace{.495mm}\fcanon_{∨} : \NProp \to \NProp\\
 &\begin{array}{llll}
   \fcanon_{∨} &(φ ∨ ⊥)   &= φ \\
   \fcanon_{∨} &(⊥ ∨ φ)   &= φ \\
   \fcanon_{∨} &(⊤ ∨ φ)   &= ⊤  \\
   \fcanon_{∨} &(φ ∨ ⊤)   &= ⊤  \\
   \fcanon_{∨} &(φ₁ ∨ φ₂) &= \frm_∨~(\fndisj_∨~(φ₁ ∨ φ₂)) \\
   \fcanon_{∨} &φ         &= φ.
  \end{array}
\end{aligned}
\end{equation*}

To remove inside the formula the first case in Fig.~\ref{fig:or-redundancies},
we use the $\fndisj_{∨}$ function.

\begin{equation*}
\label{eq:ndisj-or}
\begin{aligned}
  &\hspace{.495mm}\fndisj_{∨} : \NProp \to \NProp\\
  &\begin{array}{lll}
    \fndisj_{∨} &(¬ φ₁ ∨ φ₂) &=
        \begin{cases}
         ⊤, &\text{ if } φ₁ ∈_{∨} φ₂\\
         ⊤, &\text{ if } \fndisj_{∨}~φ₂≡ ⊤\\
         ¬ φ₁ ∨ \fndisj_{∨}~φ₂, &\text{ otherwise.}
        \end{cases}\\

  \fndisj_{∨} &(φ₁ ∨ φ₂)&=
        \begin{cases}
         ⊤, &\,\,\,\,\,\,\text{ if } ¬φ₁ ∈_{∨} φ₂ \\
         ⊤, &\,\,\,\,\,\,\text{ if } \fndisj_{∨}~φ₂~ ≡ ⊤\\
         φ₁ ∨ \fndisj_{∨}~φ₂, &\,\,\,\,\,\,\text{ otherwise.}
        \end{cases}\\
    \fndisj_{∨}&φ &=φ.
    \end{array}
\end{aligned}
\end{equation*}
\end{mainlemma}

Now, we have removed redundancies in the disjunctions by applying the
$\fcanon_{∨}$ function. In a similar way, we define the $\fcanon_{∧}$ function
to work with conjunctions.

\begin{notation}
In a conjunction, $φ ≡ φ₁ ∧ φ₂ ∧ \cdots ∧ φₙ$, we say
$ψ ∈_{∧} φ$, if there is some $i = 1, \cdots, n$ such that $ψ ≡ φᵢ$.
Note that $ψ ∈_{∧} φ$ as the equality $ψ ≡ \fconjunct~φ~ψ$.
\end{notation}

\begin{mainlemma}
  \label{lem:rm-and}
  If $Γ ⟝ φ$ then $Γ ⟝ \frm_{∧}~φ$ where,

  \begin{align*}
    \begin{split}
    &\frm_{∧} : \NProp \to \NProp\\
    &\frm_{∧}~φ =
    \begin{cases}
      rm_{∧}~φ₂,      &\text{ if }φ ≡ φ₁ ∧ φ₂\text{ and }φ₁ ∈_{∧} φ₂\\
      φ₁ ∧ rm_{∧}~φ₂, &\text{ if }φ ≡ φ₁ ∧ φ₂\\
      φ,               &\text{ otherwise.}
    \end{cases}
    \end{split}
  \end{align*}
\end{mainlemma}

In right-associative conjunctions we remove the redundancies
in \ref{fig:and-redundancies} using Lemma~\ref{lem:canon-and}.

\begin{figure}
\label{fig:and-redundancies}
\begin{equation*}
\begin{bprooftree}
  \AxiomC{$Γ ⟝ φ ∧ ¬~φ$}
  \UnaryInfC{$Γ ⟝ ⊥$}
\end{bprooftree}\qquad
\begin{bprooftree}
  \AxiomC{$Γ ⟝ φ ∧ ⊤$}
  \UnaryInfC{$Γ ⟝ φ$}
\end{bprooftree}\qquad
\begin{bprooftree}
  \AxiomC{$Γ ⟝ φ ∧ ⊥$}
  \UnaryInfC{$Γ ⟝ ⊥$}
\end{bprooftree}\qquad
\begin{bprooftree}
  \AxiomC{$Γ ⟝ φ ∧ φ$}
  \UnaryInfC{$Γ ⟝ φ$}
\end{bprooftree}
\end{equation*}
\caption{Redundancies in conjunctions.}
\end{figure}

\begin{mainlemma}
  \label{lem:canon-and}
  If $Γ ⟝ φ$ then $Γ ⟝ \fcanon_{∧}~φ$ where,
  \begin{equation*}
   \label{eq:canon-and}
    \begin{aligned}
     &\hspace{.495mm}\fcanon_{∧} : \NProp \to \NProp\\
      &\begin{array}{lll}
        \fcanon_{∧} &(⊤ ∧ φ)  &= φ \\
        \fcanon_{∧} &(φ ∧ ⊤)  &= φ \\
        \fcanon_{∧} &(⊥ ∧ φ)  &= ⊥  \\
        \fcanon_{∧} &(φ ∧ ⊥)  &= ⊥  \\
        \fcanon_{∧} &(φ₁ ∧ φ₂) &= \frm_∧~(\fndisj_∧~(φ₁ ∧ φ₂)) \\
        \fcanon_{∧} &φ         &= φ
       \end{array}
    \end{aligned}
    \end{equation*}

To remove inside the formula the first case in
Fig.~\ref{fig:and-redundancies}, we use the $\fndisj_{∧}$ function.

  \begin{equation*}
   \label{eq:ndisj-and}
    \begin{aligned}
    &\hspace{.495mm}\fndisj_{∧} : \NProp \to \NProp\\
    &\begin{array}{lll}
      \fndisj_{∧}&(¬ φ₁ ∧ φ₂) &=
        \begin{cases}
          ⊥, &\text{ if } φ₁ ∈_{∧} φ₂\\
          ⊥, &\text{ if } \fndisj_{∧}~φ₂~≡ ⊥\\
          ¬ φ₁ ∧ \fndisj_{∧}~φ₂, &\text{ otherwise.}
        \end{cases}\\
      \fndisj_{∧}&(φ₁ ∧ φ₂) &=
        \begin{cases}
          ⊥,  &\,\,\,\,\,\,\text{ if } φ₁ ∈_{∧} φ₂\\
          ⊥,  &\,\,\,\,\,\,\text{ if } \fndisj_{∧}~φ₂~≡ ⊥\\
          φ₁ ∧ \fndisj_{∧}~φ₂, &\,\,\,\,\,\,\text{ otherwise.}
        \end{cases}\\
      \fndisj_{∧} &φ &=φ.
     \end{array}
    \end{aligned}
  \end{equation*}
\end{mainlemma}

Now, we get the negative normal form of a formula by applying to it
the $\fnnfp$ function defined in~\eqref{eq:nnf-zero}.
Its definition is mainly based on the \Metis source code to normalize
formulas. To define such a function in type theory we follow the
same approach described in \citeauthor{Bezem2002}~\cite{Bezem2002}.
The authors avoid a termination problem with a similar
function by using the polarity of the formula as an additional argument of
the function.

\begin{equation}
\label{eq:nnf-zero}
  \begin{aligned}
  &\hspace{.495mm}\fnnfp : \abbre{Polarity} \to \Prop \to \NProp\\
    &\begin{array}{lll}
      \fnnfp~⊕~(φ₁ ∧ φ₂) &=\fcanon_{∧}~(\fnnfp~⊕~φ₁ ∧ \fnnfp~⊕~φ₂)\\
      \fnnfp~⊕~(φ₁ ∨ φ₂) &=\fcanon_{∨}~(\fnnfp~⊕~φ₁ ∨ \fnnfp~⊕~φ₂)\\
      \fnnfp~⊕~(φ₁ ⇒ φ₂) &=\fcanon_{∨}~(\fnnfp~⊖~φ₁ ∨ \fnnfp~⊕~φ₂)\\
      \fnnfp~⊕~(¬ φ₁)    &=\fnnfp~⊖~φ₁                              \\
      \fnnfp~⊕~φ         &=φ        \\
      \fnnfp~⊖~(φ₁ ∧ φ₂) &=\fcanon_{∨}(\fnnfp~⊖~φ₁ ∨ \fnnfp~⊖~φ₂)\\
      \fnnfp~⊖~(φ₁ ∨ φ₂) &=\fcanon_{∧}(\fnnfp~⊖~φ₁ ∧ \fnnfp~⊖~φ₂)\\
      \fnnfp~⊖~(φ₁ ⇒ φ₂) &=\fcanon_{∧}(\fnnfp~⊖~φ₂ ∧ \fnnfp~⊖~φ₁)\\
      \fnnfp~⊖~(¬ φ₁)    &=\fnnfp~⊕~φ₁\\
      \fnnfp~⊖~⊤         &=⊥\\
      \fnnfp~⊖~⊥         &=⊤\\
      \fnnfp~⊖~φ         &=φ
    \end{array}
  \end{aligned}
\end{equation}

We define the \abbre{Polarity} type which has
two constructors: $⊕$ to denote positive polarity, and $⊖$ to denote
negative polarity of a formula.

% \begin{myremark}
% The formula polarity was not
% relevant in the following description since we found using the $⊕$ polarity to
% call \fpolarity function for the first time was enough to get
% the normalized normal form for propositions.
% \end{myremark}

\begin{mainlemma}
  \label{lem:nnf}
  If $Γ ⊢ φ$ then $Γ ⟝ \fnnf~φ$ where,
  \begin{align*}
   \begin{split}
     &\fnnf : \NProp \to \NProp\\
     &\fnnf~φ = \fnnfp~⊕~φ.
   \end{split}
  \end{align*}
\end{mainlemma}

Now, to convert a formula to its conjunctive normal form, in
Lemma~\ref{lem:dist} we apply the appropriate distribution functions to the
normalized negative formula given by the $\fnnf$ function.

\begin{mainlemma}
  \label{lem:dist}
  $Γ ⟝ φ$ then $Γ ⟝ \fdist~φ$ where,
  \begin{equation*}
  \begin{aligned}
  &\hspace{.495mm}\fdist : \NProp \to \NProp\\
  &\begin{array}{lll}
    \fdist &(φ₁ ∧ φ₂) &= \fdist~φ₁ ∧ \fdist~φ₂\\
    \fdist &(φ₁ ∨ φ₂) &= \fdist_{∨}~(\fdist~φ₁ ∨ \fdist~φ₂)\\
    \fdist &φ         &= φ
   \end{array}\\
  \text{and}\\
  &\hspace{.495mm}\fdist_{∨} : \NProp \to \NProp\\
  &\begin{array}{lll}
    \fdist_{∨}&((φ₁ ∧ φ₂) ∨ φ₃) &= \fdist_{∨}~(φ₁ ∨ φ₂) ∧ \fdist_{∨}~(φ₂ ∨ φ₃)\\
    \fdist_{∨}&(φ₁ ∨ (φ₂ ∧ φ₃)) &= \fdist_{∨}~(φ₁ ∨ φ₂) ∧ \fdist_{∨}~(φ₁ ∨ φ₃)\\
    \fdist_{∨}&φ &= φ.
    \end{array}
   \end{aligned}
  \end{equation*}
\end{mainlemma}

We get the conjunctive normal form by applying
the $\fnnf$ function follow by the $\fdist$ function.

\begin{mainlemma}
\label{lem:cnf}
  If $Γ ⊢ φ$ then $Γ ⊢ \fcnf~φ$ where,
  \begin{align*}
    \begin{split}
    &\fcnf : \Prop \to \Prop\\
    &\fcnf~φ = \fdist~(\fnnf~φ)
    \end{split}
  \end{align*}
\end{mainlemma}

\begin{proof}
  Composition of Lemma~\ref{lem:dist} and Lemma~\ref{lem:nnf}.
  and \ref{lem:nnf}.
\end{proof}

Since all the transformations in Lemma~\ref{lem:nnf} and Lemma~\ref{lem:dist}
came from logical equivalences in propositional logic, we state the following lemma to reconstruct the \canonicalize rule in Theorem~\ref{thm:canonicalize}.

\begin{mainlemma}
\label{lem:cnf-inv}
  If $Γ ⊢ \fcnf~φ$ then $Γ ⊢ φ$.
\end{mainlemma}

The \fcanonicalize rule is a inference rule defined in \eqref{eq:canonicalize}
which performs normalization for a proposition. That is, depending on the role
of the formula in the problem, it converts the formula to its negated normal
form or its conjunctive normal form but in both cases, \canonicalize simplifies
the formula by removing redundancies inside it as we widely described above
Fig.~\ref{fig:redundancies}. When the formula plays the axiom or definition
role, the \canonicalize rule transforms the source formula to its negated
normal form. Otherwise, this rule converts the formula to its conjunctive
normal form.

Since this rule mostly consists of dealing with clauses, to to reconstruct this
rule, our strategy mainly consists of checking the equality of normalized
negative form between the source and the target formula. If it fails, we try to
reorder the conjunctive normal form of the source formula to match with the
conjunctive normal form of the target formula.


\begin{mainth} % (fold)
  \label{thm:canonicalize}
  If $Γ ⊢ φ$ and $ψ : \Target$ then $Γ~⊢~\fcanonicalize~φ~ψ$ where,
  \begin{equation}
  \label{eq:canonicalize}
  \begin{aligned}
  &\hspace{.495mm}\fcanonicalize : \Source \to \Target \to \Prop\\
  &\fcanonicalize~φ~ψ = \begin{cases}
        ψ, &\text{ if } ψ≡φ\\
        ψ, &\text{ if } \fcnf~ψ≡\freorder_{∧∨}~(\fcnf~φ)~(\fcnf~ψ)\\
        φ, &\text{ otherwise. }
        \end{cases}
   \end{aligned}
  \end{equation}
\end{mainth}

\begin{proof}\hspace{3mm}
\begin{itemize}
\item[∙] Case $φ ≡ ψ$. By substitution theorem we conclude $Γ ⊢ ψ$.
\item[∙] Case $ψ ≡ \fnnf~φ$.
\begin{equation*}
  \begin{bprooftree}
    \AxiomC{$Γ ⊢ φ$}
    \RightLabel{Lemma~\ref{lem:nnf}}
    \UnaryInfC{$Γ ⊢₂~\fnnf~φ$}
    \AxiomC{$ψ ≡ \fnnf~φ$}
    \RightLabel{\fsubst}
    \BinaryInfC{$Γ ⊢ ψ$}
  \end{bprooftree}
\end{equation*}

\item[∙] Case $\fcnf~ψ ≡ \freorder_{∧∨}~(\fcnf~φ)~(\fcnf~ψ)$.
  \begin{equation*}
    \begin{bprooftree}
      \AxiomC{$Γ ⊢ φ$}
      \RightLabel{Lemma~\ref{lem:cnf}}
      \UnaryInfC{$Γ ⊢ \fcnf~φ$}
      \RightLabel{Lemma~\ref{lem:reorder-and-or}}
      \UnaryInfC{$Γ ⊢ \freorder_{∧∨}~(\fcnf~φ)~(\fcnf~ψ)$}
      \AxiomC{$\fcnf~\freorder_{∧∨}~(\fcnf~φ)~(\fcnf~ψ)$}
      \RightLabel{\fsubst}
      \BinaryInfC{$Γ ⊢ \fcnf~ψ$}
      \RightLabel{Lemma~\ref{lem:cnf-inv}}
      \UnaryInfC{$Γ ⊢ ψ$}
    \end{bprooftree}
  \end{equation*}
\end{itemize}
\end{proof}

\end{document}
