\documentclass[../../main.tex]{subfiles}
\begin{document}


\subsubsection{Canonicalize.}
\label{sssec:canonicalize}

The \canonicalize rule is an inference that transforms a formula to a its
negative normal form or its conjunctive normal form depending on the role that
the formula plays in the problem as we will explain in a moment. However, in
both cases, this rule removes inside of the formula  any redundancy as long as
possible (\ie, tautologies or definitions).

\begin{mydefinition}
The \emph{negative normal form} of a formula is the logical equivalent version of it
in which negations appear only in the literals and the formula does not
contain any implications.

\end{mydefinition}

\begin{mydefinition}

The \emph{conjunctive normal form} of a formula also called clausal normal form
is the logical equivalent version expressed as a conjunction of clauses where
a \emph{clause} is the disjunction of zero or more literals.

\end{mydefinition}

The \canonicalize rule is used by \Metis to introduce the subgoals in their
refutation proofs but also helps to simplify formulas at intermediate steps in
the derivations. As far as we know, the \canonicalize rule in these cases uses
the conjunctive normal form with simplifications. Otherwise, when an axiom,
definition or hypothesis is needed to prove some goal, this rule gets the
negative normal form of the formula. The \canonicalize rule jointly with the
\clausify rule perform the so-called \emph{clausification} process mainly
described in~\cite{Sutcliffe1996}.

To reconstruct the \canonicalize rule, we adapted some ideas from the \Metis
source code. The presentation of this reconstruction is as follows. We firstly
describe functions to remove redundancies inside of the formula. After, we
present the negative normal form conversion in~Lemma~\ref{lem:nnf} and the
conjunctive normal form in Lemma~\ref{lem:cnf}. At the end of this section, we
state Theorem~\ref{thm:canonicalize} to reconstruct the \canonicalize rule.

Now, we say that there are redundancies in a formula when some of the theorems
in~Fig.~\ref{fig:redundancies} can be applied inside of it.

\begin{figure}
  \[
    \begin{bprooftree}
      \AxiomC{$Γ ⟝ φ ∨ ⊥$}
      \UnaryInfC{$Γ ⟝ φ$}
    \end{bprooftree}
    \qquad
    \begin{bprooftree}
      \AxiomC{$Γ ⟝ φ ∨ ⊤$}
      \UnaryInfC{$Γ ⟝ ⊤$}
    \end{bprooftree}
    \qquad
    \begin{bprooftree}
      \AxiomC{$Γ ⟝ φ ∨ ¬ φ$}
      \UnaryInfC{$Γ ⟝ ⊤$}
    \end{bprooftree}
    \qquad
    \begin{bprooftree}
      \AxiomC{$Γ ⟝ φ ∨ φ$}
      \UnaryInfC{$Γ ⟝ φ$}
    \end{bprooftree}
  \]

  \[
    \begin{bprooftree}
      \AxiomC{$Γ ⟝ φ ∧ ⊥$}
      \UnaryInfC{$Γ ⟝ ⊥$}
    \end{bprooftree}
    \qquad
    \begin{bprooftree}
      \AxiomC{$Γ ⟝ φ ∧ ⊤$}
      \UnaryInfC{$Γ ⟝ φ$}
    \end{bprooftree}
    \qquad
    \begin{bprooftree}
      \AxiomC{$Γ ⟝ φ ∧ φ$}
      \UnaryInfC{$Γ ⟝ φ$}
    \end{bprooftree}
    \qquad
    \begin{bprooftree}
      \AxiomC{$Γ ⟝ φ ∧ ¬ φ$}
      \UnaryInfC{$Γ ⟝ ⊥$}
    \end{bprooftree}
  \]
  \caption{Theorems to remove redundancies inside of a formula.}
\label{fig:redundancies}
\end{figure}


\begin{notation}
In a disjunction, $φ ≡ φ₁ ∨ φ₂ ∨ \cdots ∨ φₙ$, we say $ψ ∈_{∨} φ$,
if there is some $i = 1, \cdots, n$ such that $ψ ≡ φᵢ$.
Note that $ψ ∈_{∨} φ$ is another name for the equality
$ψ ≡ \fconjunctor~φ~ψ$.
\end{notation}

In right-associative disjunctions we remove the redundancies
in Fig.~\ref{fig:or-redundancies} using Lemma~\ref{lem:canon-or}.
We assume the formulas to be right-associative unless otherwise stated.

\begin{figure}
\begin{equation*}
\begin{bprooftree}
  \AxiomC{$Γ ⟝ φ ∨ ⊥$}
  \UnaryInfC{$Γ ⟝ φ$}
\end{bprooftree}\qquad
\begin{bprooftree}
  \AxiomC{$Γ ⟝ φ ∨ ⊤$}
  \UnaryInfC{$Γ ⟝ ⊤$}
\end{bprooftree}\qquad
\begin{bprooftree}
  \AxiomC{$Γ ⟝ φ ∨ φ$}
  \UnaryInfC{$Γ ⟝ φ$}
\end{bprooftree}
\begin{bprooftree}
  \AxiomC{$Γ ⟝ φ ∨ ¬~φ$}
  \UnaryInfC{$Γ ⟝ ⊤$}
\end{bprooftree}\qquad
\end{equation*}
\caption{Theorems to remove redundancies inside of a disjunction.}
\label{fig:or-redundancies}
\end{figure}

\begin{mainlemma}
  \label{lem:canon-or}

  If $Γ ⟝ φ$ and let $φ : \Prop$ be a right-associative formula then $Γ ⟝ \fcanon_{∨}~φ$, where

\begin{equation}
\label{eq:canon-or}
\begin{aligned}
 &\hspace{.495mm}\fcanon_{∨} : \NProp \to \NProp\\
 &\begin{array}{lll}
   \fcanon_{∨} &(φ ∨ ⊥)     &= \fcanon_{∨}~φ \\
   \fcanon_{∨} &(⊥ ∨ φ)     &= \fcanon_{∨}~φ \\
   \fcanon_{∨} &(φ ∨ ⊤)     &= ⊤  \\
   \fcanon_{∨} &(⊤ ∨ φ)     &= ⊤  \\
   % \fcanon_{∨} &(¬ φ₁ ∨ φ₂) &=
   %      \begin{cases}
   %       \fcanon_{∨}~φ₂,        &\text{ if } ¬ φ₁ ∈_{∨} φ₂;\\
   %       ⊤,                     &\text{ if } φ₁ ∈_{∨} φ₂;\\
   %       ⊤,                     &\text{ if } \fcanon_{∨}~φ₂≡ ⊤;\\
   %       ¬ φ₁,                  &\text{ if } \fcanon_{∨}~φ₂≡ ⊥;\\
   %       ¬ φ₁ ∨ \fcanon_{∨}~φ₂, &\text{ otherwise.}
   %      \end{cases}\\
   \fcanon_{∨} &(φ₁ ∨ φ₂)   &=
        \begin{cases}
         ⊤,                     & \text{ if } φ₁ ≡ ¬ ψ \text{ for some }ψ : \Prop \text{ and } ψ ∈_{∨} φ₂;\\
         ⊤,                     & \text{ if } (¬ φ₁) ∈_{∨} φ₂;\\
         \fcanon_{∨}~φ₂,        & \text{ if } φ₁ ∈_{∨} φ₂;\\
         ⊤,                     & \text{ if } \fcanon_{∨}~φ₂~ ≡ ⊤;\\
         φ₁,                    & \text{ if } \fcanon_{∨}~φ₂~ ≡ ⊥;\\
         φ₁ ∨ \fcanon_{∨}~φ₂,   & \text{ otherwise.}
        \end{cases}\\
   \fcanon_{∨} &φ         &= φ.
  \end{array}
\end{aligned}
\end{equation}
\end{mainlemma}

\begin{myexamplenum}

The formula $φ ∨ (ψ ∨ (φ ∨ φ)) ∨ φ$ in \eqref{eq:lemma-rm-or-example} has
redundancies. To remove such redundancies we first use Lemma~\ref{lem:rassoc}
to get the right-associative version of the formula. Then, we can use the
$\fcanon_{∨}$ function to get the logical equivalent formula $ψ ∨ φ$.

\begin{equation}
\label{eq:lemma-rm-or-example}
  \begin{bprooftree}
  \AxiomC{$Γ ⟝ φ ∨ (ψ ∨ (φ ∨ φ)) ∨ φ $}
  \RightLabel{Lemma~\ref{lem:rassoc}}
  \UnaryInfC{$Γ ⟝ \fassoc_{∨}~(φ ∨ (ψ ∨ (φ ∨ φ)) ∨ φ) $}
  \RightLabel{by (\ref{eq:fassoc})}
  \UnaryInfC{$Γ ⟝ φ ∨ (ψ ∨ (φ ∨ (φ ∨ φ)))$}
  \RightLabel{Lemma~\ref{lem:canon-or}}
  \UnaryInfC{$Γ ⟝ \fcanon_{∨}~(φ ∨ (ψ ∨ (φ ∨ (φ ∨ φ))))$}
  \RightLabel{\eqref{eq:canon-or}}
  \UnaryInfC{$Γ ⟝ ψ ∨ φ$}
  \end{bprooftree}
  \end{equation}
\end{myexamplenum}

Now, we have removed redundancies in disjunctions by applying the
$\fcanon_{∨}$ function. In a similar way, we define the $\fcanon_{∧}$ function
to work with conjunctions.

\begin{notation}
In a conjunction, $φ ≡ φ₁ ∧ φ₂ ∧ \cdots ∧ φₙ$, we say
$ψ ∈_{∧} φ$, if there is some $i = 1, \cdots, n$ such that $ψ ≡ φᵢ$.
Note that $ψ ∈_{∧} φ$ as the equality $ψ ≡ \fconjunct~φ~ψ$.
\end{notation}

In right-associative conjunctions we remove the redundancies
in Fig.~\ref{fig:and-redundancies} using Lemma~\ref{lem:canon-and}.

\begin{figure}
\begin{equation*}
\begin{bprooftree}
  \AxiomC{$Γ ⟝ φ ∧ ⊤$}
  \UnaryInfC{$Γ ⟝ φ$}
\end{bprooftree}\qquad
\begin{bprooftree}
  \AxiomC{$Γ ⟝ φ ∧ ⊥$}
  \UnaryInfC{$Γ ⟝ ⊥$}
\end{bprooftree}\qquad
\begin{bprooftree}
  \AxiomC{$Γ ⟝ φ ∧ φ$}
  \UnaryInfC{$Γ ⟝ φ$}
\end{bprooftree}\qquad
\begin{bprooftree}
  \AxiomC{$Γ ⟝ φ ∧ ¬~φ$}
  \UnaryInfC{$Γ ⟝ ⊥$}
\end{bprooftree}
\end{equation*}
\caption{Theorems to remove redundancies inside of a conjunction.}
\label{fig:and-redundancies}
\end{figure}

\begin{mainlemma}
  \label{lem:canon-and}

  If $Γ ⟝ φ$ and let $φ : \Prop$ be a right-associative formula
  then $Γ ⟝ \fcanon_{∧}~φ$, where

  \begin{equation}
   \label{eq:canon-and}
    \begin{aligned}
     &\hspace{.495mm}\fcanon_{∧} : \NProp \to \NProp\\
      &\begin{array}{lll}
        \fcanon_{∧} &(⊥ ∧ φ)     &= ⊥  \\
        \fcanon_{∧} &(φ ∧ ⊥)     &= ⊥  \\
        \fcanon_{∧} &(⊤ ∧ φ)     &= \fcanon_{∧}~φ \\
        \fcanon_{∧} &(φ ∧ ⊤)     &= \fcanon_{∧}~φ \\
        % \fcanon_{∧} &(¬ φ₁ ∧ φ₂) &=
        %   \begin{cases}
        %     ⊥,                     &\text{ if } φ₁ ∈_{∧} φ₂;\\
        %     \fcanon_{∧}~φ₂,        &\text{ if } ¬ φ₁ ∈_{∧} φ₂;\\
        %     ⊥,                     &\text{ if } \fcanon_{∧}~φ₂~≡ ⊥;\\
        %      \fcanon_{∧}~φ₂,       &\text{ if } \fcanon_{∧}~φ₂~≡ ⊤;\\
        %     ¬ φ₁ ∧ \fcanon_{∧}~φ₂, &\text{ otherwise.}
        %   \end{cases}\\
        \fcanon_{∧} &(φ₁ ∧ φ₂) &=
          \begin{cases}
            ⊥,                   & \text{ if } φ₁ ≡ ¬ ψ \text{ for some }ψ : \Prop \text{ and } ψ ∈_{∧} φ₂;\\
            ⊥,                   & \text{ if } (¬ φ₁) ∈_{∧} φ₂;\\
            \fcanon_{∧}~φ₂,      & \text{ if } φ₁ ∈_{∧} φ₂;\\
            ⊥,                   & \text{ if } \fcanon_{∧}~φ₂~≡ ⊥;\\
            φ₁,                  & \text{ if } \fcanon_{∧}~φ₂~≡ ⊤;\\
            φ₁ ∧ \fcanon_{∧}~φ₂, &\text{ otherwise.}
          \end{cases}\\
        \fcanon_{∧} &φ         &= φ.
       \end{array}
    \end{aligned}
    \end{equation}
\end{mainlemma}

Now, we are ready to define the negative normal form of a formula after
simplifications by applying to it the $\fnnfp$ function defined
in~\eqref{eq:nnf-zero} and the $\fcanon_{∨}$ and $\fcanon_{∧}$) functions.  Its definition is mainly based on the \Metis source code to normalize formulas.
To define such a function in type theory we follow the same approach described
in~\cite{Bezem2002}. The authors avoid a termination problem by using the
polarity of the formula as an additional argument of the function.

\begin{equation}
\label{eq:nnf-zero}
  \begin{aligned}
  &\hspace{.495mm}\fnnfp : \abbre{Polarity} \to \Prop \to \NProp\\
&\begin{array}{lll}
  \fnnfp~⊕~(φ₁ ∧ φ₂) &=\fcanon_{∧}~(\fassoc_{∧}~(\fnnfp~⊕~φ₁ ∧ \fnnfp~⊕~φ₂))\\
  \fnnfp~⊕~(φ₁ ∨ φ₂) &=\fcanon_{∨}~(\fassoc_{∨}~(\fnnfp~⊕~φ₁ ∨ \fnnfp~⊕~φ₂))\\
  \fnnfp~⊕~(φ₁ ⇒ φ₂) &=\fcanon_{∨}~(\fassoc_{∨}~(\fnnfp~⊖~φ₁ ∨ \fnnfp~⊕~φ₂))\\
  \fnnfp~⊕~(¬ φ)     &=\fnnfp~⊖~φ\\
  \fnnfp~⊕~φ         &=φ        \\
  \fnnfp~⊖~(φ₁ ∧ φ₂) &=\fcanon_{∨}~(\fassoc_{∨}~(\fnnfp~⊖~φ₁ ∨ \fnnfp~⊖~φ₂))\\
  \fnnfp~⊖~(φ₁ ∨ φ₂) &=\fcanon_{∧}~(\fassoc_{∧}~(\fnnfp~⊖~φ₁ ∧ \fnnfp~⊖~φ₂))\\
  \fnnfp~⊖~(φ₁ ⇒ φ₂) &=\fcanon_{∧}~(\fassoc_{∧}~(\fnnfp~⊖~φ₂ ∧ \fnnfp~⊖~φ₁))\\
  \fnnfp~⊖~(¬ φ)     &=\fnnfp~⊕~φ\\
  \fnnfp~⊖~⊤         &=⊥\\
  \fnnfp~⊖~⊥         &=⊤\\
  \fnnfp~⊖~φ         &=φ.
\end{array}
  \end{aligned}
\end{equation}

We define the \abbre{Polarity} type which has
two constructors: $⊕$ to denote positive polarity, and $⊖$ to denote
negative polarity of a formula. A function to get the polarity of a function
is defined in Appendix~\ref{app:polarity-for-propositions}.

% \begin{myremark}
% The formula polarity was not
% relevant in the following description since we found using the $⊕$ polarity to
% call \fpolarity function for the first time was enough to get
% the normalized normal form for propositions.
% \end{myremark}

\begin{mainlemma}
  \label{lem:nnf}
  If $Γ ⊢ φ$ then $Γ ⟝ \fnnf~φ$, where
  \begin{align*}
   \begin{split}
     &\fnnf : \NProp \to \NProp\\
     &\fnnf~φ = \fnnfp~⊕~φ.
   \end{split}
  \end{align*}
\end{mainlemma}


To get the conjunctive normal form, we make sure the formula is a
conjunction of disjunctions. For such a purpose, we use distributive laws in
Lemma~\ref{lem:dist} to get that form after applying the $\fnnf$ function.

\begin{mainlemma}
  \label{lem:dist}
  $Γ ⟝ φ$ then $Γ ⟝ \fdist~φ$, where
  \begin{equation*}
  \begin{aligned}
  &\hspace{.495mm}\fdist : \NProp \to \NProp\\
  &\begin{array}{lll}
    \fdist &(φ₁ ∧ φ₂) &= \fdist~φ₁ ∧ \fdist~φ₂\\
    \fdist &(φ₁ ∨ φ₂) &= \fdist_{∨}~(\fdist~φ₁ ∨ \fdist~φ₂)\\
    \fdist &φ         &= φ
   \end{array}\\
  \text{and}\\
  &\hspace{.495mm}\fdist_{∨} : \NProp \to \NProp\\
  &\begin{array}{lll}
    \fdist_{∨}&((φ₁ ∧ φ₂) ∨ φ₃) &= \fdist_{∨}~(φ₁ ∨ φ₂) ∧ \fdist_{∨}~(φ₂ ∨ φ₃)\\
    \fdist_{∨}&(φ₁ ∨ (φ₂ ∧ φ₃)) &= \fdist_{∨}~(φ₁ ∨ φ₂) ∧ \fdist_{∨}~(φ₁ ∨ φ₃)\\
    \fdist_{∨}&φ &= φ.
    \end{array}
   \end{aligned}
  \end{equation*}
\end{mainlemma}

We get the conjunctive normal form by applying
the $\fnnf$ function follow by the $\fdist$ function.

\begin{mainlemma}
\label{lem:cnf}
  If $Γ ⊢ φ$ then $Γ ⊢ \fcnf~φ$, where
  \begin{align*}
    \begin{split}
    &\fcnf : \Prop \to \Prop\\
    &\fcnf~φ = \fdist~(\fnnf~φ).
    \end{split}
  \end{align*}
\end{mainlemma}

\begin{proof}
  Composition of Lemma~\ref{lem:dist} and Lemma~\ref{lem:nnf}.
\end{proof}

Since all the transformations in Lemma~\ref{lem:nnf} and Lemma~\ref{lem:dist}
came from logical equivalences in propositional logic, we state the following
lemma used in the reconstruction of the \canonicalize rule in
Theorem~\ref{thm:canonicalize}.

\begin{mainlemma}
\label{lem:cnf-inv}
  If $Γ ⊢ \fcnf~φ$ then $Γ ⊢ φ$.
\end{mainlemma}

Now, we are ready to reconstruct the \fcanonicalize rule. This inference rule
defined in \eqref{eq:canonicalize} performs normalization for a proposition.
That is, depending on the role of the formula in the problem, it converts that
formula to its negative normal form or its conjunctive normal form. In both
cases, \canonicalize simplifies the formula by removing redundancies inside of
it as we widely described above for theorems in Fig.~\ref{fig:redundancies}.
When the formula plays the axiom or definition role, the \canonicalize rule
transforms the source formula to its negative normal form. Otherwise, this rule
converts the formula to its conjunctive normal form.

Since this rule mostly consists of dealing with clauses, to reconstruct this
rule, our strategy mainly consists of checking the equality of negative normal
form between the source and the target formula. If it fails, we try to reorder
the conjunctive normal form of the source formula to match with the conjunctive
normal form of the target formula. It definition is as follows.


\begin{mainth} % (fold)
  \label{thm:canonicalize}
   Let $ψ : \Target$. If $Γ ⊢ φ$ then $Γ~⊢~\fcanonicalize~φ~ψ$, where
  \begin{equation}
  \label{eq:canonicalize}
  \begin{aligned}
  &\hspace{.495mm}\fcanonicalize : \Source \to \Target \to \Prop\\
  &\fcanonicalize~φ~ψ = \begin{cases}
        ψ, &\text{ if  } ψ ≡ φ;\\
        ψ, &\text{ if  } ψ ≡ \fnnf~φ;\\
        ψ, &\text{ if  } \fcnf~ψ≡ \freorder_{∧∨}~(\fcnf~φ)~(\fcnf~ψ);\\
        φ, &\text{ otherwise. }
        \end{cases}
   \end{aligned}
  \end{equation}
\end{mainth}

\begin{proof}\hspace{3mm}
\begin{itemize}
\item[∙] Case $φ ≡ ψ$. By substitution theorem we conclude $Γ ⊢ ψ$.
\item[∙] Case $ψ ≡ \fnnf~φ$.
\begin{equation*}
  \begin{bprooftree}
    \AxiomC{$Γ ⊢ φ$}
    \RightLabel{Lemma~\ref{lem:nnf}}
    \UnaryInfC{$Γ ⊢₂~\fnnf~φ$}
    \AxiomC{$ψ ≡ \fnnf~φ$}
    \RightLabel{\fsubst}
    \BinaryInfC{$Γ ⊢ ψ$}
  \end{bprooftree}
\end{equation*}

\item[∙] Case $\fcnf~ψ ≡ \freorder_{∧∨}~(\fcnf~φ)~(\fcnf~ψ)$.
  \begin{equation*}
    \begin{bprooftree}
      \AxiomC{$Γ ⊢ φ$}
      \RightLabel{Lemma~\ref{lem:cnf}}
      \UnaryInfC{$Γ ⊢ \fcnf~φ$}
      \RightLabel{Lemma~\ref{lem:reorder-and-or}}
      \UnaryInfC{$Γ ⊢ \freorder_{∧∨}~(\fcnf~φ)~(\fcnf~ψ)$}
      \AxiomC{$\fcnf~ψ~≡~\freorder_{∧∨}~(\fcnf~φ)~(\fcnf~ψ)$}
      \RightLabel{\fsubst}
      \BinaryInfC{$Γ ⊢ \fcnf~ψ$}
      \RightLabel{Lemma~\ref{lem:cnf-inv}}
      \UnaryInfC{$Γ ⊢ ψ$}
    \end{bprooftree}
  \end{equation*}
\end{itemize}
\end{proof}

\end{document}
