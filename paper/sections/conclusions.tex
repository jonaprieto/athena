
\documentclass[../main.tex]{subfiles}
\begin{document}

% ===================================================================

\section{Conclusions}
\label{sec:conclusions}

We presented a reliable proof-reconstruction in type theory
for the propositional fragment of the \Metis prover.

We built \Athena translator tool written in \Haskell
to generate \Agda proof-terms of \Metis derivations.
\Agda files delivered by this translator imports
\Agda formalizations of the \Metis reasoning~\cite{AgdaProp,AgdaMetis}.

The reconstruction approach in this study was designed to use
only syntactical aspects of the logic.
This decision was in the beginning a drawback
since it demands more detailed proofs, a
description of every transformation or deduction step performed by
the prover, which is rarely included in the output of these programs,
see Section~\ref{ssec:emulating-inferences}.
Nevertheless, we chose that syntactical treatment instead of using
semantics to extend this work towards the support of first-order logic.
Recall, satisfiability in first-order logic is undecidable.

As an early future work, we propose to extend the
proof-reconstruction presented in this paper to support the
proposition fragment with equality of \Metis.
Another option is to extend the proof-reconstruction to support other \ATPs for \CPL like \name{EProver} or \name{Z3}.
This development can be carried out by following the \name{EProver}
inference rules described on Kanso's Ph.D. Thesis~\cite{Kanso2012}.

In \Agda, we found the proof-reconstruction proposed by \citeauthor{Kanso2012}
in~\cite{Kanso2012,kanso2016light}. The author present an approach similar to
the reasoning presented in Section~\ref{ssec:workflow}.
Nonetheless, his approach includes semantics reasoning that as we mentioned
above differs completely from our proof-reconstructing propose.

Summarizing, we provided for each \Metis inference rule a theorem in
type theory to emulate the rule reasoning. Our approach is mainly
exposed in Section~\ref{ssec:emulating-inferences}.
We identify some improvements for our approach that we leave it for future
work.
For instance, we can investigate the consequences of removing \clausify
inference by strengthen \canonicalize rule in Thm.~\ref{thm:canonicalize}
The \simplify rule is still lacking of some coverage since it is fairly
complex its \Metis source code, and many cases could be omitted.
Other improvements are related with computational efficiency of
rules that strongly depend on reordering tasks,
Section~\ref{sssec:resolve}.
We believe some rules like \canonicalize can get some help by
performing an analysis with \Athena.

One of the main contribution of this study was
increasing the trustworthiness of the automatic prover \Metis.
Justifying a proof by a theorem prover
has a real significant impact for these automatic tools.
The reverse engineering task to understand the prover reasoning
can reveal important issues or bugs
in many parts of these systems (e.g., preprocessing, reasoning, or
deduction modules). During this research, we had the opportunity
to improve \Metis by reporting some bugs\footnote{Issues No.~2,
No.~4. and commit \name{8a3f11e} in the \Metis' official repository
\url{https://github.com/gilith/metis}}.
Fortunately, all these problems were fixed quickly by Joe Hurd in
the \Metis 20170810 release.

\end{document}
