
\documentclass[../main.tex]{subfiles}
\begin{document}

% ===================================================================

\section{Conclusions}
\label{sec:conclusions}

We presented a reliable proof-reconstruction of \Metis solutions
for problems in classical propositional logic without logic.
The reconstruction approach in this study was designed to use
only a syntactic treatment of the logic without considering the
semantics. This aspect represent a difficulty but also an opportunity
to extend this work towards the support for first-order logic.
As a first future work, we see the support of \Metis propositional
fragment with equality.

Summarizing, we provided for each \Metis inference rule a theorem that
emulate the rule deduction. The emulated
version rules exposed in Section~\ref{ssec:emulating-inferences} admit
improvements, we left this for future work, particularly, a better
implementation of \simplify rule.
Some of the improvements are related with computational efficiency.
For instance, we use widely functions to check for syntactical equivalence
between propositions and to reorder propositions to match with another.
We believe both are expensive tasks and can get help in the analysis
step inside the translator tool.

The proof-reconstruction tool presented in this paper
is able to translate \Metis derivations in \TSTP format to \Agda proof-terms.
The development presented in this paper can be extended to support
other \ATPs for \CPL like \name{EProver} or \name{Z3}.
This future work can be carried out by following the \name{EProver} inference
rules described on Kanso' Ph.D. Thesis~\cite{Kanso2012}.

One of the main contribution of this study is
increasing the trustworthiness of these automatic provers,
in our case, \Metis. Justify a proof-object of a theorem prover has a real
significant impact for these automatic tools.
The reverse engineering task can reveal important issues or bugs
in many parts of these system (e.g., preprocessing, reasoning, or
deduction modules). While doing this research, we had the opportunity
to find some of those kind of bugs\footnote{Issues No.~2, No.~4. and
commit \name{8a3f11e} in the \Metis' official repository
\url{https://github.com/gilith/metis}}. Fortunately, all these bugs
were fixed quickly for release 20170810 by Joe Hurd.

\end{document}
