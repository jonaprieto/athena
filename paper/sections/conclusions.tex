
\documentclass[../main.tex]{subfiles}
\begin{document}

% ===================================================================

\section{Conclusions}
\label{sec:conclusions}

We presented a reliable proof-reconstruction in type theory of \Metis
solutions for problems in classical propositional logic without
equality. As a product of this research, we built a proof-reconstruction tool written in \Haskell. This tool generates
\Agda proof-terms of \Metis \TSTP derivations. \Agda files
delivered by the aforementioned tool are supported by the
formalization in \Agda~\cite{AgdaProp,AgdaMetis} of the \Metis rules
described in Section~\ref{ssec:emulating-inferences}.

The reconstruction approach in this study was designed to use
only syntactical aspects of the logic. This decision was first a
difficulty since it demands more detailed proof-objects, a
description of every transformation or deduction step performed by
the prover, see
Section~\ref{sssec:normalization} and Section~\ref{sssec:reordering}.
But, we chose that syntactical treatment as the opportunity
to extend this work towards the support of first-order logic.
Recall, satisfiability in first-order logic is undecidable.

As an early future work, we propose to extend the
proof-reconstruction presented in this paper to support the
proposition fragment with equality of \Metis.
Another option is to extend the proof-reconstruction to support other \ATPs for \CPL like \name{EProver} or \name{Z3}.
This development can be carried out by following the \name{EProver}
inference rules described on Kanso' Ph.D. Thesis~\cite{Kanso2012}.

Summarizing, we provided for each \Metis inference rule a theorem in
type theory to emulate the rule reasoning. This approach mainly
exposed in Section~\ref{ssec:emulating-inferences}
admits some improvements, we leave it for future work. We concern
mainly for a better coverage of the \simplify rule.
Other improvements are related with computational efficiency of
rules strongly depend on reordering tasks,
Section~\ref{sssec:reordering}.
We believe some rules like \canonicalize can get some help from a analysis done in the translator tool.

One of the main contribution of this study was
increasing the trustworthiness of the automatic provers,
in our case, \Metis.
Justifying a proof-object of a theorem prover
has a real significant impact for these automatic tools.
The reverse engineering task can reveal important issues or bugs
in many parts of these system (e.g., preprocessing, reasoning, or
deduction modules). During this research, we had the opportunity
to improve \Metis by reporting some bugs\footnote{Issues No.~2,
No.~4. and commit \name{8a3f11e} in the \Metis' official repository
\url{https://github.com/gilith/metis}} in \Metis.
Fortunately, all these problems were fixed quickly by Joe Hurd in
20170810 release.

\end{document}
