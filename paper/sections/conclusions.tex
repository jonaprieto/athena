
\documentclass[../main.tex]{subfiles}
\begin{document}

% ===================================================================

\section{Conclusions}
\label{sec:conclusions}

In this document, we reconstruct proofs
of propositional logic without equality deliver by \Metis
successfully.

Despite the fact, our logic domain is more straightforward than
first-order or high-order logic where there are many
proof-reconstruction tools (see a list of some of them in the
related work, Section \ref{sec:related-work}), we found some issues
around the reconstruction that show this task is far from trivial.
Depending  on the systems involved in the translation, some
difficulties can arise. We listed some of them in Section
\ref{ssec:metis-language-and-proofs} and in Section
\ref{sec:proof-reconstruction}.

For instance, just showing equality between Normal Forms for the
same formula in the output of the \canonicalize inference and a
common implementation for this kind of conversion; we observed that
\Metis performs reordering for each inner expression mainly in the
presence of negative literals. Another aspect to take into account
is the \Metis' treatment for conjunctions and disjunction;
these connectives were treated as right associative and binary
operators in the \TSTP format but that assumptions is not a standard.
We had to update the \TSTP parser module introduced in
\cite{Gomez-Londono2015} to include features like these.

An important matter was how dealing with
the equality between formulas since, in our approach, we only work
with the syntactic treatment of the logic without considering the
semantics.
One of the main reasons for such a restriction relay on our plans
further to support first-order logic, where a semantic treatment is
not viable since it computational complexity.\unsure{Please confirm this.}

Nevertheless, the proof-reconstruction is certainly possible, and
the verification of solutions of \CPL problems can be carried out
successful in \Agda. Our tool is able to translate \Metis
derivations in \TSTP format to deliver \Agda code of the proof,
\change{Better uses \emph{proof-terms}?}
and this proof can be type-checked by \Agda.

The source code of our tool can be extended to support
other \ATPs for \CPL like \name{E}. This work can be carried out
may following the description in Kanso' Ph.D. Thesis~\cite{Kanso2012}
for the inference rules of \name{E} or \name{Z3}.

Trying to justify a proof-object of a theorem prover has a
significant impact, may increase the trustworthiness of the prover
proving its correctness, or contrary, revealing important issues to
improve the system. In both cases, there is a benefit for both
sides, the user of this technology, and their developers.

For instance, while writing this document, we found a soundness bug,
a bug in the printing out of the \TSTP derivations, and a bug in the
parsing for \TPTP problem in \Metis. Fortunately, all these bugs
were fixed quickly by Hurd\footnote{Issues No. 4, No. 2, and the
commit \name{8a3f11e} from the \Metis' official repository
\url{https://github.com/gilith/metis}}, and a new version of \Metis
was released (version 2.3, release 20170810).

% \subsubsection*{Future Work.}
% \label{ssec:future-work}

\end{document}
