% -*- root: main.tex -*-
\documentclass[main.tex]{subfiles}
\begin{document}

% ===================================================================

\section{Conclusions}
\label{sec:conclusions}

Besides our scope in this document, propositional logic without
equality, is straightforward compared with other logics like
first-order or high-order logic; the proof reconstruction task is
far from trivial. Depending on the systems involved in the
translation, some difficulties can arise. We listed some of them in
section \ref{sec:metis-language-and-proofs} and section
\ref{sec:proof-reconstruction}.

For instance, just showing equality between Normal Forms for the
same formula in the output of the \canonicalize inference and a
common implementation for this kind of conversion; we observed that
\Metis performs reordering for each expression mainly in the
presence of negative literals. Another example was the \Metis'
convention for conjunctions and disjunction as right associative and
binary operators in the \TSTP format. We had to update the \TSTP
parser module introduced in \cite{Gomez-Londono2015} to include some
feature like this.

However, we believe one of our biggest obstacles was dealing with
the Equality between formulas since, in our approach, we only work
with the syntactic treatment of the logic without considering the
semantics. One of the main reasons for such a restriction resides in
our plans to support further first-order logic where a semantic
treatment is not viable.\unsure{Please confirm this assumption.}

Nevertheless, the proof reconstruction is certainly possible, and
the verification of solutions of \CPL problems can be carried out
successful in \Agda. Our tool is capable of translating \Metis
derivations in \TSTP format to deliver \Agda code of the proof, and
this proof can be type-checked by \Agda.

We believe the source code of our tool can be extended to support
other \ATPs for \CPL like \name{E}. For example, we incorporate the
inferences rules of \name{E} or \name{Z3} from the description in
Kanso's Thesis~\cite{Kanso2012}.

Trying to justify output from a theorem prover has a real impact,
increasing the trustworthy of the prover. For instance, we found a
soundness bug, a bug in the printing out of the \TSTP derivations,
and a bug in the parsing for \TPTP problem in \Metis.
Fortunately, all these bugs were fixed quickly by Hurd\footnote{
Issues No. 4, No. 2, and the commit \name{8a3f11e} from the
\Metis' official repository \url{https://github.com/gilith/metis}},
and a new version of \Metis was released
(version 2.3, release 20170810).

% \subsubsection*{Future Work.}
% \label{ssec:future-work}

\end{document}