\documentclass[../main.tex]{subfiles}
\begin{document}

% ===================================================================

\section{Preliminaries}
\label{sec:preliminaries}

\subsection*{Classical Propositional Logic}
\label{ssec:CPL}

\textit{Propositions.}
We define the propositional logic as usual.
A proposition is an expression of indivisible propositional atoms
(e.g., symbols $φ₀, φ₁, \cdots$), the logic constants:
$⊥$, $⊤$, the binary connectives ($∧, ∨, ⇒, ⇔)$, and the negation
$(¬)$. We use the inductive definition of the set \Prop
presented in \cite{VanDalen1994} as the mechanism to produce any
propositional formula.

\begin{itemize}
\item $⊥, ⊤ \in \Prop$.
\item $ϕ_i \in \Prop$ for all $i \in \mathbb{N}$.
\item If $φ, ψ\in \Prop$, then $(φ\ \square\ ψ) \in \Prop$,
where the $\square$ symbol is one of the binary connectives mentioned above.
\item If $φ \in \Prop$, then $(¬ φ) \in \Prop$.
\end{itemize}

\emph{Inference Rule.}
An inference rule is a deduction step of the form of a set of propositional
\emph{premises} that produces propositions as the conclusion.
A derivation is a

is a mechanism to obtain the conclusion $φ$ from
the application of some deduction steps. Such deduction step has

As we show in Fig.~\ref{fig:derivation}, the premises are always above the line, the conclusion
is below it, and the name of the inference or the theorem used is
on the left or on the right of the line.
% As a notation convention, if the name of a rule used in the derivation step is typed using verbatim font face, for example, \verb!resolve!, it means the inference rule belongs to our implementation~\cite{AgdaProp, AgdaMetis} in the Proof Assistant of a inference rule presented
% with a similar name.

\begin{figure}
\label{fig:derivation}
\begin{equation*}
\begin{bprooftree}
  \AxiomC{$φ ∧ ψ$}
  \RightLabel{∧-proj₂}
  \UnaryInfC{$ψ$}
  \AxiomC{$φ ∧ ψ$}
  \RightLabel{∧-proj₁}
  \UnaryInfC{$φ$}
  \RightLabel{∧-intro}
  \BinaryInfC{$ψ ∧ φ$}
\end{bprooftree}
\end{equation*}
\caption{Derivation example.}
\end{figure}

\emph{Theorems.}
A sequent $Γ ⊢ φ$ represents a theorem, but we
define it as a relation between a set of propositions premises $Γ$,
and $φ$ as the conclusion of the sequent.
The symbol $⊢$ is called turnstile.
If $Γ = ∅$, we write $⊢ φ$, and we say that $φ$ is a theorem.
A theorem, $Γ ⊢ φ$ , means that there is a derivation with
conclusion $φ$ and with all (uncanceled) hypotheses in $Γ$.


% \emph{Postulates.}
% We postulate the principle of excluded third
% (\abbre{PEM}). Assuming \abbre{PEM}, 

% \begin{equation*}
% \label{eq:PEM-RAA}
% \begin{bprooftree}
% \AxiomC{}
% \RightLabel{PEM,}
% \UnaryInfC{$Γ ⊢ φ ∨ ¬ φ$}
% \end{bprooftree}\qquad
% \begin{bprooftree}
% \AxiomC{$Γ, ¬ φ ⊢ ⊥$}
% \RightLabel{RAA.}
% \UnaryInfC{$Γ ⊢ φ$}
% \end{bprooftree}
% \end{equation*}

\emph{Natural Deductions.}
The natural deduction introduced by Gentzen in 1930s defines a formal system
for derivations that uses a set of inference rules as the only way to obtain a
conclusion. We use this system to refer to formal proofs as natural deduction
proofs and conversely.

In this document, we extend the \CPL formal system
Fig.~\ref{fig:CPL-inference-rules} described on
paper~\cite{Altenkirch2015} to include inferences rules for the biconditional
connective, $PEM$ and $RAA$ postulates, among others.


\emph{Proof Trees.} We use trees to represent derivations and
natural deduction proofs (see for instance, the proof tree in a
\Metis' derivation in Fig.~\ref{fig:metis-example}).
In a proof tree, the root is the conclusion of the entire
derivation. The nodes labeled with the name of the inference rule
contains the result of applying that rule to their parents. Lastly,
the leaves in the tree become the premises, not all necessary
canceled.
 % (see a better discussion about canceling  hypotheses in proof trees in Section 2.4 in \citeauthor{VanDalen1994}~\cite{VanDalen1994}).

\end{document}
