
\documentclass[../main.tex]{subfiles}
\begin{document}

% ===================================================================

\section{Implementation}
\label{sec:implementation}


\subsection{A Brief Review of Agda}
\label{ssec:agda-definition}

\Agda is an interactive system for constructing proofs and programs,
based on Martin-L\"{o}f's type theory and extended with records,
parametrised modules, among other features. One of the main
strengths of \Agda is its support for writing proofs, which we shall
call \Agda's proof engine. It consists of: support for inductively
defined types, including inductive families, and function
definitions using pattern matching on such types, normalisation
during type- checking, commands for refining proof-terms, coverage
checker and termination checker.

The inductive approach for representing classical propositional
logic is better because we benefit from \Agda's proof engine and its
Unicode support that allows us writing proofs similar as we find in
math text books.

In \Agda, we have defined a proposition formula
as an inductive type using the keyword \texttt{data} and the name
\name{PropFormula}.
We extended the definition for propositions presented in
\cite{Altenkirch2015} to mathc the definion in Section
\ref{sec:preliminaries}.
We include every connective and logic constant as a constructor.
We also define the constructor for atomic propositions denoted by
\verb!Var!. These atomic expressions are represented using a natural
number $i$ bounded by a size of the universe $n$ for those atomic
propositions (in \Agda, we use the data type \verb!Fin! from the
standard  library for such a purpose).

\begin{verbatim}
  data PropFormula : Set where
    Var : Fin n → PropFormula
    ⊤ : PropFormula
    ⊥ : PropFormula
    _∧_ _∨_ _⇒_ _⇔_ : (φ ψ : PropFormula) → PropFormula
    ¬_ : (φ : PropFormula) → PropFormula
\end{verbatim}

\begin{myremark}
The underscores in the above definition denote holes for a function of arity equivalent to the number of such holes.
\end{myremark}

\begin{myremark}
The type notation
\begin{verbatim}
  _⇔_ : (φ ψ : PropFormula) → PropFormula
\end{verbatim}
is a syntax sugar for the type
\begin{verbatim}
  PropFormula → PropFormula → PropFormula.
\end{verbatim}
\end{myremark}

\begin{myexamplenum}
\hspace*{5cm}\\[3mm]
\begin{verbatim}
  r : PropFormula
  p = Var (# 0)
  q = Var (# 1)
  r = Var (# 2)

  φ : PropFormula
  φ = (¬ r ∨ p) ∧ (¬ r ∨ q)
\end{verbatim}
\end{myexamplenum}

\end{document}
