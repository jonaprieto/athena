In the following, we use the \Bool type for booleans defined as usual as
well as $not$ function, $and$ function for the boolean conjunction,
and $if_then_else$ dependent function type.


We noted in \TSTP derivations that \Metis does not take into account
some inference rules for the \Metis' official documentation; we
refer to the \Metis' logic kernel.
Besides the six \Metis' inference rules from
Fig.~\ref{fig:metis-inferences}, its proof-objects, the \TSTP
derivations, exhibit an entirely different set of rules.

For instance, for the propositional logic fragment without equality,
only three rules are valid from the \Metis' Logic Kernel:
\emph{assume}, \emph{axiom}, and \emph{resolve}. Nevertheless, as we
observed in \TSTP derivations, for such a propositional fragment,
\Metis uses in its deductions steps, six rules where only the
\emph{resolve} inference persists from the original list. In the
case of \resolve, it differs from its original version since, in
\TSTP, this rule has by definition an argument, a literal. The other
undocumented rules found in \TSTP derivations are: \canonicalize,
 \conjunct, \negate, \simplify, and \strip.

\begin{figure}
\centering
  \begin{bprooftree}\tt
    \AxiomC{}
    \RightLabel{negate}
    \UnaryInfC{$\neg p$}
    \RightLabel{strip}
    \UnaryInfC{$\neg p$}
    \AxiomC{}
    \RightLabel{axiom}
    \UnaryInfC{$p$}
    \RightLabel{canonicalize}
    \UnaryInfC{$p$}
    \RightLabel{simplify}
    \BinaryInfC{$⊥$}
    \RightLabel{canonicalize}
    \UnaryInfC{$⊥$}
  \end{bprooftree}
  \caption{The \Metis' proof tree for $p \vdash p$ from the
  derivation in Fig.~\ref{fig:metis-proof-tstp}}
  \label{fig:metis-example}
\end{figure}

This lack of the \Metis' documentation represents one of the first barriers to build proof-reconstruction tools since some inference rules
like \canonicalize or \simplify summarize many essential steps to deduce the conclusion, making them hard to justify and emulate
(see a comment related on the presenting of \name{Sledgehammer}
in \citeauthor{paulson2007source}~\cite{paulson2007source} and \citeauthor{Farber2015}~\cite{Farber2015}).

Henceforth, one of our main purposes with this document consists of
giving a response to these rules looking at the \Metis' source code but also reviewing a representative set of solutions generated by
\Metis for the \CPL collection of \TPTP problems~\cite{Prieto-Cubides2017} including some problem from the \Metis repository itself.

In the following, we briefly describe each \Metis' rule from our perspective, but we present a formal description for each rule in Section \ref{sec:proof-reconstruction}.

\vskip 2mm

% ...................................................................

\textit{Splitting}. To prove a goal, \Metis splits the goal into
disjoint cases. This process produces a list of new subgoals, the
conjunction of these subgoals implies the goal. Then, a proof of the
goal becomes in smaller proofs, one refutation for each subgoal.
These subgoals are introduced in the \TSTP derivation with the \strip
inference rule. We refer the reader to see Section~\ref{sssec:strip-a-goal} for the formalization of \strip rule.

\begin{verbatim}
fof(goal, conjecture, p & r & q).
fof(subgoal_0, plain, p, inference(strip, [], [goal])).
fof(subgoal_1, plain, p => r, inference(strip, [], [goal])).
fof(subgoal_2, plain, (p & r) => q, inference(strip, [], [goal])).
\end{verbatim}

% ...................................................................

\textit{Clausification.} The \clausify rule transforms a
propositional formula in its clausal normal form, a conjunction
of clauses. Where a \emph{clause} is the disjunction of zero or more
literals and a \emph{literal} is an atom (positive literal) or a
negation of an atom (negative literal). Be aware this kind of conversion between one formula to its clausal normal form is not unique, and \Metis has a customized approach to perform that transformation. We refer the reader to see Section~\ref{sssec:clausification} for the formalization of \clausify rule.

\vskip 2mm

% ...................................................................

\textit{Normalization.} We observed \clausify rule described above is preceded by \canonicalize rule when \Metis introduces at some point of the derivation an axiom or a definition of the problem. In that case, both rules aim to perform a common task for
automatic theorem provers, Clausification, mainly described on paper
by \citeauthor{Sutcliffe1996}~\cite{Sutcliffe1996}. When the
aforementioned case does not occur we argue \canonicalize transforms
proposition into another one that mainly matches with its negative
normal form, its conjunctive normal form or its disjunctive normal
form. In addition, \canonicalize could also perform other tasks like
removing logic constants (⊥, ⊤), definitions, and tautologies, or even reorder both conjunctions and disjunctions of the proposition.

For instance, assuming the commutative properties for both
connectives, conjunction and disjunction, and using the theorems
in Fig.~\ref{fig:conjunctive-disjunctive-simpl}, \canonicalize could
use them to reduce a proposition to a one smaller by applying
recursively on the structure of the formula those theorems.
We refer the reader to see Section~\ref{sssec:normalization} to continue our approach to emulate this rule.

\begin{figure}
\[%\scalebox{0.9}{
  \begin{bprooftree}
    \AxiomC{$φ ∧ ⊥$}
    \UnaryInfC{$⊥$}
  \end{bprooftree}
  \qquad
  \begin{bprooftree}
    \AxiomC{$φ ∧ ⊤$}
    \UnaryInfC{$φ$}
  \end{bprooftree}
  \qquad
  \begin{bprooftree}
    \AxiomC{$φ ∧ ¬ φ$}
    \UnaryInfC{$⊥$}
  \end{bprooftree}
  \qquad
  \begin{bprooftree}
    \AxiomC{$φ ∧ φ$}
    \UnaryInfC{$φ$}
  \end{bprooftree}
%}
\]

\[%\scalebox{0.9}{
  \begin{bprooftree}
    \AxiomC{$φ ∨ ⊥$}
    \UnaryInfC{$φ$}
  \end{bprooftree}
  \qquad
  \begin{bprooftree}
    \AxiomC{$φ ∨ ⊤$}
    \UnaryInfC{$⊤$}
  \end{bprooftree}
  \qquad
  \begin{bprooftree}
    \AxiomC{$φ ∨ ¬ φ$}
    \UnaryInfC{$⊤$}
  \end{bprooftree}
  \qquad
  \begin{bprooftree}
    \AxiomC{$φ ∨ φ$}
    \UnaryInfC{$φ$}
  \end{bprooftree}
%}
\]
\caption{Definitions for conjunction and disjunction simplification
in the \canonicalize inference.}
\label{fig:conjunctive-disjunctive-simpl}
\end{figure}
