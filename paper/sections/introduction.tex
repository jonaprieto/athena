
\documentclass[../main.tex]{subfiles}
\begin{document}

% ===================================================================

\section{Introduction}
\label{sec:introduction}

An automatic theorem provers (henceforth \ATP) is a tool that intends to prove
conjectures based on assumptions and hypothesis in a logic specification.
Verifying proofs generated by \ATPs are non-trivial tasks.
In the last decades, \ATPs are fast becoming a key instrument in
different disciplines and real applications (\eg, verifying railway interlocking systems in~\cite{Kanso2012}).
Therefore, researchers from academy and industry have shown an increased interest to prove the validity of their deduction algorithms.

See for instance bugs found in ATPs in~\cite{Keller2013,Bohme2011,Fleury2014}.
For that reason, in order to give confidence to the user of these systems,
many ATPs have started to include in their outputs, a witness of their
conclusions. However, existing research recognizes that in many cases these proofs encode non-trivial reasoning
hard to reconstruct and therefore hard to verify. The proof-reconstruction
address this problem.
It becomes in mostly cases a reverse engineering to justify the prover reasoning.
That happen when the reconstruction is made by another and not by
the developers of the \ATP.

Therefore, the proof generated by the prover and its presentation play an important role for proof-reconstruction. To verify such a proof, it is necessary to have
it in a consistent format, that is, a full script
describing step-by-step with exhaustive details and without ambiguities, the derivation to get the proof.
For classical propositional logic (henceforth \CPL) from a list of at least forty\footnote{\ATPs
available from the web
service \name{SystemOnTPTP} of the TPTP World.} \ATPs, just a few
provers were able to output proofs.

One approach to address the proof-reconstruction
is integrating the \ATP, the \emph{source} system, with a proof-assistant, the
\emph{target} system.
The target system becomes in the proof checker. These proof-assistants allow
us to define the formal language used in the proofs, \ie, operators, logic
variables, axioms, and theorems. A proof-reconstruction tool provides such an
integration, translating the replied generated by the prover into the formalism of the proof-assistant.

Previous studies have reported proof-reconstruction using proof-assistants
based on classical logic where the development is at a
mature stage~\cite{paulson2010three,hurlin07practical,kaliszyk2013}.

Another approaches has been proposed for proof-reconstructing using
the formalism of type theory.
The reader can review~\cite{Bezem2002,armand2011,Ekici2017,kanso2016light}.

In this paper, we describe how to reconstruct proofs in type theory
generated by the \Metis prover\footnote{v2.3 release 2017102}~\cite{hurd2003first}.
We ported the subset of the \Metis inference rules for the propositional
fragment using a syntactical treatment of the logic. The \Metis reasoning was
formalized in \Agda~\cite{agdateam} in~\cite{AgdaProp,AgdaMetis}. We built a
proof-reconstruction tool named \Athena~\cite{Athena} written in \Haskell
that is able to generate \Agda proof-terms for \Metis derivations.

This paper has been organized in the following way.
In Section~\ref{sec:limitations-type-theory} some limitations of type theory
are discussed from proof-reconstruction point view.
In Section~\ref{sec:metis-language-and-proofs}, we introduce the
\Metis prover.
In Section~\ref{sec:proof-reconstruction} we show our
approach to reconstruct \Metis proofs.
Related work is described in Section~\ref{sec:related-work}.
Conclusions and suggestions for future work are  presented in
Section~\ref{sec:conclusions}.

% Finally, in appendix \textsc{A}, we provide a small example of the
% proof-reconstruction workflow.

The program and formalizations in \Agda and \Haskell mentioned in this paper are available
as \verb!Git! repositories at \verb!Github!:

\begin{itemize}
  \item The \Athena program that translates proofs generated by \Metis to \Agda:
  \url{http://github.com/jonaprieto/athena}.
  \item The \verb!agda-prop! library as a formalization in \Agda for propositional logic:
  \url{http://github.com/jonaprieto/agda-prop}.
  \item The \verb!agda-metis! library as a formalization in \Agda to justify \Metis proofs of propositional
  logic:
  \url{http://github.com/jonaprieto/agda-prop}.
\end{itemize}

\end{document}
