% -*- root: main.tex -*-
\documentclass[../main.tex]{subfiles}
\begin{document}

% ===================================================================

\section{Introduction}
\label{sec:introduction}

Proof-reconstruction is a hard labor since it depends on the
integration of two complex systems. On the one hand, we have the
automatic theorem provers (henceforth \ATP) and their specification
logic. These tools are usually classified in at least one of the
following categories. A \SAT solver (e.g., \name{zChaff}
\cite{Moskewicz2001} and \name{MiniSat}~\cite{Een2004}) to prove
unsatisfiability of \CNF formulas, a \abbre{QBF} solver (e.g.,
\name{GhostQ}~\cite{Klieber2014} and \name{DepQBF}
\cite{Lonsing2017}) to prove satisfiability and invalidity of
quantified Boolean formulas, a \SMT solver (e.g \name{CVC4}
\cite{Barrett2011}, \name{veriT}~\cite{bouton2009}, and \name{Z3}
\cite{DeMoura2008}) to prove unsatisfiability of formulas from
first-order logic with theories, and a prover for validity of
formulas from first-order logic with equality (e.g., \name{E}
\cite{Schulz:AICOM-2002}, \name{leanCoP}~\cite{Otten2008}, \Metis
\cite{hurd2003first}, \name{SPASS} ~\cite{Weidenbach2009} and
\name{Vampire}~\cite{Riazanov1999}), high-order logic (e.g.,
\name{Leo-II} \cite{Benzmuller2008} and \name{Satallax}
\cite{Brown2012}) or intuitionistic logic (e.g., \name{ileanCoP}
\cite{Otten2008}, \name{JProver} \cite{Schmitt2001}, and
\name{Gandalf}~\cite{Tammet1997}), among others. On the other hand,
we have the proof checkers, interactive theorem provers (henceforth
\ITP) or proof assistants (e.g., \Agda~\cite{agdateam}, \name{Coq}
\cite{coqteam}, \name{Isabelle}~\cite{paulson1994isabelle}, and
\name{HOL4}~\cite{norrish2007hol}). The \ITP tools provide us the
logic framework to check and validate the reply of the \ATPs since
they allow us to define the formal language for the problems like
operators, logic variables, axioms, and theorems.

A proof-reconstruction tool provides such an integration translating
the reply delivered by the prover into the formalism of the proof
assistant. Because the formalism of the source (the evidence
generated by the \ATP) is not necessarily the same logic in the
target, the reconstruction turns out in a ``reverse engineering''
task. Then, reconstructing a proof involves an in-depth
understanding of the algorithms in the \ATP and the specification
logic in the \ITP. To begin with the proof-reconstruction, it is
necessary to have a proof-object from the \ATP in a consistent
format to work with, that is, a full script describing step-by-step
with exhaustive details and without ambiguities, the derivation to
get the actual proof.
For problems in classical propositional logic (henceforth \CPL),
from a list of at least forty\footnote{\ATPs available from the web
service \name{SystemOnTPTP} of the TPTP World.} \ATPs, just a few
provers were able to deliver proofs (e.g., \name{CVC4}
\cite{Barrett2011}, \name{SPASS}, and \name{Waldmeister}
\cite{hillenbrand1997}) and a little bit less replied with a proof
in a file format like \TSTP~\cite{sutcliffe2004tstp} (e.g.,
\name{E}, \Metis, \name{Vampire}, and \name{Z3}), \len{LFSC}
\cite{Stump2008}, or the \len{SMT-LIB}~\cite{Bohme2011} format.

Many approaches have been proposed and some tools have been
implemented for proof-reconstruction in the last decades. These
programs are relevant not only because it helps to spread their
usage but they also increase the confidence of their users about
their algorithms and their correctness (see, for example, bugs in
\ATPs~\cite{Keller2013}, \cite{Bohme2011}, \cite{Fleury2014} and
\cite{Kanso2012}). We mention some of these tools in the related 
work, section \ref{sec:related-work}.

In this paper, we describe the integration of \Metis prover with the
proof assistant \Agda by parsing \TSTP derivations to generate \Agda
proof-terms. We structure the paper as follows. 
We briefly mention notation and syntax of the classical propositional
logic in section \ref{sec:preliminaries}.
In section \ref{sec:metis-language-and-proofs}, we introduce the
\Metis prover and its deduction system.
In section \ref{sec:proof-reconstruction}, we present our
approach to reconstruct proofs deliver by \Metis in type theory.
We listed some conclusions and future work in section
\ref{sec:conclusions}. At the end, we list some related works in 
proof reconstruction in section \ref{sec:related-work}.
The appendix \textsc{A} shows a complete example of the proof 
reconstruction workflow.

\end{document}