
\documentclass[../main.tex]{subfiles}
\begin{document}

% ===================================================================

\section{Introduction (ESCRIBIENDO)}
\label{sec:introduction}

Reconstructing and verifying proofs delivered by automatic theorem provers
(henceforth ATP) are non-trivial tasks.
In the last decades, \ATPs is fast becoming a key instrument in 
different disciplines (\eg verifying cryptographic protocols for information security) and researchers have shown an increased interest to prove the validity of the deduction algorithms.
One reason to say that are the bugs found in many ATPs.
For that reason, in order to give confidence to the user of these systems,
many ATPs has started to include in their outputs, the proofs of the
conclusion. However, existing research recognizes that in many cases these proofs encode non-trivial reasoning
hard to reconstruct them and therefore to verify them.

Hence proof-reconstructing becomes a reverse engineering task
in which the source, the proof, and its quality play an important role.
To begin with the proof-reconstruction, it is necessary to have
the proof of the \ATP in a consistent format, that is, a full script 
describing step-by-step with exhaustive details and without ambiguities, the 
derivation to get the actual proof.  
For classical propositional logic (henceforth \CPL), from a list of at least forty\footnote{\ATPs 
available from the web
service \name{SystemOnTPTP} of the TPTP World.} \ATPs, just a few
provers were able to deliver proofs and fewer like \name{EProver} or \Metis 
replied the proof in an admissible file format.

One approach to address the proof-reconstruction 
is integrating the \ATP, the \emph{source} system, with a proof assistant, the \emph{target} system.
The target system becomes in the proof checker.   
Previous studies have reported proof-reconstruction using proof assistant for classical logic where the development is at a mature stage.
In contrast, we choose a proof assistant for type theory
where proof verification becomes a type-checking task of the
proof-term following the Curry-Howard isomorphism.



% To guarantee the validity of their methods,
% many ATPs has been opted to include proofs in their output.

% The problem address in this paper concern about justifying such proofs.

% One approach to address  these tasks could be use a proof assistant
% to verify the reply of the ATP.



% % During the last decades, the research field of automated reasoning has been experimenting a growth in the development of new methods, tools, and systems. One of the major achievements of this field are the automatic
% % theorem provers (henceforth ATPs).

% At the beginning in its development, the main goal was prove automatically
% problems in different logics (\eg, first-order logic, high-order logic).
% Nowadays, they continuing extending its support, performance and precision
% in their deductive methods. 


% Nevertheless, it has been argued that some of their deductions are not
% valid even in delivering a proof. 

% This problem to verify proofs delivered by ATPs is a non-trivial task
% and involves an in-depth understanding of the algorithms in the \ATP.
% The proof-reconstruction is a non-trivial problem.

% The proof-reconstruction is a non-trivial task since it depends on the
% integration between two complex systems.

% The proof-reconstruction problem aims to solve this issue.
% One way to address this problem is using another major system in this field, the proof assistants.

% A proof-reconstruction tool using this approach will try to verify the proof delivered by the ATP using the proof assistant.

% In this paper, we purpose one of such proof-reconstruction tools.
% Our approach describes the design and implementation to reconstruct
% proofs of the \Metis prover into a proof assistant for type theory like
% \Agda. We use type theory to guarantee that proofs of the \ATP are valid.



% The main disadvantage of X

% One of the main obstacles to reconstruct proof generated by this tools
% are their

% One of the main obstacles to reconstruct proofs generated by
% automatic provers consists of ...

% However,

% The performance of X is limited by ...

% synopsys

%

% Existing research recognizes the critical role by ...

% What we known about X is largely based on ....

% Few published studied have examined the role of

% focus on a full description of the inference rules in the ATP logic.

% No previous studies has given sufficient considerations to ...

% This paper describes the design and implementation of X...

% This study provides new insights into the automatic reasoning of \Metis prover.
% Understanding the link between the ATP and ITP will help xxxx

% DECIR QUE ES NOVEDOSO EN AGDA Y Q NO ESTA HECHO EN TEORIA DE TIPOS
% SYNTATICA VS SEMANTICA.
% APPLICACIONES

% We describe how to reconstruct proofs delivered by an automatic theorem prover
% (henceforth \ATP)
% and an interactive theorem prover (henceforth
% \ITP).

% One way to reconstruct proofs of \ATPs consists of using an \ITP.
% An \ITP system provide us the logic framework to check and validate
% the reply of the \ATPs since they allow us to define the formal language
%  for the problems \ie, operators, logic variables, axioms, and theorems.


% A proof-reconstruction tool provides such an integration, translating
% the reply generated by the prover into the formalism of the proof
% assistant.

% Since the formalism of the source (the evidence
% generated by the \ATP) is not necessarily the same logic in the
% target, the reconstruction turns out to be a ``reverse engineering''
% task.

% Then, reconstructing a proof involves an in-depth
% understanding of the algorithms in the \ATP and the logic specification
% in the \ITP. To begin with the proof-reconstruction, it is
% necessary to have a proof from the \ATP in a consistent
% format, that is, a full script describing step-by-step
% with exhaustive details and without ambiguities, the derivation to
% get the actual proof.

% For problems in classical propositional logic (henceforth \CPL),
% from a list of at least forty\footnote{\ATPs available from the web
% service \name{SystemOnTPTP} of the TPTP World.} \ATPs, just a few
% provers were able to deliver proofs (e.g.,
% \name{CVC4}~\cite{Barrett2011}, \name{SPASS}, and
% \name{Waldmeister}~\cite{hillenbrand1997}) and fewer replied with
% a proof
% in a file format like \TSTP~\cite{Sutcliffe-Schulz-Claessen-VanGelder-2006}
% (e.g., \name{E}, \Metis, \name{Vampire}, and \name{Z3}),
% \len{LFSC}~\cite{Stump2008}, or the \len{SMT-LIB}~\cite{Bohme2011} format.

%  These
% programs are relevant not only because it helps to spread their
% usage but they also increase the confidence of their users about
% their algorithms and their correctness (see, for example, bugs in
% \ATPs~\cite{Keller2013,Bohme2011,Fleury2014} and
% \cite{Kanso2012}).
% We mention some of these tools in Section~\ref{sec:related-work}.

% The recontruction of  proof can play an important role in addressing a issue
% of ...
% Moreover, verification is fast becoming a key instrument in automatic reasoning.

% The type theory provides a useful account of how ...

% No previous study has investigated to reconstruct proofs using a syntactical approach
% in \Agda for any \ATPs.

In this paper, we ported a subset of the \Metis inference rules to
\Agda, the propositional fragmented. The formalization of
these rules allowed us to justify step-by-step \Metis derivations by
translating automatically to \Agda by a the \Athena tool.


This paper has been organized in the following way.
We mention in Section~\ref{sec:limitations-type-theory}
some limitations of the proof-reconstruction from the type theory
point of view. In Section~\ref{sec:metis-language-and-proofs}, we introduce the
\Metis prover and its deduction system.
In Section~\ref{sec:proof-reconstruction}, we show our
approach to reconstruct proofs deliver by \Metis.
Related work is described in Section~\ref{sec:related-work}.
Section~\ref{sec:conclusions} summarizes the main findings and suggest future work.

% Finally, in appendix \textsc{A}, we provide a small example of the
% proof-reconstruction workflow.

The program and formalizations in \Agda mentioned in this paper are available
as \verb!Git! repositories at \verb!Github!:

\begin{itemize}
  \item The \Athena program to translate proofs generated by \Metis to \Agda code:
  \url{http://github.com/jonaprieto/athena}.
  \item The \verb!agda-prop! library as a formal system in \Agda for propositional logic:
  \url{http://github.com/jonaprieto/agda-prop}.
  \item The \verb!agda-metis! library as a formalization in \Agda to justify \Metis proofs in propositional
  logic:
  \url{http://github.com/jonaprieto/agda-prop}.
\end{itemize}

\end{document}
