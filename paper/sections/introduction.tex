
\documentclass[../main.tex]{subfiles}
\begin{document}

% ===================================================================

\section{Introduction}
\label{sec:introduction}

Proof-reconstruction is a non-trivial task since it depends on the
integration between two complex systems.
We work in this study with one integration between an automatic theorem
prover (henceforth \ATP) and an interactive theorem prover (henceforth
\ITP).

The \ITP tools provide us the
logic framework to check and validate the reply of the \ATPs since
they allow us to define the formal language for the problems i.e.,
operators, logic variables, axioms, and theorems.

A proof-reconstruction tool provides such an integration, translating
the reply generated by the prover into the formalism of the proof
assistant.

% Since the formalism of the source (the evidence
% generated by the \ATP) is not necessarily the same logic in the
% target, the reconstruction turns out to be a ``reverse engineering''
% task.

Then, reconstructing a proof involves an in-depth
understanding of the algorithms in the \ATP and the logic specification
in the \ITP. To begin with the proof-reconstruction, it is
necessary to have a proof-object from the \ATP in a consistent
format, that is, a full script describing step-by-step
with exhaustive details and without ambiguities, the derivation to
get the actual proof.
For problems in classical propositional logic (henceforth \CPL),
from a list of at least forty\footnote{\ATPs available from the web
service \name{SystemOnTPTP} of the TPTP World.} \ATPs, just a few
provers were able to deliver proofs (e.g.,
\name{CVC4}~\cite{Barrett2011}, \name{SPASS}, and
\name{Waldmeister}~\cite{hillenbrand1997}) and fewer replied with
a proof
in a file format like \TSTP~\cite{Sutcliffe-Schulz-Claessen-VanGelder-2006}
(e.g., \name{E}, \Metis, \name{Vampire}, and \name{Z3}),
\len{LFSC}~\cite{Stump2008}, or the \len{SMT-LIB}~\cite{Bohme2011} format.

Many approaches have been proposed and some tools have been
implemented for proof-reconstruction in the last decades. These
programs are relevant not only because it helps to spread their
usage but they also increase the confidence of their users about
their algorithms and their correctness (see, for example, bugs in
\ATPs~\cite{Keller2013,Bohme2011,Fleury2014} and
\cite{Kanso2012}).
We mention some of these tools in Section~\ref{sec:related-work}.

In this paper, we ported a subset of the \Metis' inference rules to
\Agda, the propositional fragmented. The formalization of
these rules allowed us to justify step-by-step \Metis' \TSTP derivations by
translating automatically to \Agda by a proof-reconstruction tool built for
this research named \Athena.
This paper has been organized in the following way.
In Section~\ref{ssec:CPL}, we introduce some
notation and syntax of propositional logic.
In Section~\ref{ssec:metis-language-and-proofs}, we introduce the
\Metis prover and its deduction system.
In Section~\ref{sec:proof-reconstruction}, we present our
approach to reconstruct proofs deliver by \Metis in type theory.
Section~\ref{sec:conclusions} summarizes the main findings and includes a
discussion of future work.
Some related works to this paper are presented in Section~\ref{sec:related-work}.
Finally, in appendix \textsc{A}, we provide a small example of the
proof-reconstruction workflow.

The program and formalizations in \Agda described in this paper are available
as \verb!Git! repositories at \verb!Github!:

\begin{itemize}
  \item The \Athena program to translate \Metis proofs to \Agda code:\\
  \hfill\url{http://github.com/jonaprieto/athena}.\hfill
  \item A formal system in \Agda for propositional logic:\\
  \hfill\url{http://github.com/jonaprieto/agda-prop}.\hfill
  \item A formalization in \Agda to justify \Metis proofs in propositional
  logic:\\
  \hfill\url{http://github.com/jonaprieto/agda-prop}.\hfill
\end{itemize}

\end{document}
