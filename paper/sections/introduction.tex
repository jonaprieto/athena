
\documentclass[../main.tex]{subfiles}
\begin{document}

% ===================================================================

\section{Introduction}
\label{sec:introduction}

Reconstructing and verifying proofs delivered by automatic theorem provers
(henceforth ATP) are non-trivial tasks.
In the last decades, \ATPs is fast becoming a key instrument in 
different disciplines (\eg~verifying cryptographic protocols for information security) and researchers have shown an increased interest to prove the validity of their deduction algorithms.
One reason to say that is the bugs found in many ATPs~\cite{Keller2013,Bohme2011,Fleury2014}.
For that reason, in order to give confidence to the user of these systems,
many ATPs has started to include in their outputs, the proof of their
conclusions. However, existing research recognizes that in many cases these proofs encode non-trivial reasoning
hard to reconstruct and therefore hard to verify. The proof-reconstruction
address this problem and becomes in some cases a reverse engineering of the
prover reasoning.

Therefore, the proof replied by the prover and its quality play an important role for proof-reconstruction. To verify such a proof, it is necessary to have
it in a consistent format, that is, a full script 
describing step-by-step with exhaustive details and without ambiguities, the derivation to get the proof.
For classical propositional logic (henceforth \CPL) from a list of at least forty\footnote{\ATPs 
available from the web
service \name{SystemOnTPTP} of the TPTP World.} \ATPs, just a few
provers were able to deliver proofs and fewer like \name{EProver} or \Metis 
replied the proof in an admissible file format.

One approach to address the proof-reconstruction 
is integrating the \ATP, the \emph{source} system, with a proof assistant, the 
\emph{target} system.
The target system becomes in the proof checker. These proof assistants allow 
us to define the formal language used in the proofs, \ie, operators, logic 
variables, axioms, and theorems. A proof-reconstruction tool provides such an
integration, translating the replied generated by the prover into the formalism of the proof assistant.

Previous studies have reported proof-reconstruction using proof assistants for classical logic where the development is at a mature stage.
Even though we choose a proof assistant for type theory
where proof verification becomes a type-checking task of the proof-term following the Curry-Howard isomorphism. Therefore the proof-reconstruction tool generates proof-terms by reconstructing the proof using the formalization 
the prover in type theory. However, this approach suffers from some limitations of type theory as we describe in Section~\ref{sec:limitations-type-theory}.

In this paper, we describe how to reconstruct proofs in type theory
replied by \Metis prover\footnote{v2.3 release 2017102}.
We ported the subset of the \Metis inference rules for the propositional 
fragment using a syntactical treatment of the logic. The \Metis reasoning was formalized in \Agda in~\cite{AgdaProp,AgdaMetis}. We built a proof-reconstruction tool named \Athena~\cite{Athena} that is able to generate \Agda\footnote{v2.5.3} proof-terms for \Metis derivations. We have avoided the use of propositions meanings towards a future work to support other logics
but also contrast the usage of a deep embedding of the logic with
a shallow embedding.

This paper has been organized in the following way.
In Section~\ref{sec:limitations-type-theory} some limitations of type theory
are discussed for proof-reconstruction point view.
In Section~\ref{sec:metis-language-and-proofs}, we introduce the
\Metis prover and its deduction system.
In Section~\ref{sec:proof-reconstruction}, we show our
approach to reconstruct proofs deliver by \Metis.
Related work is described in Section~\ref{sec:related-work}.
Main findings and suggestion for future work is presented Section~\ref{sec:conclusions}.

% Finally, in appendix \textsc{A}, we provide a small example of the
% proof-reconstruction workflow.

The program and formalizations in \Agda mentioned in this paper are available
as \verb!Git! repositories at \verb!Github!:

\begin{itemize}
  \item The \Athena program to translate proofs generated by \Metis to \Agda code:
  \url{http://github.com/jonaprieto/athena}.
  \item The \verb!agda-prop! library as a formal system in \Agda for propositional logic:
  \url{http://github.com/jonaprieto/agda-prop}.
  \item The \verb!agda-metis! library as a formalization in \Agda to justify \Metis proofs in propositional
  logic:
  \url{http://github.com/jonaprieto/agda-prop}.
\end{itemize}

\end{document}
