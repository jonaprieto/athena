
\documentclass[../main.tex]{subfiles}
\begin{document}

% ===================================================================

\section{Introduction (ESCRIBIENDO)}
\label{sec:introduction}


disciplina
en el tiempo
hilighting an import

The main challenge faced of reconstrucing proofs

the main disadavantage of X

One of the main obstacles to reconstruct proofs generated by
automatic provers consists of ...

However,

The performance of X is limited by ...

synopsys

Previous studies have reported proof-reconstruction for

Existing research recognizes the critical role by ...

What we known about X is largely based on ....

Few published studied have examined the role of

focus on a full description of the inference rules in the ATP logic.

No previous studies has given sufficient considerations to ...

This paper describes the design and implementation of X...

This study provides new insights into the automatic reasoning of \Metis prover.
Understanding the link between the ATP and ITP will help xxxx

The reader should bear in mind that the study is based on
reverse engineering.

DECIR QUE ES NOVEDOSO EN AGDA Y Q NO ESTA HECHO EN TEORIA DE TIPOS
SYNTATICA VS SEMANTICA.
APPLICACIONES

% We describe how to reconstruct proofs delivered by an automatic theorem prover
% (henceforth \ATP)
% and an interactive theorem prover (henceforth
% \ITP).

% One way to reconstruct proofs of \ATPs consists of using an \ITP.
% An \ITP system provide us the logic framework to check and validate
% the reply of the \ATPs since they allow us to define the formal language
 % for the problems \ie, operators, logic variables, axioms, and theorems.

% This task is non-trivial since it depends on the
% integration between two complex systems.

% A proof-reconstruction tool provides such an integration, translating
% the reply generated by the prover into the formalism of the proof
% assistant.

% % Since the formalism of the source (the evidence
% % generated by the \ATP) is not necessarily the same logic in the
% % target, the reconstruction turns out to be a ``reverse engineering''
% % task.

% Then, reconstructing a proof involves an in-depth
% understanding of the algorithms in the \ATP and the logic specification
% in the \ITP. To begin with the proof-reconstruction, it is
% necessary to have a proof from the \ATP in a consistent
% format, that is, a full script describing step-by-step
% with exhaustive details and without ambiguities, the derivation to
% get the actual proof.

% For problems in classical propositional logic (henceforth \CPL),
% from a list of at least forty\footnote{\ATPs available from the web
% service \name{SystemOnTPTP} of the TPTP World.} \ATPs, just a few
% provers were able to deliver proofs (e.g.,
% \name{CVC4}~\cite{Barrett2011}, \name{SPASS}, and
% \name{Waldmeister}~\cite{hillenbrand1997}) and fewer replied with
% a proof
% in a file format like \TSTP~\cite{Sutcliffe-Schulz-Claessen-VanGelder-2006}
% (e.g., \name{E}, \Metis, \name{Vampire}, and \name{Z3}),
% \len{LFSC}~\cite{Stump2008}, or the \len{SMT-LIB}~\cite{Bohme2011} format.

 These
% programs are relevant not only because it helps to spread their
% usage but they also increase the confidence of their users about
% their algorithms and their correctness (see, for example, bugs in
% \ATPs~\cite{Keller2013,Bohme2011,Fleury2014} and
% \cite{Kanso2012}).
% We mention some of these tools in Section~\ref{sec:related-work}.

The recontruction of  proof can play an important role in addressing a issue
of ...
Moreover, verification is fast becoming a key instrument in automatic reasoning.

The type theory provides a useful account of how ...

No previous study has investigated to reconstruct proofs using a syntactical approach
in \Agda for any \ATPs.

In this paper, we ported a subset of the \Metis inference rules to
\Agda, the propositional fragmented. The formalization of
these rules allowed us to justify step-by-step \Metis derivations by
translating automatically to \Agda by a the \Athena tool.



This paper has been organized in the following way.
We mention in Section~\ref{sec:limitations-type-theory}
some limitations of the proof-reconstruction from the type theory
point of view. In Section~\ref{sec:metis-language-and-proofs}, we introduce the
\Metis prover and its deduction system.
In Section~\ref{sec:proof-reconstruction}, we show our
approach to reconstruct proofs deliver by \Metis.
Related work is described in Section~\ref{sec:related-work}.
Section~\ref{sec:conclusions} summarizes the main findings and suggest future work.

% Finally, in appendix \textsc{A}, we provide a small example of the
% proof-reconstruction workflow.

The program and formalizations in \Agda mentioned in this paper are available
as \verb!Git! repositories at \verb!Github!:

\begin{itemize}
  \item The \Athena program to translate proofs generated by \Metis to \Agda code:
  \url{http://github.com/jonaprieto/athena}.
  \item The \verb!agda-prop! library as a formal system in \Agda for propositional logic:
  \url{http://github.com/jonaprieto/agda-prop}.
  \item The \verb!agda-metis! library as a formalization in \Agda to justify \Metis proofs in propositional
  logic:
  \url{http://github.com/jonaprieto/agda-prop}.
\end{itemize}

\end{document}
