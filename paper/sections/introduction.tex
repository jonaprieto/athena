
\documentclass[../main.tex]{subfiles}
\begin{document}

% ===================================================================

\section{Introduction}
\label{sec:introduction}

Proof-reconstruction is a hard labor since it depends on the
integration of two complex systems. On the one hand, we have the
automatic theorem provers (henceforth \ATP) and their specification
logic. These tools are usually classified in at least one of the
following categories.
A \SAT solver (e.g., \name{zChaff}~\cite{Moskewicz2001} and
\name{MiniSat}~\cite{Een2004}) to prove unsatisfiability of \CNF
formulas, a \abbre{QBF} solver (e.g.,
\name{GhostQ}~\cite{Klieber2014} and
\name{DepQBF}~\cite{Lonsing2017}) to prove satisfiability and
invalidity of quantified Boolean formulas, a \SMT solver
(e.g \name{CVC4}~\cite{Barrett2011}, \name{veriT}~\cite{bouton2009},
and \name{Z3}~\cite{DeMoura2008}) to prove unsatisfiability of
formulas from first-order logic with theories, and a prover for
validity of formulas from first-order logic with equality
(e.g., \name{E}~\cite{Schulz:AICOM-2002},
\name{leanCoP}~\cite{Otten2008},
\Metis~\cite{hurd2003first},
\name{SPASS}~\cite{Weidenbach2009} and
\name{Vampire}~\cite{Riazanov1999}), high-order logic (e.g.,
\name{Leo-II}~\cite{Benzmuller2008} and
\name{Satallax}~\cite{Brown2012}) or intuitionistic logic (e.g.,
\name{ileanCoP}~\cite{Otten2008},
\name{JProver}~\cite{Schmitt2001}, and
\name{Gandalf}~\cite{Tammet1997}), among others. On the other hand,
we have the proof checkers, interactive theorem provers (henceforth
\ITP) or proof assistants (e.g., \Agda~\cite{agdateam},
\name{Coq}~\cite{coqteam},
\name{Isabelle}~\cite{paulson1994isabelle}, and
\name{HOL4}~\cite{norrish2007hol}). The \ITP tools provide us the
logic framework to check and validate the reply of the \ATPs since
they allow us to define the formal language for the problems i.e.,
operators, logic variables, axioms, and theorems.

A proof-reconstruction tool provides such an integration, translating
the reply generated by the prover into the formalism of the proof
assistant. Since the formalism of the source (the evidence
generated by the \ATP) is not necessarily the same logic in the
target, the reconstruction turns out to be a ``reverse engineering''
task. Then, reconstructing a proof involves an in-depth
understanding of the algorithms in the \ATP and the logic specification
in the \ITP. To begin with the proof-reconstruction, it is
necessary to have a proof-object from the \ATP in a consistent
format, that is, a full script describing step-by-step
with exhaustive details and without ambiguities, the derivation to
get the actual proof.
For problems in classical propositional logic (henceforth \CPL),
from a list of at least forty\footnote{\ATPs available from the web
service \name{SystemOnTPTP} of the TPTP World.} \ATPs, just a few
provers were able to deliver proofs (e.g.,
\name{CVC4}~\cite{Barrett2011}, \name{SPASS}, and
\name{Waldmeister}~\cite{hillenbrand1997}) and fewer replied with
a proof
in a file format like \TSTP~\cite{Sutcliffe-Schulz-Claessen-VanGelder-2006}
(e.g., \name{E}, \Metis, \name{Vampire}, and \name{Z3}),
\len{LFSC}~\cite{Stump2008}, or the \len{SMT-LIB}~\cite{Bohme2011} format.

Many approaches have been proposed and some tools have been
implemented for proof-reconstruction in the last decades. These
programs are relevant not only because it helps to spread their
usage but they also increase the confidence of their users about
their algorithms and their correctness (see, for example, bugs in
\ATPs~\cite{Keller2013,Bohme2011,Fleury2014} and
\cite{Kanso2012}).
We mention some of these tools in Section~\ref{sec:related-work}.

In this paper, we ported a subset of the \Metis' inference rules to
\Agda, the propositional fragmented without equality. The formalization of
these rules allowed us to justify step-by-step \Metis' \TSTP derivations by
translating automatically to \Agda by a proof-reconstruction tool built for
this research named \Athena.
This paper has been organized in the following way.
The first section of the paper, Section~\ref{ssec:CPL}, will mention some
notation and syntax of the classical propositional logic.
In Section~\ref{ssec:metis-language-and-proofs}, we introduce the
\Metis prover and its deduction system.
In Section~\ref{sec:proof-reconstruction}, we present our
approach to reconstruct proofs deliver by \Metis in type theory.
Section~\ref{sec:conclusions} summarizes the main findings and includes a
discussion of future work.
Some related works to this paper are presented in Section~\ref{sec:related-work}.
Finally, in appendix \textsc{A}, we provide a small example of the
proof-reconstruction workflow.

\end{document}
