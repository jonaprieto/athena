%
\documentclass[../main.tex]{subfiles}
\begin{document}

% ===================================================================

% \begin{myremark}
% We address proof-reconstruction in this paper for refutation proofs.
% We annotate \abbre{CPL} problems using \verb!fof! formulas that
% \Metis converts to \abbre{CNF} clauses to look for a refutation.
% \end{myremark}


\section{Metis: Language and Proofs}
\label{sec:metis-language-and-proofs}

\Metis is an automatic theorem prover for first-order logic with
equality developed by Hurd~\cite{hurd2003first}. This prover is
suitable for proof-reconstruction since it provides well-documented
proofs to justify its deduction steps from the basis of only six
inference rules for first-order logic (see, for example,
\cite{paulson2007source,Farber2015}). For the propositional fragment,
\Metis has three inference rules, see Fig.~\ref{fig:metis-inferences}.

\begin{figure}
\begin{equation*}
  \begin{bprooftree}
    \AxiomC{}
    \RightLabel{axiom}
    \UnaryInfC{$Γ ⊢ φ$}
  \end{bprooftree}
  \qquad
  \begin{bprooftree}
    \AxiomC{}
    \RightLabel{assume $φ$}
    \UnaryInfC{$Γ ⊢ φ ∨ ¬ φ$}
  \end{bprooftree}
  \end{equation*}
  \vskip2mm
  \begin{equation*}
  \begin{bprooftree}
    \AxiomC{$Γ ⊢ l ∨ φ$}
    \AxiomC{$Γ ⊢ ¬ l ∨ ψ$}
    \RightLabel{resolve $l$}
    \BinaryInfC{$Γ ⊢ φ ∨ ψ$}
  \end{bprooftree}
\end{equation*}
\caption{Propositional logic inference rules of the \texttt{Metis} prover.}
\label{fig:metis-inferences}
\end{figure}

% -------------------------------------------------------------------

\subsection{Input language}
\label{ssec:input-language}

The \TPTP language is the input language to encode problems used by \Metis.
It includes the first-order form (denoted by \name{fof}) and clause normal form (denoted by \name{cnf}) formats~\cite{sutcliffe2009}.
The \TPTP syntax describes a well-defined grammar to handle annotated
formulas with the following form:
\begin{verbatim}
  language(name, role, formula).,
\end{verbatim}
where the \name{language} can be \name{fof} or \name{cnf}. The
\name{name} serves to identify the formula within the problem. Each
formula assumes a \name{role}, this could be an \name{axiom},
\name{conjecture}, \name{definition}, \name{plain} or an
\name{hypothesis}.  The formulas include the constants
\verb!$true! and \verb!$false!, the negation unary operator~\verb!~!,
and the binary connectives \verb!&!, \verb!|!, \verb!=>!, and
\verb!<=>! to represent $⊤$, $⊥$, $¬$, $∧$, $∨$, $⇒$, and $⇔$
respectively.

\begin{myexamplenum}
  For instance, let us express the problem
  $p\, ⊢ ¬ (p ∧ ¬ p) ∨ (q ∧ ¬ q)$ in \TPTP syntax. We begin by
  declaring the $p$ axiom using the \verb!axiom!  keyword. Next, we
  include the expected conclusion using the \verb!conjecture! keyword.
\begin{verbatim}
  fof(h, axiom, p).
  fof(goal, conjecture, ~ ((p & ~ p) | (q & ~ q))).
\end{verbatim}
\end{myexamplenum}


\subsection{Output language}
\label{ssec:output-language}

The \TSTP language is an output language for derivations of
ATPs~\cite{Sutcliffe-Schulz-Claessen-VanGelder-2006}.
A \TSTP derivation is a directed acyclic graph, a proof tree,
where each leaf is a formula from the \TPTP input. A node is a formula
inferred from the parent formulas. The root is the final derived formula,
such a derivation is a list of annotated formulas with the following form:

\begin{verbatim}
  language(name, role, formula, source [,useful info]).
\end{verbatim}

where the \name{source} field is an inference record with the following
pattern:

\begin{verbatim}
  inference(rule, useful info, parents).
\end{verbatim}

The \name{rule} in the line above stands for the inference name;
the other fields are supporting arguments or useful information to
apply the reasoning step, and list the parents nodes.

\begin{myexamplenum}

In the script below, \strip is the name of the inference.
It has no arguments and derives from one parent node named  \verb!goal!. The
result of this inference when it applies to the \verb!goal! formula is
\verb!p!.

\begin{verbatim}
  fof(subgoal_0, plain, p, inference(strip, [], [goal])).
\end{verbatim}
\end{myexamplenum}

\begin{notation}

We adopt a customized \TSTP syntax to keep as short as possible the \Metis
derivations for increasing the readability in this paper.

\begin{myexamplenum}
\label{fig:metis-example-tree}

Let us consider the following \TSTP derivation using
the customized \TSTP syntax (see the original \TSTP derivation and customizations in Appendix~\ref{app:tstp-syntax}).

\begin{verbatim}
  fof(premise, axiom, p).
  fof(goal, conjecture, p).
  fof(sg0, p, inf(strip, [goal])).
  fof(neg0, ¬ p, inf(negate, [sg0])).
  fof(norm0, ¬ p, inf(canonicalize, [neg0])).
  fof(norm1, p, inf(canonicalize, [premise])).
  fof(norm2, ⊥, inf(simplify, [norm0, norm1]))
  cnf(refute0, ⊥, inf(canonicalize, [norm2])).
\end{verbatim}

\end{myexamplenum}
\end{notation}

% -------------------------------------------------------------------

\subsection{Proofs}
\label{ssec:metis-proofs}

A proof generated by \Metis encodes a natural
deduction proof.
With the inference rules in Fig.~\ref{tab:agda-metis-table}
as the only valid deduction steps, \Metis
attempts to prove conjectures by refutation (\eg,
\emph{falsium} in the root of the \TSTP derivation).

These proofs are directed acyclic graphs, trees of refutations
Each node stands for an application of an inference rule and the leaves
in the tree represent formulas in the given problem. Each node is
labeled with a name of the inference rule (\eg, \canonicalize).
Each edge links a premise with one conclusion.
The proof graphs have at their root the conclusion
$⊥$, since \Metis proofs are refutations.

\begin{figure}[!ht]
\centering
  \begin{bprooftree}\tt
    \AxiomC{}
    \RightLabel{negate}
    \UnaryInfC{$\neg p$}
    \RightLabel{strip}
    \UnaryInfC{$\neg p$}
    \AxiomC{}
    \RightLabel{axiom}
    \UnaryInfC{$p$}
    \RightLabel{canonicalize}
    \UnaryInfC{$p$}
    \RightLabel{simplify}
    \BinaryInfC{$⊥$}
    \RightLabel{canonicalize}
    \UnaryInfC{$⊥$}
  \end{bprooftree}
  \caption{\Metis proof tree of derivation in Example~\ref{fig:metis-example-tree}.}
  \label{fig:metis-example}
\end{figure}

\subsection{Inference rules}
\label{ssec:metis-inferences-rules}

We present the list of inference rules used by \Metis in
the \TSTP derivations for propositional logic
in Table~\ref{tab:agda-metis-table}.
We reconstruct these rules in Section~\ref{sec:proof-reconstruction}.
The reader may notice that the
inference rules presented in Fig~\ref{fig:metis-inferences} diverge from the
rules in aforementioned table. However, the former rules are
implemented by the latter rules. For instance, as far as we know, in
\TSTP derivations, \canonicalize, \clausify, \conjunct, and \simplify, are all
the rules that represent the \emph{axiom} rule by \Metis.

We first present the \strip inference rule since it is the rule that appears
first after each conjecture. The other rules are sorted mainly follow their
level of complexity of their definitions and the formalization presented in
Section~\ref{ssec:emulating-inferences}. Some inference rules depend on the
formalizations of other rules. For instance, the \simplify rule and the
\clausify rule need theorems developed for the \canonicalize rule. The
\canonicalize rule needs theorems developed in the \resolve section. But the
\resolve rule depends on the \conjunct rule.

% \Metis looks for a refutation to prove \verb!fof! problems.
% A \verb!fof! problem contains the conjecture, the \emph{goal}, in the
% problem. This goal is stripped into subgoals by using the \strip rule. The
% subgoals are the conjuncts in a general conjunction that is equivalent to the
% aforementioned goal.
% Therefore, to prove the goal, \Metis proves independently each subgoal since
% they tend to be less complex or even fewer propositions. At the end, the
% conjunction of all subgoal proofs must prove the goal. Nevertheless, this last
% reasoning is not explicit neither in the \TSTP derivations nor in the
% documentation available about \Metis.
% Since \Metis replies refutation proofs, to prove each subgoal, in the \TSTP
% derivation, the inference rule \negate introduces the negation of such
% subgoals. The \negate rule is a specification of the \emph{assume} rule.

\begin{table}[!ht]

\caption{\Metis inference rules.}
  \begin{center}
  {\renewcommand{\arraystretch}{1.6}%
    \label{tab:agda-metis-table}
    \begin{tabular}{|@{\hspace{2mm}}l@{\hspace{2mm}}l@{\hspace{2mm}}c@{\hspace{2mm}}|}
    \hline
    \textbf{\Metis rule} & \textbf{Purpose} &\textbf{Theorem number}\\ \hline

      \texttt{strip}
      &Strip a goal into subgoals
      &\ref{thm:strip}
      \\

      \texttt{conjunct}
      &Takes a formula from a conjunction
      &\ref{thm:conjunct}
      \\

      \texttt{resolve}
      &A general form of the resolution theorem
      &\ref{thm:resolve}
      \\

      \texttt{canonicalize}
      &Normalization of the formula
      &\ref{thm:canonicalize}
      \\

      \texttt{clausify}
      &Performs clausification
      &\ref{thm:clausify}
      \\

      \texttt{simplify}
      &Simplify definitions and theorems
      &\ref{thm:simplify}
      \\[1ex]
    \hline
    \end{tabular}}
  \end{center}
\end{table}

\end{document}
