% -*- root: main.tex -*-
\documentclass[../main.tex]{subfiles}
\begin{document}

% ===================================================================

\section{Proof Reconstruction}
\label{sec:proof-reconstruction}

\begin{figure}
\centering

\begin{tikzpicture}[scale=0.9]
\node[text width=2cm, align=center](problem) at (0,0)
  {1. CPL \\ Problem.};

\node[right = 1.1cm of problem, text width=2cm, align=center]
  (tptp){2. \TPTP \\ Problem.};

\node[right= 1.1cm of tptp, text width=2cm, align=center]
   (metis) {3. \Metis \\ Prover.};

\node[right= 1.1cm of metis, text width=2cm, align=center, inner sep=10pt]
   (tstp) {4. \TSTP \\ Derivation.};

\node[below= 0.5cm of tstp, text width=2cm, align=center, inner sep=10pt]
  (athena) {5. \Haskell \\ Traslator.};

\node[left = 1.1cm of athena, text width=2cm, align=center]
   (agdafile) {6. \Agda \\ Proof.};

\node[left = 1.1cm of agdafile, text width=2cm, align=center]
   (agda) {7. Proof Checking.};

\node[below = 0.5cm of problem, text width=2cm, align=center]
   (verified) {8.1 Proof \\ Checked.};

\node[below = 0.5cm of verified, text width=2cm, align=center]
   (failure) {8.2 Failure.};

% node[below] {send to}
\draw[->, thick] (problem) to
  % node[below] {\tiny encoding}
  (tptp);
\draw[->, thick] (tptp) to
  % node[below] {\tiny }
  (metis);
\draw[->, thick] (metis) to
  % node[below] {\tiny replies on}
  (tstp);
\draw[->, thick] (tstp) to
  % node[right] {\tiny parsing}
  (athena);
\draw[->, thick] (athena) to
  % node[below] {\tiny traslation}
  (agdafile);
\draw[->, thick] (agdafile) to
  % node[below] {\tiny type-checking}
  (agda);
\draw[->, thick] (agda) to (verified);
\draw[->, thick, gray] (agda) to (failure);
\end{tikzpicture}
\caption{Proof reconstruction overview.}
\label{fig:proof-reconstruction-workflow}
\end{figure}

The proof reconstruction approach proposed here consists of a series 
of steps similar to the workflow presented by Sultana in
\cite{sultana2015}. This process is a translation from a source 
system to a target system. In our case, the system of origin, the 
automatic theorem prover, is \Metis; the target system is the proof 
assistant, \Agda. We choose \Agda, but another proof assistant with 
support of type theory and features similar like \Agda is valid; our 
results do not depend on the proof assistant but the support of inductive data types, and termination checking, and type-checking.
\improvement{Make this statement stronger and clear.}

Following the diagram in Fig.
\ref{fig:proof-reconstruction-workflow}, the process can begin with 
a problem in \CPL that we encode in \TPTP format to be sent as the 
input of the \Metis prover.
The proof-search algorithms of \Metis in step No. 3 perform a 
search for a solution; if they success, \Metis 
replies a derivation in \TSTP format of the proof, else \Metis may 
no stop running forever or replies the problem is counter feasible.

With such a derivation, we process it with our \Haskell translator 
tool, \Athena~\cite{Athena}.
\Athena parses the \TSTP format, analyze the 
derivation and generates a representation of the natural deduction 
proof using a version of a tree data structure (see the properties 
of this tree in subsection \ref{ssec:metis-proofs}). As a result,
\Athena generates an \Agda file with such a proof. We have included 
in this file, imports for \Agda libraries that give the support for 
the logic~\cite{AgdaProp}, and versions of the \Metis' inference 
rules~\cite{AgdaMetis}. 

Finally, we use the proof assistant in step No. 7 to type-check the 
proof and verify its correctness. We have only one answer from two 
possible answers: type-checking succeed (i.e., the proof is 
valid) or type-checking failed.
If the type-check fails, one of the three actors are 
responsible: \Metis, \Athena, or \Agda. We are responsible for 
errors committed in the translation from \TSTP to \Agda; lack of the 
coverage of the \Metis' inference rules, or bad assumptions about 
them.

In the rest of this section, we provide a formal description using 
type theory to build definitions and theorems for the functions 
necessary to emulate the \Metis' inference rules. 

\end{document}