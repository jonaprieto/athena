
\documentclass[../main.tex]{subfiles}
\begin{document}

% ===================================================================

\section{Related Work}
\label{sec:related-work}

Let us mention a list of some representative of such tools mainly based on the
description by \citeauthor{Sicard-Ramirez2016}~\cite{Sicard-Ramirez2016}.

The \name{Isabelle} proof assistant has the \name{Sledgehammer} tool.
This program provides a full integration between
automatic theorem provers~\cite{meng2006automation,blanchette2013extending,Fleury2014,hurlin07practical,bohme2010,blanchette2013extending} and
\name{Isabelle/HOL}~\cite{nipkow2002isabelle}, the specialization of
\name{Isabelle} for higher-order logic.
A modular proof-reconstruction workflow is presented jointly with
the full integration of \name{Leo-II} and \name{Satallax} provers with
\name{Isabelle/HOL} in \citeauthor{Een2004}~\cite{Een2004}.

\name{SMTCoq}~\cite{armand2011,Ekici2017} is a tool for the
\name{Coq} proof assistant \cite{coqteam} which provides a certified
checker for proof witness coming from the \SMT solver
\name{veriT}~\cite{bouton2009} and adds a new tactic named verit,
that calls \name{veriT} on any \name{Coq} goal.
In \cite{Bezem2002},
given a fixed but arbitrary first-order signature, the authors
transform a proof produced by the first-order automatic theorem
prover \name{Bliksem} \cite{deNivelle2003} in a \name{Coq} proof
term.

Hurd~\cite{Hurd1999} integrates the first-order resolution prover
\name{Gandalf} with the high-order
theorem prover \name{HOL}~\cite{norrish2007hol}.
Its \verb!GANDALF_TAC! tactic is able to reconstruct \name{Gandalf} proofs
by using a \abbre{LCF} model. For \name{HOL Light}, a version of
\name{HOL} but with a simpler logic core, the \SMT solver \name{CVC4}
was integrated. \citeauthor{kaliszyk2013}~\cite{kaliszyk2013}
reconstruct proofs from different \ATPs with the \name{PRocH} tool by
replaying detailed inference steps from the \ATPs with internal
inference methods implemented in \name{HOL Light}.

For the proof assistant \Agda, we found two proof-reconstruction tools.
First, the integration with the \name{Waldmeister}~\cite{hillenbrand1997}
prover by \citeauthor{foster2011integrating}~\cite{foster2011integrating}.
A prover for pure equational logic, i.e.
first-order logic with equality but no other predicate symbols and no
functions symbols.
Second, Kanso and Setzer in~\cite{kanso2016light} integrate the propositional
fragment of the \name{EProver}. The tool presented by the authors
translate automatically propositional logic problems written in \Agda
to \TPTP problems. By calling an external wrapper of the \ATP, this
tool gets a \TSTP derivation. The tool parses the \name{EProver}
derivation and outputs \Agda code.

\end{document}
