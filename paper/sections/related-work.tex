
\documentclass[../main.tex]{subfiles}
\begin{document}

% ===================================================================

\section{Related Work}
\label{sec:related-work}

\name{Sledgehammer} is a tool for \name{Isabelle} proof assistant
\cite{paulson1994isabelle} that provides a full integration of
automatic theorem provers including \ATPs (see, for example,
\cite{meng2006automation}, \cite{blanchette2013extending} and
\cite{Fleury2014}) and \SMT solvers (see, for instance,
\cite{hurlin07practical}, \cite{bohme2010},
\cite{blanchette2013extending}, and \cite{Fleury2014}) with
\name{Isabelle/HOL} \cite{nipkow2002isabelle}, the specialization of
\name{Isabelle} for higher-order logic.
\citeauthor{Een2004}~\cite{Een2004} integrates \name{Leo-II} and
\name{Satallax}, two theorem provers for high-order logic with
\name{Isabelle/HOL} proposing a modular proof-reconstruction work
flow.

\name{SMTCoq}~\cite{armand2011,Ekici2017} is a tool for the
\name{Coq} proof assistant \cite{coqteam} which provides a certified
checker for proof witness coming from the \SMT solver
\name{veriT}~\cite{bouton2009} and adds a new tactic named verit,
that calls \name{veriT} on any \name{Coq} goal.
In \cite{Bezem2002},
given a fixed but arbitrary first-order signature, the authors
transform a proof produced by the first-order automatic theorem
prover \name{Bliksem} \cite{deNivelle2003} in a \name{Coq} proof
term.

Hurd~\cite{Hurd1999} integrates the first-order resolution prover
\name{Gandalf} with \name{HOL}~\cite{norrish2007hol}, a high-order
theorem prover, following an \abbre{LCF} model implementing the
tactic \verb!GANDALF_TAC!. For \name{HOL Light}, a version of
\name{HOL} but with a simpler logic core, the \SMT solver \name{CVC4}
was integrated, and \citeauthor{kaliszyk2013}~\cite{kaliszyk2013}
reconstruct proofs from different ATPs with their \name{PRocH} tool,
replaying the detailed inference steps from the ATPs with internal
inference methods implemented in \name{HOL Light}.

% Taken from: % Färber, M., & Kaliszyk, C. (2015). Metis-based
% Paramodulation Tactic for HOL

% Light. In GCAI 2015. Global Conference on Artificial Intelligence
% Metis-based

% (Vol. 36, pp. 127–136).

% HOL(y)Hammer [KU15] is an automated deduction framework for HOL4 and
% HOL Light.

% Given a conjecture, it attempts to find suitable premises, then
% calls external ATPs such as E[Sch13], Vampire [KV13], or Z3 [dMB08],
% and attempts to reconstruct the proof using the premises used by the
% ATP. To reconstruct proofs, it uses tactics such as MESON,
% simplification, and a few other decision procedures, however, these
% are sometimes not p owerful enough to reconstruct proofs found by
% the external ATPs.

\name{Waldmeister} is an automatic theorem prover for unit
equational logic \cite{hillenbrand1997}.
\citeauthor{foster2011integrating}~\cite{foster2011integrating}
integrate \name{Waldmeister} into \Agda
\cite{agdateam}. This integration requires a proof-reconstruction
step, but authors' approach is restricted to pure equational logic
--also called identity theory~\cite{humberstone2011}-- that is,
first-order logic with equality but no other predicate symbols and
no functions symbols~\cite{appel1959}.

Kanso and Setzer~\cite{kanso2016light} integrate the propositional
fragment of \name{EProver} in \Agda. In this work, the authors
translate automatically propositional logic problems written in \Agda
to \TPTP problems. By calling an external wrapper of the ATP, this
tool gets a \TSTP derivation. The tool parses the \name{EProver}
derivation and outputs \Agda code.
We are able to say their proof-reconstruction is almost
similar as the approach presented in Section~\ref{ssec:workflow}.
Nevertheless, Kanso and Setzer aim to justify some of the inference
rules by using propositional semantics instead of only syntactic
treatment as shown above.

\end{document}
