
\documentclass[../main.tex]{subfiles}
\begin{document}

% ===================================================================

\section{Related Work}
\label{sec:related-work}

Many approaches have been proposed for proof-reconstruction and some tools have been implemented in the last decades.

In type theory, for \Agda and  for the same propositional fragmented used in this paper,
we found the proof-reconstruction proposed by \citeauthor{Kanso2012}
in~\cite{Kanso2012,kanso2016light}. The author present an approach similar to
the reasoning presented in Section~\ref{ssec:workflow}.
Nonetheless, his approach includes semantics reasoning that as we mentioned
in the introduction differs completely from our proof-reconstruction approach.

In \Agda but with another logic, \citeauthor{foster2011integrating}~\cite{foster2011integrating}
describe the proof-reconstruction for \name{Waldmeister}~\cite{hillenbrand1997}
prover to pure equational logic, \ie,
first-order logic with equality but no other predicate symbols and no
functions symbols.

In \name{Coq}~\cite{coqteam}, and proof assistant for type theory like \Agda, 
we found the \name{SMTCoq}~\cite{armand2011,Ekici2017} tool which provides a
certified checker for proof witness coming from the \SMT solver
\name{veriT}~\cite{bouton2009} and adds a new tactic named verit,
that calls \name{veriT} on any \name{Coq} goal.
Also for \name{Coq}, 
given a fixed but arbitrary first-order signature,
\citeauthor{Bezem2002} in \cite{Bezem2002}
transform a proof produced by the first-order automatic theorem
prover \name{Bliksem}~\cite{deNivelle2003} in a \name{Coq} proof
term.

We found many successful attemps using proof assistant for classical logic
instead of type theory.
Let us mention some representative of such tools. This descriptions is
mainly based on~\citeauthor{Sicard-Ramirez2016}~\cite{Sicard-Ramirez2016}.

The \name{Isabelle} proof assistant has the \name{Sledgehammer} tool.
This program provides a full integration between
automatic theorem provers~\cite{blanchette2013extending,Fleury2014,bohme2010} and
\name{Isabelle/HOL}~\cite{nipkow2002isabelle}, the specialization of
\name{Isabelle} for higher-order logic.
A modular proof-reconstruction workflow is presented jointly with
the full integration of \name{Leo-II} and \name{Satallax} provers with
\name{Isabelle/HOL} in \citeauthor{Een2004}~\cite{Een2004}.

\citeauthor{Hurd1999}~\cite{Hurd1999} integrates the first-order resolution prover
\name{Gandalf} with the high-order
theorem prover \name{HOL}~\cite{norrish2007hol}.
Its \verb!GANDALF_TAC! tactic is able to reconstruct \name{Gandalf} proofs
by using a \abbre{LCF} model. For \name{HOL Light}, a version of
\name{HOL} but with a simpler logic core, the \SMT solver \name{CVC4}
was integrated. \citeauthor{kaliszyk2013}~\cite{kaliszyk2013}
reconstruct proofs from different \ATPs with the \name{PRocH} tool by
replaying detailed inference steps from the \ATPs with internal
inference methods implemented in \name{HOL Light}.

\end{document}
