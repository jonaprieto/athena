
\documentclass[../paper.tex]{subfiles}
\begin{document}

% ===================================================================

\section{Conclusions}
\label{sec:conclusions}

We presented a proof-reconstruction approach in type theory for the
propositional fragment of the \Metis prover. In
Section~\ref{ssec:emulating-inferences}, we provided for each \Metis inference
rule a formal description in type theory following a syntactical approach
type-checked in the proof-assistant \Agda in~\cite{AgdaProp,AgdaMetis}. With
this formalization, we were able to type-check \Agda proof-terms for
\TSTP derivations generated by \Metis. The \Agda proof-terms were generated by
the \Athena translator~\cite{Athena}, a tool written in \Haskell.

% This
% approach differs from our proof-reconstruction since we left out completely the
% use of propositions meanings towards a future work to support other logics where
% a syntactical approach plays an important role~(we refere the reader to
% \cite{Agudelo-Agudelo2017} for some examples)

Our approach to reconstruct the proofs for \Metis \TSTP derivations only used
syntactical aspects of the logic. We chose that syntactical treatment instead of
using semantics to extend this work towards the support of first-order logic or
other non-classical logics. For instance, let us recall that
for first-order logic, satisfiability is undecidable and its syntactical aspect
plays an important role to reconstruct its proofs.

Lastly, we strongly believe that by justifying proofs by theorem provers, we
help to increase their trustworthiness. In this work, we increased the
trustworthiness for the automatic prover \Metis. The reverse engineering to
grasp the reasoning of a theorem prover can help to reveal issues in these
systems maybe unkwon even for their own developers. During this research, we had
the opportunity to contribute to \Metis by reporting some bugs---see Issues
No.~2, No.~4, and commit \name{8a3f11e} in \Metis' official
repository.\footnote{\url{https://github.com/gilith/metis}.} Fortunately, all
these problems were fixed quickly by Hurd in \Metis~2.3 (release~20170822).
\end{document}
