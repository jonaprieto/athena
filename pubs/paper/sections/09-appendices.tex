
\documentclass[../paper.tex]{subfiles}
\begin{document}
%
\clearpage
%
\begin{subappendices}
%
\renewcommand{\thesection}{\Alph{section}}%
%
\section{Customized TSTP syntax}
\label{app:tstp-syntax}

We adopted a special \TSTP syntax to improve the readability of the \TSTP
examples shown in this document. Some of the modifications to the original
presentation of \TSTP syntax in Section~\ref{ssec:output-language} are the
following.

\begin{itemize}
  \item The formulas names are sub indexed (\eg, instead of \verb!axiom_0!,
  we write \verb!axiom₀!).
  \item We use \verb!inf! instead of \verb!inference! field.
  \item We shorten names generated automatically by \Metis, (\eg,
\verb!s₀! instead of \verb!subgoal_0! or \verb!n₀!
instead of \verb!normalize_0!).
  \item We remove the \verb!plain! role.
  \item We remove empty fields in the inference information.
  \item The brackets in the argument of a unary inference are removed (\eg,
instead of \verb!inf(rule, [], [n₀]))!, we write
\verb!inf(rule, [], n₀))!).
  \item If the inference rule does not need arguments except its parent nodes,
  we remove the field of useful information (\eg,
 \verb!inf(canonicalize, premise)! instead of
 \verb!inf(canonicalize, [], premise)!).
  \item We use the symbols (⊤, ⊥, ¬, ∧, ∨, ⇒) for formulas instead of
  (\verb!$false!, \verb!$true!, \verb!~!, \verb!&!, \verb!|!, \verb!=>!) \TPTP symbols.
  \item When the purpose to show a \TSTP derivation does not include
  some parts of the derivation we use the ellipsis (\verb!...!) to avoid
  such unnecessary parts.
\end{itemize}

For example, let us consider the \TSTP derivation generated by \Metis
in Fig.~\ref{fig:metis-proof-tstp} and its customized version in
Fig.~\ref{fig:metis-proof-tstp-customized}

\begin{figure}
\begin{verbatim}
  fof(premise, axiom, p).
  fof(goal, conjecture, p).
  fof(subgoal_0, plain, p, inference(strip, [], [goal])).
  fof(negate_0_0, plain, ~ p, inference(negate, [], [subgoal_0])).
  fof(normalize_0_0, plain, ~ p,
    inference(canonicalize, [], [negate_0_0])).
  fof(normalize_0_1, plain, p,
    inference(canonicalize, [], [premise])).
  fof(normalize_0_2, plain, $false,
    inference(simplify, [], [normalize_0_0, normalize_0_1]))
  cnf(refute_0_0, plain, $false,
    inference(canonicalize, [], [normalize_0_2])).
\end{verbatim}
\caption{\texttt{Metis}' \texttt{TSTP} derivation for the
problem $p\vdash p$.}
\label{fig:metis-proof-tstp}
\end{figure}

\clearpage
\begin{figure}[!ht]
\begin{verbatim}
  fof(premise, axiom, p).
  fof(goal, conjecture, p).
  fof(s₀, p, inf(strip, goal)).
  fof(neg₀, ¬ p, inf(negate, s₀)).
  fof(n₀, ¬ p, inf(canonicalize, neg₀)).
  fof(n₁, p, inf(canonicalize, premise)).
  fof(n₂, ⊥, inf(simplify, [n₀, n₁]))
  cnf(r₀, ⊥, inf(canonicalize, n₂)).
\end{verbatim}
\caption{\texttt{Metis}' \texttt{TSTP} derivation using a customized syntax}
\label{fig:metis-proof-tstp-customized}
\end{figure}
\vfill

% \clearpage
% \section{Bounded Recursion of the Strip Function}
% \label{app:strip-cm}
%
% In Section~\ref{sssec:strip-a-goal} we describe the
% \strip function to get the subgoals of a certain a goal.
% We define the first version of this function with the
% $\fstrip₀$ function in~\eqref{eq:strip-zero} but the reader can note that this function is not a structurally recursive function.
% Therefore, we define a structurally recursive function of this function in~\eqref{eq:strip-fixed}.
%
% \begin{equation}
% \label{eq:strip-zero}
% \begin{aligned}
% &\hspace{.495mm}\fstrip₀ : \Prop → \Prop\\
% &\begin{array}{llll}
% \fstrip₀ &(φ₁ ∧ φ₂)     &= \fuh~(\fstrip₀~φ₁) ∧ \fuh~(φ₁ ⇒ \fstrip₀~φ₂)\\
% \fstrip₀ &(φ₁ ∨ φ₂)     &= \fuh~((¬ φ₁) ⇒ \fstrip₀~φ₂)\\
% \fstrip₀ &(φ₁ ⇒ φ₂)     &= \fuh~(φ₁ ⇒ \fstrip₀~φ₂)\\
% \fstrip₀ &(¬ (φ₁ ∧ φ₂)) &= \fuh~(φ₁ ⇒ \fstrip₀~(¬ φ₂))\\
% \fstrip₀ &(¬ (φ₁ ∨ φ₂)) &= \fuh~(\fstrip₀~(¬ φ₁)) ∧ \fuh~((¬ φ₁) ⇒ \fstrip₀~(¬ φ₂))\\
% \fstrip₀ &(¬ (φ₁ ⇒ φ₂)) &= \fuh~(\fstrip₀~φ₁) ∧ \fuh~(φ₁ ⇒ \fstrip₀~(¬ φ₂))\\
% \fstrip₀ &(¬ (¬ φ₁))    &= \fuh~(\fstrip₀~φ₁)\\
% \fstrip₀ &(¬ ⊥)         &= ⊤\\
% \fstrip₀ &(¬ ⊤)         &= ⊥\\
% \fstrip₀ &φ             &= φ.
% \end{array}
% \end{aligned}
% \end{equation}
%
% The complexity measure of $\fstrip₀$ is given by the $\fstrip_{cm}$ function
% defined in~\eqref{eq:strip-cm}.
%
% \begin{equation*}
% \label{eq:strip-cm}
% \begin{aligned}
% &\hspace{.495mm}\fstrip_{cm} : \Prop \to \Nat \to \Prop\\
% &\begin{array}{lll}
% \fstrip_{cm} &(φ₁ ∧ φ₂)      &= \textrm{max}~(\fstrip_{cm}~φ₁)~(\fstrip_{cm}~φ₂) + 1\\
% \fstrip_{cm} &(φ₁ ∨ φ₂)      &= \fstrip_{cm}~φ₂ + 1\\
% \fstrip_{cm} &(φ₁ ⇒ φ₂)      &= \fstrip_{cm}~φ₂ + 1\\
% \fstrip_{cm} &(¬ ⊤)          &= \fstrip_{cm}~(¬ φ₂) + 1\\
% \fstrip_{cm} &(¬ ⊥)          &= \textrm{max}~(\fstrip_{cm}~(¬ φ₁))~(\fstrip_{cm}~(¬ φ₂)) + 1\\
% \fstrip_{cm} &(¬ (φ₁ ∧ φ₂))  &= \textrm{max}~(\fstrip_{cm}~φ₁)~(\fstrip_{cm}~(¬ φ₂)) + 1\\
% \fstrip_{cm} &(¬ (φ₁ ∨ φ₂))  &= \textrm{max}~(\fstrip_{cm}~(¬ φ₁))~(\fstrip_{cm}~(¬ φ₂)) + 1\\
% \fstrip_{cm} &(¬ (φ₁ ⇒ φ₂))  &= \fstrip_{cm}~φ₁ + 1\\
% \fstrip_{cm} &(¬ (¬ φ))      &= 1\\
% \fstrip_{cm} &φ              &= 1
% \end{array}
% \end{aligned}
% \end{equation*}
%
% \section{Another Case in the Proof of the \strip Inference Rule}
% \label{app:strip-proof-case}
%
% \begin{itemize}
% \item[∙] Case $φ ≡ φ₁ ⇒ φ₂$.
% \begin{equation*}
%   \begin{bprooftree}
%   \AxiomC{}
%   \RightLabel{assume}
%   \UnaryInfC{$Γ , φ₁ ⊢ φ₁$}
%   \AxiomC{$Γ ⊢ \fstrip₁~(φ₁ ⇒ φ₂)~(\suc~n)$}
%   \RightLabel{by~\eqref{eq:strip-fixed}}
%   \UnaryInfC{$Γ ⊢ \fuh~(φ₁ ⇒ \fstrip₁~φ₂~n)$}
%   \RightLabel{Lemma~\ref{lem:inv-uh-lem}}
%   \UnaryInfC{$Γ ⊢ φ₁ ⇒ \fstrip₁~φ₂~n$}
%   \RightLabel{weaken}
%   \UnaryInfC{$Γ , φ₁ ⊢ φ₁ ⇒ \fstrip₁~φ₂~n$}
%   \RightLabel{⇒-elim}
%   \BinaryInfC{$Γ , φ₁ ⊢ \fstrip₁~φ₂~n$}
%   \RightLabel{by~ind.~hyp.}
%   \UnaryInfC{$Γ , φ₁ ⊢ φ₂$}
%   \RightLabel{⇒-intro.}
%   \UnaryInfC{$Γ ⊢ φ₁ ⇒ φ₂$}
%   \end{bprooftree}
% \end{equation*}
% \end{itemize}
%
%
% \clearpage
% \section{Bounded Recursion of the Negative Normal Form Function}
% \label{app:polarity-for-propositions}
%
% In Section~\ref{sssec:canonicalize} we discuss a custom negative
% normal form of a formula. To convert a formula to such a normal
% form, we define the function $\fnnf_{0}$ in~\eqref{eq:nnf-zero}.
%
% \begin{equation}
% \label{eq:nnf-zero}
% \begin{aligned}
% &\hspace{.495mm}\fnnf_{0} : \Prop \to \Prop\\
% &\begin{array}{lll}
% \fnnf_{0} &(φ₁ ∧ φ₂)      &= \fcanon_{∧}~(\fassoc_{∧}~(\fnnf_{0}~φ₁ ∧ \fnnf_{0}~φ₂))\\
% \fnnf_{0} &(φ₁ ∨ φ₂)      &= \fcanon_{∨}~(\fassoc_{∨}~(\fnnf_{0}~φ₁ ∨ \fnnf_{0}~φ₂))\\
% \fnnf_{0} &(φ₁ ⇒ φ₂)      &= \fcanon_{∨}~(\fassoc_{∨}~(\fnnf_{0}~((¬ φ₁) ∨ φ₂)))\\
% \fnnf_{0} &(¬ (φ₁ ∧ φ₂))  &= \fcanon_{∨}~(\fassoc_{∨}~(\fnnf_{0}~((¬ φ₁) ∨ (¬ φ₂))))\\
% \fnnf_{0} &(¬ (φ₁ ∨ φ₂))  &= \fcanon_{∧}~(\fassoc_{∧}~(\fnnf_{0}~((¬ φ₁) ∧ (¬ φ₂))))\\
% \fnnf_{0} &(¬ (φ₁ ⇒ φ₂))  &= \fcanon_{∧}~(\fassoc_{∧}~(\fnnf_{0}~((¬ φ₂) ∧ φ₁)))\\
% \fnnf_{0} &(¬ (¬ φ))      &= \fnnf_{0}~φ₁\\
% \fnnf_{0} &(¬ ⊤)          &= ⊥\\
% \fnnf_{0} &(¬ ⊥)          &= ⊤\\
% \fnnf_{0} &φ              &= φ
% \end{array}
% \end{aligned}
% \end{equation}
%
% However, then $\fnnf_{0}$ function is not a structurally recursive
% function. Therefore, we define a bounded recursion in~\eqref{eq:nnf}
% using as the second argument for the bounded recursion its complexity
% measure. The $\fnnf_{cm}$ function in~\eqref{eq:nnf-cm} computes that
% complexity measure.
%
% \begin{equation}
% \label{eq:nnf-cm}
% \begin{aligned}
% &\hspace{.495mm}\fnnf_{cm} : \Prop \to \Nat \to \Prop\\
% &\begin{array}{lll}
% \fnnf_{cm} &(φ₁ ∧ φ₂)      &= \fnnf_{cm}~φ₁ + \fnnf_{cm}~φ₂ + 1 \\
% \fnnf_{cm} &(φ₁ ∨ φ₂)      &= \fnnf_{cm}~φ₁ + \fnnf_{cm}~φ₂ + 1 \\
% \fnnf_{cm} &(φ₁ ⇒ φ₂)      &= 2 \cdot \fnnf_{cm}~φ₁  + \fnnf_{cm}~φ₂ + 1 \\
% \fnnf_{cm} &(¬ (φ₁ ∧ φ₂))  &= \fnnf_{cm}~(¬ φ₁) + \fnnf_{cm}~(¬ φ₂) + 1 \\
% \fnnf_{cm} &(¬ (φ₁ ∨ φ₂))  &= \fnnf_{cm}~(¬ φ₁) + \fnnf_{cm}~(¬ φ₂) + 1 \\
% \fnnf_{cm} &(¬ (φ₁ ⇒ φ₂))  &= \fnnf_{cm}~(¬ φ₁) + 1 \\
% \fnnf_{cm} &(¬ (¬ φ))      &= \fnnf_{cm}~φ₁ + \fnnf_{cm}~(¬ φ₂) + 3 \\
% \fnnf_{cm} &(¬ ⊤)          &= 1 \\
% \fnnf_{cm} &(¬ ⊥)          &= 1 \\
% \fnnf_{cm} &φ              &= 1 \\
% \end{array}
% \end{aligned}
% \end{equation}
%
% Another approach to define the negative normal form in type theory
% without using a complexity measure for the bounded recursion would
% modify the definition of $\fnnf$ defined in~\cite{Bezem2002}. The
% authors avoid the termination problem by using the polarity of the
% formula as an additional argument of its negative normal form
% function. However, be aware the polarity function is not standard and
% \Metis has its own definition.
%
% % \begin{equation}
% % \label{def:polarity}
% %   \begin{aligned}
% %   &\hspace{.495mm}\fpolarity : \Prop \to \abbre{Polarity}\\
% %     &\begin{array}{lll}
% %       \fpolarity &(φ₁ ∧ φ₂) &= ⊕\\
% %       \fpolarity &(φ₁ ∨ φ₂) &= ⊖\\
% %       \fpolarity &(φ₁ ⇒ φ₂) &= ⊖\\
% %       \fpolarity &(¬ φ)     &=
% %         \begin{cases}
% %         ⊕, &\text{ if }\fpolarity~φ=⊖;\\
% %         ⊖, &\text{ if }\fpolarity~φ=⊕;
% %         \end{cases}\\
% %       \fpolarity &φ     &=⊕
% %     \end{array}
% %   \end{aligned}
% % \end{equation}
%
%
% % \section{A Small Example}
% % \label{appendix}
%
% % \subsection{\CPL Problem}
%
% % \begin{equation*}
% % \{\ (p ∨ q) ∧ (p ∨ r)\ \} \vdash p ∨ (q ∧ r)
% % \end{equation*}
%
% % \subsection{\TPTP Problem}
%
% % \begin{verbatim}
% %   fof(premise, axiom, ((p | q) & (p | r))).
% %   fof(goal, conjecture, (p | (q & r))).
% % \end{verbatim}
%
% % \subsection{\Metis \TSTP Derivation}
% % \verbatiminput{sections/data/problem.tstp}
%
% % \subsection{\Agda Proof-term}
% % \verbatiminput{sections/data/problem.agda}
%
%
% % \section{Rank for a proposition}
% % \label{app:rank-for-a-proposition}
% % The rank for a proposition defined in~\cite{VanDalen1994} is
% % a complexity measure.
% % \begin{equation}
% % % \label{eq:rank-definition}
% % \begin{aligned}
% % &\hspace{.495mm}\frank : \Prop → \Nat\\
% % &\begin{array}{llll}
% % \frank~(¬~φ)           &= \frank~φ₁~+~c_{¬}\\
% % \frank~(φ₁~\square~φ₂) &= \fmax~(\frank~φ₁)~(\frank~φ₂) + c_{\square}\\
% % \frank~φ               &= 0.
% % \end{array}
% % \end{aligned}
% % \end{equation}
%
%
% \clearpage
% \section{A Complete Example of Proof-Reconstruction}
% \label{app:complete-example}
%
% \subsection{Installing Athena}
%
% \Athena is the proof-reconstruction tool that accompanying this paper.
% This tool is written in \Haskell and it was tested with
% \prg{GHC}~8.2.1. To install \Athena, the package manager \verb!cabal!
% is required as well. \Athena was tested with \verb!cabal!~1.24.0.
%
% Let us download the \Athena repository running the following command:
%
% \begin{verbatim}
%   $ git clone https://github.com/jonaprieto/athena.git
%   $ cd athena
% \end{verbatim}
%
% To install \Athena run the following command:
%
% \begin{verbatim}
%   $ make install
% \end{verbatim}
%
% To install the \Agda libraries, \verb!agda-prop!, \verb!agda-metis!, and
% the \Agda standard library, run the following command:
%
% \begin{verbatim}
%   $ make install-libraries
% \end{verbatim}
%
% \subsection{Installing Metis}
%
% To install the \Metis prover v2.3 (release 20171021),
% we refer the reader to its official
% repository at \url{https://github.com/gilith/metis}.
%
% As an alternative to install the prover from the \Metis sources,
% we have provided a \Haskell client
% to use this prover but also other provers with \name{Online-ATPs}
% tool\footnote{\url{https://github.com/jonaprieto/online-atps}.}.
% To install this tool run the following command:
%
% \begin{verbatim}
%   $ make online-atps
%   $ online-atps --version
%   Online-atps version 0.1.1
% \end{verbatim}
%
% \subsection{TPTP problem}
%
% Let us consider the theorem in~\eqref{eq:complete-example-problem}
% This problem can be encode in \TPTP syntax (file \verb!problem.tptp!) as follows:
%
% \begin{verbatim}
%   $ cat problem.tptp
%   fof(premise, axiom, (p => q) & (q => p)).
%   fof(goal, conjecture, (p | q) => (p & q)).
% \end{verbatim}
%
% \subsection{Metis derivation}
%
% To obtain the \Metis derivation of the \TPTP problem showed above,
% make sure your \Metis version is supported by running the following
% command. Recall we support the version 2.3 (release 20171021).
%
% \begin{verbatim}
%   $ metis --version
%   metis 2.3 (release 20171021)
% \end{verbatim}
%
% To generate the \TSTP derivation of \verb!problem.tptp!
% run the following command:
%
% \begin{verbatim}
%   $ metis --show proof problem.tptp > problem.tstp
%   $ cat problem.tstp
%   ...
%   fof(premise, axiom, ((p => q) & (q => p))).
%   fof(goal, conjecture, ((p | q) => (p & q))).
%   fof(subgoal_0, plain, ((p | q) => p), inference(strip, [], [goal])).
%   fof(subgoal_1, plain, (((p | q) & p) => q), inference(strip, [], [goal])).
%   fof(negate_0_0, plain, (~ ((p | q) => p)),
%       inference(negate, [], [subgoal_0])).
%   ...
% \end{verbatim}
%
% If we are using the \name{Online-ATPs} tool run the following command:
%
% \begin{verbatim}
%   $ online-atps --atp=metis problem.tptp  > problem.tstp
% \end{verbatim}
%
% Using our customized \TSTP syntax, the above \Metis derivation looks like:
%
% \begin{verbatim}
%   fof(premise, axiom, (p ⇒ q) ∧ (q ⇒ p)).
%   fof(goal, conjecture, (p ∨ q) ⇒ (p ∧ q)).
%   fof(s₀, (p ∨ q) ⇒ p, inf(strip, goal)).
%   fof(s₁, ((p ∨ q) ∧ p) ⇒ q, inf(strip, goal)).
%   fof(neg₀, ¬ ((p ∨ q) ⇒ p), inf(negate, s₀)).
%   fof(n₀₀, (¬ p ∨ q) ∧ (¬ q ∨ p), inf(canonicalize, premise)).
%   fof(n₀₁, ¬ q ∨ p, inf(conjunct, n₀₀)).
%   fof(n₀₂, ¬ p ∧ (p ∨ q), inf(canonicalize, neg₀)).
%   fof(n₀₃, p ∨ q, inf(conjunct, n₀₂)).
%   fof(n₀₄, ¬ p, inf(conjunct, n₀₂)).
%   fof(n₀₅, q, inf(simplify, [n₀₃, n₀₄])).
%   cnf(r₀₀, ¬ q ∨ p, inf(canonicalize, n₀₁)).
%   cnf(r₀₁, q, inf(canonicalize, n₀₅)).
%   cnf(r₀₂, p, inf(resolve, q, [r₀₁, r₀₀])).
%   cnf(r₀₃, ¬ p, inf(canonicalize, n₀₄)).
%   cnf(r₀₄, ⊥, inf(resolve, p, [r₀₂, r₀₃])).
%   fof(neg₁, ¬ ((p ∨ q) ∧ p) ⇒ q), inf(negate, s₁)).
%   fof(n₁₀, ¬ q ∧ p ∧ (p ∨ q), inf(canonicalize, neg₁)).
%   fof(n₁₁, (¬ p ∨ q) ∧ (¬ q ∨ p), inf(canonicalize, premise)).
%   fof(n₁₂, ¬ p ∨ q, inf(conjunct, n₁₁)).
%   fof(n₁₃, ⊥, inf(simplify,[n₁₀, n₁₂])).
%   cnf(r₁₀, ⊥, inf(canonicalize, n₁₃)).
% \end{verbatim}
% % \caption{Customized \Metis \TSTP derivation for the theorem in~\eqref{eq:complete-example-problem}}
% % \label{app:customized-metis-complete-example}
% % \end{figure}
%
% \subsection{Generating the Agda file}
%
% \begin{table}[!ht]
% \caption{\Metis inference rules implemented in \name{agda-metis}.}
%   \begin{center}
%   {\renewcommand{\arraystretch}{1.6}%
%     \begin{tabular}
%       {|@{\hspace{2mm}}l@{\hspace{2mm}}c@{\hspace{2mm}}l@{\hspace{2mm}}|}
%
%     \hline
%     \textbf{\Metis rule} &\textbf{Theorem number}
%      &\textbf{Implementation}
%      \\ \hline
%
%       \texttt{strip}
%       &\ref{thm:strip}
%       &\texttt{strip-thm}
%       \\
%
%       \texttt{conjunct}
%       &\ref{thm:conjunct}
%       &\texttt{conjunct-thm}
%       \\
%
%       \texttt{resolve}
%       &\ref{thm:resolve}
%       &\texttt{resolve-thm}
%       \\
%
%       \texttt{canonicalize}
%       &\ref{thm:canonicalize}
%       &\texttt{canonicalize-thm}
%       \\
%
%       \texttt{clausify}
%       &\ref{thm:clausify}
%       &\texttt{clausify-thm}
%       \\
%
%       \texttt{simplify}
%       &\ref{thm:simplify}
%       &\texttt{simplify-thm}
%       \\[1ex]
%     \hline
%     \end{tabular}}
%   \end{center}
% \end{table}
%
%
% To obtain the \Agda proof-term of the \Metis derivation run
% the following command:
%
% \begin{verbatim}
%   $ athena problem.tstp
% \end{verbatim}
%
% The correspondent \Agda file will be created in the same directory
% that contains \verb!problem.tstp! using the same name but the
% extension of \Agda, that is, \verb!.agda!.
%
%
% \begin{verbatim}
%   $ cat problem.agda
%   ------------------------------------------------------------------------------
%   -- Athena version 0.1-f54e580.
%   -- TSTP file: problem.tstp.
%   ------------------------------------------------------------------------------
%
%   module problem where
%
%   ------------------------------------------------------------------------------
%
%   open import ATP.Metis 2 public
%   open import Data.PropFormula 2 public
%
%   ------------------------------------------------------------------------------
%
%   -- Variables.
%
%   p : PropFormula
%   p = Var (# 0)
%
%   q : PropFormula
%   q = Var (# 1)
%
%   -- Axiom.
%
%   a₁ : PropFormula
%   a₁ = ((p ⊃ q) ∧ (q ⊃ p))
%
%   -- Premise.
%
%   Γ : Ctxt
%   Γ = [ a₁ ]
%
%   -- Conjecture.
%
%   goal : PropFormula
%   goal = ((p ∨ q) ⊃ (p ∧ q))
%
%   -- Subgoals.
%
%   subgoal₀ : PropFormula
%   subgoal₀ = ((p ∨ q) ⊃ p)
%
%   subgoal₁ : PropFormula
%   subgoal₁ = (((p ∨ q) ∧ p) ⊃ q)
%
%   ------------------------------------------------------------------------------
%   -- Proof of subgoal₀.
%   ------------------------------------------------------------------------------
%
%   proof₀ : Γ ⊢ subgoal₀
%   proof₀ =
%     (RAA
%       (resolve-thm ⊥ p
%         (resolve-thm p q
%           (simplify-thm q
%             (conjunct-thm (p ∨ q)
%               (canonicalize-thm ((¬ p) ∧ (p ∨ q))
%                 (assume {Γ = Γ} (¬ subgoal₀))))
%             (conjunct-thm (¬ p)
%               (canonicalize-thm ((¬ p) ∧ (p ∨ q))
%                 (assume {Γ = Γ} (¬ subgoal₀)))))
%           (conjunct-thm ((¬ q) ∨ p)
%             (canonicalize-thm (((¬ p) ∨ q) ∧ ((¬ q) ∨ p))
%               (weaken (¬ subgoal₀)
%                 (assume {Γ = ∅} a₁)))))
%         (conjunct-thm (¬ p)
%           (canonicalize-thm ((¬ p) ∧ (p ∨ q))
%             (assume {Γ = Γ} (¬ subgoal₀))))))
%
%   ------------------------------------------------------------------------------
%   -- Proof of subgoal₁.
%   ------------------------------------------------------------------------------
%
%   proof₁ : Γ ⊢ subgoal₁
%   proof₁ =
%     (RAA
%       (simplify-thm ⊥
%         (canonicalize-thm ((¬ q) ∧ (p ∧ (p ∨ q)))
%           (assume {Γ = Γ} (¬ subgoal₁)))
%         (conjunct-thm ((¬ p) ∨ q)
%           (canonicalize-thm (((¬ p) ∨ q) ∧ ((¬ q) ∨ p))
%             (weaken (¬ subgoal₁)
%               (assume {Γ = ∅} a₁))))))
%
%   ------------------------------------------------------------------------------
%   -- Proof of the goal.
%   ------------------------------------------------------------------------------
%
%   proof : Γ ⊢ goal
%   proof =
%     ⊃-elim
%       strip-thm
%       (∧-intro proof₀ proof₁)
%   \end{verbatim}
%
% Now, we are ready to verify the \Metis derivation by type-checking with
% \Agda the reconstructed proof showed above. Make sure the \Agda version is
%  2.5.3.
%
% \begin{verbatim}
%   $ agda --version
%   Agda version 2.5.3
%   $ agda problem.agda
% \end{verbatim}
%
% As we can see in the \Agda code showed above, the term \verb!proof!,
% the proof-term of the \Metis derivation is referring to the
% proof-terms \verb!proof₀! and \verb!proof₁!. Recall, \Metis stripes
% the goal into subgoals to prove it. Therefore, these terms are the
% proof-terms for the refutations of the subgoals s₀ and s₁. We show
% in the following sections the respective natural deduction trees for
% these refutations.
%
% \subsection{First refutation tree}
%
% The \TSTP derivation that corresponds with the refutation
% tree of the subgoal $s₀$ is the following:
%
% \begin{verbatim}
%   fof(premise, axiom, (p ⇒ q) ∧ (q ⇒ p)).
%   fof(goal, conjecture, (p ∨ q) ⇒ (p ∧ q)).
%   fof(s₀, (p ∨ q) ⇒ p, inf(strip, goal)).
%   ...
%   fof(neg₀, ¬ ((p ∨ q) ⇒ p), inf(negate, s₀)).
%   fof(n₀₀, (¬ p ∨ q) ∧ (¬ q ∨ p), inf(canonicalize, premise)).
%   fof(n₀₁, ¬ q ∨ p, inf(conjunct, n₀₀)).
%   fof(n₀₂, ¬ p ∧ (p ∨ q), inf(canonicalize, neg₀)).
%   fof(n₀₃, p ∨ q, inf(conjunct, n₀₂)).
%   fof(n₀₄, ¬ p, inf(conjunct, n₀₂)).
%   fof(n₀₅, q, inf(simplify,[n₀₃, n₀₄])).
%   cnf(r₀₀, ¬ q ∨ p, inf(canonicalize, n₀₁)).
%   cnf(r₀₁, q, inf(canonicalize, n₀₅)).
%   cnf(r₀₂, p, inf(resolve, q, [r₀₁, r₀₀])).
%   cnf(r₀₃, ¬ p, inf(canonicalize, n₀₄)).
%   cnf(r₀₄, ⊥, inf(resolve, p, [r₀₂, r₀₃])).
%   ...
% \end{verbatim}
%
% The refutation tree is the following:
%
% \begin{center}
% \begin{scprooftree}{1}
% \AxiomC{$\mathcal{D}_1$}
% %\AxiomC{$\mathcal{D}_3$}
% \AxiomC{}
% \RightLabel{assume $\neg s_0$}
% \UnaryInfC{$Γ, \neg s_0 ⊢ \neg s_0$}
% % \UnaryInfC{$Γ, \neg s_0 ⊢ (p ∨ q) ⇒ p$}
% %\LeftLabel{$(\mathcal{D}_3)$\hspace{1.5cm}}
% \RightLabel{Theorem~\ref{thm:canonicalize}}
% \UnaryInfC{$Γ, \neg s_0 ⊢ \neg p ∧ (p ∨ q)$}
% \RightLabel{Theorem~\ref{thm:conjunct}}
% \UnaryInfC{$Γ, \neg s_0 ⊢ \neg p$}
% \RightLabel{Theorem~\ref{thm:resolve}~with $ℓ = p$}
% \LeftLabel{$(\mathcal{R}_{1})$\hspace{2mm}}
% \BinaryInfC{$Γ, \neg s_0 ⊢ \bot$}
% \RightLabel{RAA.}
% \UnaryInfC{$Γ ⊢ s_0$}
% \end{scprooftree}
% \end{center}
% \medskip
%
% \begin{equation*}
% \begin{bprooftree}
% \AxiomC{$\mathcal{D}_2$}
% \UnaryInfC{$Γ, \neg s_0 ⊢ \neg q ∨ p$}
%
% \AxiomC{$\mathcal{D}_3$}
% \UnaryInfC{$Γ, \neg s_0 ⊢ p ∨ q$}
% %
% \AxiomC{$\mathcal{D}_4$}
% \UnaryInfC{$Γ, \neg s_0 ⊢ \neg p$}
% %
% %\LeftLabel{$(\mathcal{D}_4)$\hspace{1.5cm}}
% \RightLabel{Theorem~\ref{thm:simplify}}
% \BinaryInfC{$Γ, \neg s_0 ⊢ q$}
% \LeftLabel{$(\mathcal{D}_1)$\hspace{2mm}}
% \RightLabel{Theorem~\ref{thm:resolve}~with $ℓ = q$}
% \BinaryInfC{$Γ, \neg s_0 ⊢ p$}
% \end{bprooftree}
% \end{equation*}
%
% \begin{equation*}
% \begin{bprooftree}
%   \AxiomC{}
%   \RightLabel{axiom premise}
%   \UnaryInfC{$Γ ⊢ (p ⇒ q) ∧ (q ⇒ p)$}
%   \RightLabel{weaken}
%   \UnaryInfC{$Γ, \neg s_0 ⊢ (p ⇒ q) ∧ (q ⇒ p)$}
%   \LeftLabel{$(\mathcal{D}_2)$\hspace{2mm}}
%   \RightLabel{Theorem~\ref{thm:canonicalize}}
%   \UnaryInfC{$Γ, \neg s_0 ⊢ (\neg p ∨ q) ∧ (\neg q ∨ p)$}
%   \RightLabel{Theorem~\ref{thm:conjunct}}
%   \UnaryInfC{$Γ, \neg s_0 ⊢ \neg q ∨ p$}
% \end{bprooftree}
% \end{equation*}
% \medskip
% \begin{equation*}
% \begin{bprooftree}
% \AxiomC{}
% \RightLabel{assume}
% \UnaryInfC{$Γ, \neg s_0 ⊢ \neg s_0$}
% \LeftLabel{$(\mathcal{D}_3)$\hspace{2mm}}
% \RightLabel{Theorem~\ref{thm:canonicalize}}
% \UnaryInfC{$Γ, \neg s_0 ⊢ \neg p ∧ (p ∨ q)$}
% \RightLabel{Theorem~\ref{thm:conjunct}}
% \UnaryInfC{$Γ, \neg s_0 ⊢ p ∨ q$}
% \end{bprooftree}
% \end{equation*}
% \medskip
% \begin{equation*}
% \begin{bprooftree}
% \AxiomC{}
% \RightLabel{assume $\neg s_0$}
% \UnaryInfC{$Γ, \neg s_0 ⊢ \neg s_0$}
% % \UnaryInfC{$Γ, \neg s_0 ⊢ (p ∨ q) ⇒ p$}
% \LeftLabel{$(\mathcal{D}_4)$\hspace{2mm}}
% \RightLabel{Theorem~\ref{thm:canonicalize}}
% \UnaryInfC{$Γ, \neg s_0 ⊢ \neg p ∧ (p ∨ q)$}
% \RightLabel{Theorem~\ref{thm:conjunct}}
% \UnaryInfC{$Γ, \neg s_0 ⊢ \neg p$}
% \end{bprooftree}
% \end{equation*}
%
% \subsection{Second refutation tree}
%
% The \TSTP derivation that corresponds with the refutation
% tree of the subgoal $s₁$ is the following:
%
% \begin{verbatim}
%   fof(premise, axiom, (p ⇒ q) ∧ (q ⇒ p)).
%   ...
%   fof(s₁, ((p ∨ q) ∧ p) ⇒ q, inf(strip, goal)).
%   ...
%   fof(neg₁, ¬ (((p ∨ q) ∧ p) ⇒ q), inf(negate, s₁)).
%   fof(n₁₀, ¬ q ∧ p ∧ (p ∨ q), inf(canonicalize, neg₁)).
%   fof(n₁₁, (¬ p ∨ q) ∧ (¬ q ∨ p), inf(canonicalize, premise)).
%   fof(n₁₂, ¬ p ∨ q, inf(conjunct, n₁₁)).
%   fof(n₁₃, ⊥, inf(simplify,[n₁₀, n₁₂])).
%   cnf(r₁₀, ⊥, inf(canonicalize, n₁₃)).
% \end{verbatim}
%
% The refutation tree is the following:
%
% \begin{center}
% \begin{scprooftree}{0.95}
% \AxiomC{}
% \RightLabel{assume ($\neg s₁)$}
% \UnaryInfC{$Γ,\neg s₁ ⊢ \neg s₁$}
% \RightLabel{Theorem~\ref{thm:canonicalize}}
% \UnaryInfC{$Γ, \neg s₁⊢ \neg q ∧ p ∧ (p ∨ q)$}
% \AxiomC{}
% \RightLabel{axiom premise}
% \UnaryInfC{$Γ ⊢ (p ⇒ q) ∧ (q ⇒ p)$}
% \RightLabel{weaken}
% \UnaryInfC{$Γ, \neg s₁ ⊢ (p ⇒ q) ∧ (q ⇒ p)$}
% \RightLabel{Theorem~\ref{thm:canonicalize}}
% \UnaryInfC{$Γ, \neg s₁ ⊢ (\neg p ∨ q) ∧ (\neg q ∨ p)$}
% \RightLabel{Theorem~\ref{thm:conjunct}}
% \UnaryInfC{$Γ, \neg s₁ ⊢ \neg p ∨ q$}
% \LeftLabel{$(\mathcal{R}_{2})$\hspace{1mm}}
% \RightLabel{Theorem~\ref{thm:simplify}}
% \BinaryInfC{$Γ, \neg s₁ ⊢ \bot$}
% \RightLabel{RAA.}
% \UnaryInfC{$Γ ⊢ s₁$}
% \end{scprooftree}
% \end{center}
%
% \subsection{The proof of the goal}
%
% \begin{center}
% \begin{scprooftree}{1}
% \AxiomC{}
% \RightLabel{Theorem~\ref{thm:strip}}
% \UnaryInfC{$Γ ⊢ (s_0 ∧ s₁) ⇒ \text{goal}$}
% \AxiomC{$\mathcal{R}_{1}$}
% \UnaryInfC{$Γ ⊢ s_0$}
% \AxiomC{$\mathcal{R}_2$}
% \UnaryInfC{$Γ ⊢ s₁$}
% \RightLabel{$\wedge$-intro}
% \BinaryInfC{$Γ ⊢ s_0 ∧ s₁$}
% \RightLabel{$⇒$-elim}
% \BinaryInfC{$Γ ⊢ \text{goal}$}
% \end{scprooftree}
% \end{center}
%
\end{subappendices}

\end{document}
