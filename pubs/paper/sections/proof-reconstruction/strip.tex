\documentclass[../../main.tex]{subfiles}
\begin{document}

\subsubsection{Strip.}
\label{sssec:strip-a-goal}

To prove a goal, \Metis splits the goal into disjoint cases. This process
produces a list of new subgoals, the conjunction of these subgoals implies the
goal. Then, a proof of the goal becomes into a set of smaller proofs, one
refutation for each subgoal.

\begin{myexamplenum}
In Fig.~\ref{complete-metis-example}, we can see how
the subgoals associated to a goal are introduced
in the \TSTP derivation line \ref{line3} with the \strip inference rule.
The conjecture in line \ref{line2}, $(p ∨ q) ⇒ (p ∧ q)$,
is stripped into two subgoals (line~\ref{line4} and \ref{line5}):
$(p ∨ q) ⇒ p$ and $((p ∨ q) ∧ p) ⇒ q$.
\Metis proves each subgoal separately following the same order as
they appeared in the \TSTP derivation.
\end{myexamplenum}

\begin{remark}
\Metis does not make explicit in the \TSTP derivation the way it uses the
subgoals to prove the goal. We show this approach in the correctness of the
\strip inference rule in Theorem~\ref{thm:strip}. To show this theorem, we prove
Lemma~\ref{lem:inv-uh-lem} and  Lemma~\ref{lem:lem-inv-strip}. In the former, we
introduce the auxiliary function $\fuh_1$ to define the $\fuh$ function used in
in the latter in the definition of the $\fstrip$ function.
\end{remark}

\begin{notation}
$\Bound$ is a synonym for $\Nat$ type to refer to a natural number that constrains
the number of recursive calls done by a function. We use this type mostly in
the type of the function.
\end{notation}

\begin{mainlemma}
  \label{lem:inv-uh-lem}
Let be $n : \Bound$. If $Γ ⊢ \fuh₁~φ~n$ then $Γ ⊢ φ$ where

\begin{empheq}[box=\fcolorbox{bocolor}{bgcolor}]{equation}
  \label{eq:uh-structured}
  \begin{aligned}
  &\hspace{.495mm} \fuh_{1} : \Prop → \Bound → \Prop\\
  &\begin{array}{llll}
  \fuh_{1} &(φ₁ ⇒ (φ₂ ⇒ φ₃)) &(\suc~n) &= \fuh_{1}~((φ₁ ∧ φ₂) ⇒ φ₃)~n\\
  \fuh_{1} &(φ₁ ⇒ (φ₂ ∧ φ₃)) &(\suc~n) &= \fuh_{1}~(φ₁ ⇒ φ₂)~n ∧ \fuh_{1}~(φ₁ ⇒ φ₃)~n\\
  \fuh_{1} &φ &n &= φ.
  \end{array}
  \end{aligned}
\end{empheq}
\end{mainlemma}

To provide such a number given a formula
$\varphi$, we call the $\fuh_{cm}$ function defined in~\eqref{eq:uh-complexity}.
For the convenience in the following descriptions, we have the
function $\fuh : \Prop → \Prop$ for $\varphi \mapsto \fuh_{1}~φ~(\fuh_{cm}~φ)$.

\begin{empheq}[box=\fcolorbox{bocolor}{bgcolor}]{equation}
  \label{eq:uh-complexity}
  \begin{aligned}
    &\hspace{.495mm}\fuh_{cm} : \Prop → \Bound\\
    &\begin{array}{llll}
    \fuh_{cm}~(φ₁ ⇒ (φ₂ ⇒ φ₃)) &= \fuh_{cm}~φ₃ + 2\\
    \fuh_{cm}~(φ₁ ⇒ (φ₂ ∧ φ₃)) &= \fmax~(\fuh_{cm}~φ₂)~(\fuh_{cm}~φ₃) + 1\\
    \fuh_{cm}~φ                &= 0.
    \end{array}
  \end{aligned}
\end{empheq}

Now, we define the $\fstrip$ function in \eqref{eq:strip} in a similar way as we
did above for the $\fuh$ function. We have a definition of the $\fstrip₁$ function
in~\eqref{eq:strip-fixed} and a function to find its bound for its recursive calls
by $\fstrip_{cm}$ defined in~\cite{Prieto-Cubides2017a}.
% \vspace*{-5mm}
\begin{empheq}[box=\fcolorbox{bocolor}{bgcolor}]{align}
  \label{eq:strip}
  \begin{split}
  &\fstrip : \Prop → \Prop\\
  &\fstrip~φ~ = \fstrip₁~φ~(\fstrip_{cm}~φ).
  \end{split}
\end{empheq}

The $\fstrip$ function yields the conjunction of subgoals that implies the goal
of the problem in the \Metis \TSTP derivations. Its definition comes mainly from
the reading of the \Metis source code and some examples from \TSTP derivations.

\begin{empheq}[box=\fcolorbox{bocolor}{bgcolor}]{equation}
  \label{eq:strip-fixed}
  \begin{aligned}
  &\hspace{.495mm}\fstrip₁ : \Prop → \Bound → \Prop\\
  &\begin{array}{llll}
  \fstrip₁ &(φ₁ ∧ φ₂)     &(\suc~n) &= \fuh~(\fstrip₁~φ₁~n) ∧ \fuh~(φ₁ ⇒ \fstrip₁~φ₂~n)\\
  \fstrip₁ &(φ₁ ∨ φ₂)     &(\suc~n) &= \fuh~((¬ φ₁) ⇒ \fstrip₁~φ₂~n)\\
  \fstrip₁ &(φ₁ ⇒ φ₂)     &(\suc~n) &= \fuh~(φ₁ ⇒ \fstrip₁~φ₂~n)\\
  \fstrip₁ &(¬ (φ₁ ∧ φ₂)) &(\suc~n) &= \fuh~(φ₁ ⇒ \fstrip₁~(¬ φ₂)~n)\\
  \fstrip₁ &(¬ (φ₁ ∨ φ₂)) &(\suc~n) &= \fuh~(\fstrip₁~(¬ φ₁)~n) ∧ \fuh~((¬ φ₁) ⇒ \fstrip₁~(¬ φ₂)~n)\\
  \fstrip₁ &(¬ (φ₁ ⇒ φ₂)) &(\suc~n) &= \fuh~(\fstrip₁~φ₁~n) ∧ \fuh~(φ₁ ⇒ \fstrip₁~(¬ φ₂)~n)\\
  \fstrip₁ &(¬ (¬ φ₁))    &(\suc~n) &= \fuh~(\fstrip₁~φ₁~n)\\
  \fstrip₁ &(¬ ⊥)         &(\suc~n) &= ⊤\\
  \fstrip₁ &(¬ ⊤)         &(\suc~n) &= ⊥\\
  \fstrip₁ &φ             &n        &= φ.
  \end{array}
  \end{aligned}
\end{empheq}

\begin{mainlemma}
\label{lem:lem-inv-strip}
Let $n : \Bound$. If $Γ~⊢~\fstrip₁~φ~n$ then $Γ~⊢~φ$.
\end{mainlemma}

The following lemma is convenient to substitute in a theorem equals by equals in
the conclusion of the sequent. Recall the equality (≡) is symmetric and
transitive, and we use these properties without any mention.

\begin{mainlemma}[\fsubst]
  \label{lem:subst}
  If $Γ ⊢ φ$ and $ψ ≡ φ$ then $Γ ⊢ ψ$.
\end{mainlemma}

We can now formulate the result that justifies the stripping strategy of \Metis
to prove goals. For the sake of brevity, we state the following theorem for the
\strip function when the goal has two subgoals. In other cases, we extend the
theorem in the natural way.

\begin{mainth}
\label{thm:strip}
Let $s_1$ and $s_2$ be the subgoals of the goal $φ$, that is,
$\fstrip~\varphi≡~s_{1}∧~s_{2}$.
% $\fstrip₁~φ~n~≡~s₂∧~s₃$ where $\text{n} : \Nat$ is the complexity measure $\fstrip_{cm}$.
If $Γ ⊢ s_{1}$ and $Γ ⊢ s_{2}$ then $Γ ⊢ φ$.
\end{mainth}

% % No creo que sea necesario presentar las demostraciones
% \begin{proof}
% \begin{equation*}
%   \begin{bprooftree}
%   \AxiomC{$ Γ ⊢ s_1 $}
%   \AxiomC{$ Γ ⊢ s_2 $}
%   \RightLabel{∧-intro}
%   \BinaryInfC{$Γ ⊢ s_1\wedge s_2$}
%   \AxiomC{$\fstrip~φ ≡ s_1\wedge s_2 $}
%   \RightLabel{By def.~\eqref{eq:strip}}
%   \UnaryInfC{$\fstrip₁~φ~(\fstrip_{cm}~\varphi) ≡ s_1\wedge s_2 $}
%   \RightLabel{\fsubst}
%   \BinaryInfC{$Γ ⊢ \fstrip₁~φ~n$}
%   \RightLabel{Lemma~\ref{lem:lem-inv-strip}}
%   \UnaryInfC{$Γ ⊢ φ$}
% \end{bprooftree}
% \end{equation*}
% \end{proof}

% -------------------------------------------------------------------

\begin{remark}

The user of \Metis will probably note another rule in the derivations related
with these subgoals, the \negate rule. Since \Metis proves a conjecture by
refutation, to prove a subgoal, \Metis assumes its negation using the \negate
rule. We will find this rule always after the \strip rule per each subgoal in the
\TSTP derivation.

\end{remark}

\end{document}
