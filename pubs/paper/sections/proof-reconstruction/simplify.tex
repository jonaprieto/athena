\documentclass[../../main.tex]{subfiles}
\begin{document}

\subsubsection{Simplify.}
\label{sssec:simplify}

The \simplify rule is an inference that performs simplification of tautologies
in a list of derivations. This rule simplifies such a list by transversing the
list while applying Lemma~\ref{lem:simplify-or}, Lemma~\ref{lem:simplify-and},
among others theorems presented in the following description. The goal of this
rule despites simplifying the list of derivations to a new formula (often
smaller than its input formulas), consists of finding a contradiction.

We observe based on the analysis of different cases in the \TSTP
derivations that \simplify can be modeled by a function with three
arguments: two source formulas and the target formula.

Since the main purpose of the \simplify rule is to simplify formulas to get
$⊥$, we have defined the $\freduce$ function to help removing the negation
of a given literal $ℓ$ from a input formula.

\begin{empheq}[box=\fcolorbox{bocolor}{bgcolor}]{equation}
\label{def:remove-literal}
  \begin{aligned}
  &\hspace{.495mm}\freduce : \Prop → \Lit → \Prop\hspace*{3.5cm}\\
  &\begin{array}{lll}
\freduce &(φ₁~∧~φ₂) &ℓ~= \fsimplify_{∧}~(\freduce~φ₁~ℓ~∧~\freduce~φ₂~ℓ)\\
\freduce &(φ₁~∨~φ₂) &ℓ~= \fsimplify_{∨}~(\freduce~φ₁~ℓ~∨~\freduce~φ₂~ℓ)\\[2mm]
\freduce &φ &ℓ~=\begin{cases}
  ⊥,  &\text{ if }φ\text{ is a literal and }ℓ ≡ \fnnf(¬ φ);\\
  φ,  &\text{ otherwise.}
  \end{cases}
   \end{array}
  \end{aligned}
\end{empheq}

\begin{mainlemma}
\label{lem:reduce-literal}
Let $ℓ$ be a literal and $φ : \Prop$. If $Γ ⊢ φ$ and $Γ ⊢ ℓ$ then
$Γ~⊢~\freduce~φ~ℓ$.
\end{mainlemma}

The \fsimplify function is defined in~\eqref{eq:simplify}. If some
input formula is equal to the target formula, we derive that formula.
If any formula is $⊥$, we derive the target formula by using the
⊥-elim inference rule.
Otherwise, we proceed by cases on the structure of the formula.
For example, when the second source formula is a literal, we
use the $\freduce$ function and Lemma~\ref{lem:reduce-literal}.

\begin{empheq}[box=\fcolorbox{bocolor}{bgcolor}]{equation}
\label{eq:simplify}
  \begin{aligned}
  &\hspace{.495mm}\fsimplify : \Source → \Source → \Target → \Prop \hspace*{3.5cm}\\
  &\begin{array}{l}
  \fsimplify~φ₁~φ₂~ψ~=\\
  \hspace{3mm}\begin{cases}
  ψ, &\text{ if }φᵢ ≡ ⊥\text{ for some }i = 1, 2;\\
  ψ, &\text{ if }φᵢ ≡ ψ\text{ for some }i = 1, 2;\\
  \fsimplify_{∧}~(\fsimplify~φ₁~φ₂₁~ψ)~φ₂₂~ψ,
  &\text{ if }φ₂ ≡ φ₂₁ ∧ φ₂₂;\\
   \fsimplify_{∨}~(\fsimplify~φ₁~φ₂₁~ψ~∨~\fsimplify~φ₁~φ₂₂~ψ)
  &\text{ if }φ₂ ≡ φ₂₁ ∨ φ₂₂;\\
  \freduce~φ₁~φ₂, &\text{ if }φ₂\text{ is a literal};\\
  φ₁,  &\text{ otherwise.}
  \end{cases}
  \end{array}
  \end{aligned}
\end{empheq}

\begin{mainlemma}
  \label{lem:binary-simplify}
  Let $ψ : \Target$. If $Γ ⊢ φ₁$ and $Γ ⊢ φ₂$ then
  $Γ ⊢ \fsimplify~φ₁~φ₂~ψ$.
\end{mainlemma}

\begin{mainth}
  \label{thm:simplify}
  Let $ψ : \Target$. Let $φᵢ : \Source$ such that $Γ ⊢ φᵢ$ for
  $i = 1, \cdots, n$ and $n \geq 2$.
  Then $Γ ⊢ \fsimplify~γ_{n-1}~φ_{n}~ψ$ where
  $γ₁ = φ₁$, and $γᵢ ≡ \fsimplify~γ_{i-1}~φᵢ~ψ$.
\end{mainth}

\begin{myremark}

Besides the fact that $\List~\Prop~\to~\Prop$ is the type that mostly
fit with the \simplify rule, we choose a different option. In the
translation from \TSTP to \Agda, we take the list of derivations and
we apply the rule by using a left folding (the \verb!foldl! function
in functional programming) with the $\fsimplify$ function over the
list of $φ₁, φ₂, \cdots, φₙ$. This design decision avoids us to define a new theorem
type to support \List~\Prop type in the premise side.
\end{myremark}

\begin{myexamplenum}
Let us review the following \TSTP excerpt where \simplify was used twice.

\begin{verbatim}
  fof(n₀, (¬ p ∨ q) ∧ ¬ r ∧ ¬ q ∧ (p ∨ (¬ s ∨ r)), ...
  fof(n₁, p ∨ (¬ s ∨ r), inf(conjunct, n₀)).
  fof(n₂, ¬ p ∨ q, inf(conjunct, n₀)).
  fof(n₃, ¬ q, inf(conjunct, n₀)).
  fof(n₄, ¬ p, inf(simplify, [n₂, n₃])).
  fof(n₅, ¬ r, inf(conjunct, n₀)).
  fof(n₆, ⊥, inf(simplify, [n₁, n₄, n₅])).
\end{verbatim}

\begin{enumerate}
\item The \simplify rule derives $¬ p$ in \verb!n₄!
from \verb!n₂! and \verb!n₃! derivations.

$$\fsimplify~(¬ p ∨ q)~(¬ q)~(¬ p) = ¬ p.$$
% \begin{verbatim}
% fof(norm2, ¬ p ∨ q, inf(conjunct, [norm0])).
% fof(norm3, ¬ q, inf(conjunct, [norm0])).
% fof(norm4, ¬ p, inf(simplify, [norm2, norm3])).
% \end{verbatim}
\item We use Theorem~\ref{thm:simplify} to derive ⊥ in \verb!n₆!, the
following is a proof.

\begin{equation*}
\begin{bprooftree}
\AxiomC{$Γ ⊢ p ∨ (¬ s ∧ r)$}
\AxiomC{$Γ ⊢ ¬ p$}
\RightLabel{Theorem~\ref{thm:simplify}}
\BinaryInfC{$Γ ⊢ ¬ s ∧ r$}
\AxiomC{$Γ ⊢ ¬ r$}
\RightLabel{Theorem~\ref{thm:simplify}}
\BinaryInfC{$Γ ⊢ ⊥$}
\end{bprooftree}
\end{equation*}

% \begin{verbatim}
% fof(n1, g ∨ (¬ s ∧ r), inf(conjunct, [n0])).
% fof(n4, ¬ p, inf(simplify, [n2, n3])).
% fof(n5, ¬ r, inf(conjunct, [n0])).
% fof(n6, ⊥, inf(simplify, [n1, n4, n5])).
% \end{verbatim}
\end{enumerate}
\end{myexamplenum}


% subsubsection simplify (end)
% -------------------------------------------------------------------

\end{document}
