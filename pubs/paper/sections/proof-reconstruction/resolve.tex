\documentclass[../../main.tex]{subfiles}
\begin{document}

% ----------------------------------------------------------------------------
% -- RESOLVE --
% ----------------------------------------------------------------------------

\subsubsection{Resolve.}
\label{sssec:resolve}

The \resolve inference rule in combination with the \simplify rule are the
responsable rules to perform resolution, in particular, the \resolve rule is is
the version of the resolution theorem showed in~Fig.~\ref{fig:metis-inferences}.

\begin{example} In the \TSTP derivation in
Fig.~\ref{eq:complete-example-problem}, there are two ocurrences in
line~\ref{line14} and line~\ref{line16}~of the \resolve rule. This rule has two
inputs in the following order: the \emph{resolvent} which is the literal denoted
by ℓ in~Fig.~\ref{fig:metis-inferences} and a list formed by two formulas to
resolve. \end{example}

In~\cite{Prieto-Cubides2017a}, we describe our first attempt to reconstruct this
rule that performs reordering of formulas jointly with the application of a
customized version of the resolution theorem to reconstruct the rule.
Unfortunately, that approach suffers of an exponential computational cost for
each searching to reorder the formulas. Therefore, we propose
in~\eqref{eq:resolve} an alternative formulation to reconstruct this rule by
using the \fsimplify rule. This observation was taken from the \TSTP derivations
and we check that improves significantly the timing of the type-checking.

\begin{mainth}
  \label{thm:resolve}
  Let $ℓ$ be a literal, $ℓ : \Lit$, and $ψ : \Target$. If $Γ ⊢ φ₁$ and
  $Γ~⊢~φ₂$ then $Γ~⊢~\fresolve~φ₁~φ₂~ℓ~ψ$, where
  \begin{empheq}[box=\fcolorbox{bocolor}{bgcolor}]{equation}
  \begin{split}
  \label{eq:resolve}
    &\fresolve : \Source \to \Source \to \Lit \to \Target \to \Prop\\
    &\fresolve~φ₁~φ₂~ℓ~ψ~=~\fsimplify~(φ₁ ∧ φ₂)~(¬ ℓ ∨ ℓ)~ψ.
  \end{split}
  \end{empheq}
\end{mainth}

%% maybe we want to show this easy proof.
% \begin{proof}
%   \begin{equation*}
%   \begin{bprooftree}
%     \AxiomC{$Γ ⊢ φ₁$}
%     \AxiomC{$Γ ⊢ φ₂$}
%     \RightLabel{∧-intro}
%     \BinaryInfC{$Γ ⊢~\varphi_1~\wedge~\varphi_2$}
%     \AxiomC{\abbre{PEM}~$\ell$}
%     \RightLabel{Theorem~\ref{thm:simplify}.}
%     \BinaryInfC{$\Gamma\vdash~\fsimplify~~(φ₁ ∧ φ₂)~(¬ ℓ ∨ ℓ)~ψ$}
%   \end{bprooftree}
%   \end{equation*}
% \end{proof}

\end{document}
