\documentclass[../../main.tex]{subfiles}
\begin{document}


\subsubsection{Clausify.}
\label{sssec:clausification}

The \clausify rule is a rule that transforms a formula into its clausal normal
form but without performing simplifications of tautologies. Recall, this kind of
conversion was already addressed by the \canonicalize rule. It is important to
notice that this kind of conversions between one formula to its clausal normal
form are not unique, and \Metis has customized approaches to perform such
transformations.

To address the reconstruction of this rule, we perform a
reordering of the conjunctive normal form given by the $\fcnf$ function defined
in Lemma~\ref{lem:cnf} with the aforementioned function  $\freorder_{∧∨}$ to the
input formula of the rule.


\begin{myexamplenum} In the following \TSTP derivation by \Metis, the \clausify
rule transforms the \texttt{n₀} formula to get the \texttt{n₁} formula, that is
just the application of the distribution laws.

\begin{verbatim}
  fof(n₀, ¬ p ∨ (q ∧ r) ...
  fof(n₁, (¬ p ∨ q) ∧ (¬ p ∨ r), inf(clausify, n₀)).
\end{verbatim}

\end{myexamplenum}

\begin{mainth}
\label{thm:clausify}
   Let $ψ : \Target$. If $Γ ⊢ φ$ then $Γ ⊢ \fclausify~φ~ψ$, where
  \begin{empheq}[box=\fcolorbox{bocolor}{bgcolor}]{equation*}
  \begin{aligned}
  &\hspace{.495mm}\fclausify : \Source → \Target → \Prop\\
  &\begin{array}{llll}
  \fclausify~φ~ψ &=
         \begin{cases}
        ψ, &\text{ if }φ≡ψ;\\
        \freorder_{∧∨}~(\fcnf~φ)~ψ, &\text{ otherwise.}
      \end{cases}
  \end{array}
  \end{aligned}
  \end{empheq}
\end{mainth}

\end{document}
