\documentclass[../../main.tex]{subfiles}
\begin{document}

% ----------------------------------------------------------------------------
% -- Preliminars --
% ----------------------------------------------------------------------------

\begin{mydefinition}

A \emph{literal} is a propositional variable (positive literal) or a
negation of a propositional variable (negative literal).

\end{mydefinition}

\begin{notation}
\Lit is a synonym of \Prop to refer to literals.
\end{notation}

\begin{mydefinition}

The \emph{negative normal form} of a formula is the logical
equivalent version of it in which negations appear only in the
literals and the formula does not contain any implications.

\end{mydefinition}

\begin{mydefinition}

The \emph{conjunctive normal form} of a formula also called clausal
normal form is the logical equivalent version expressed as a
conjunction of clauses where a \emph{clause} is the disjunction of
zero or more literals.
\end{mydefinition}


\begin{remark} The proof-reconstruction we are exposing needs a mechanism to
test logic equivalence between propositions in some cases. This task is not
trivial since we left out the semantics to treat only the syntax aspects of the
propositional formulas, and therefore we have to manipulate in somehow the
formulas to see if they are logically equivalent.

We address this challenge by converting the formulas to their conjunctive
normal. We check if their normal forms are equal and if they are not, we try to
reorder the first formula to match with the second one to fulfill the
syntactical equality. In~\cite{Prieto-Cubides2017a}, we make a description of a
set of functions and lemmas for such a purpose. For instance, we define
functions of ($\Source \to  \Target \to \Prop$) type following the pattern in
Example~\ref{ex:inference-rule-pattern} like the $\freorder_{∨}$ function. This
function try to fix the order of a \emph{premise} disjunction given by following
the order about the \emph{conclusion} disjunction, the target. Similarly, we
define the function $\freorder_{\wedge}$ for conjunctions and the function
$\freorder_{\wedge\vee}$ for conjunctive normal forms.

\end{remark}

% ----------------------------------------------------------------------------
% -- CANONICALIZE --
% ----------------------------------------------------------------------------

\subsubsection{Canonicalize.}
\label{sssec:canonicalize}

The \canonicalize rule is an inference that transforms a formula to
its negative normal form or its conjunctive normal form depending on
the role that the formula plays in the problem.
This rule jointly with the \clausify rule perform the so-called
\emph{clausification} process mainly described
in~\cite{Sutcliffe1996}.

\begin{figure}
  \[
    \text{Disjunction:}
    \begin{bprooftree}
      \AxiomC{$Γ ⟝ φ ∨ ⊥$}
      \UnaryInfC{$Γ ⟝ φ$}
    \end{bprooftree}
    \qquad
    \begin{bprooftree}
      \AxiomC{$Γ ⟝ φ ∨ ⊤$}
      \UnaryInfC{$Γ ⟝ ⊤$}
    \end{bprooftree}
    \qquad
    \begin{bprooftree}
      \AxiomC{$Γ ⟝ φ ∨ ¬ φ$}
      \UnaryInfC{$Γ ⟝ ⊤$}
    \end{bprooftree}
    \qquad
    \begin{bprooftree}
      \AxiomC{$Γ ⟝ φ ∨ φ$}
      \UnaryInfC{$Γ ⟝ φ$}
    \end{bprooftree}
  \]

  \[
    \text{Conjunction:}
    \begin{bprooftree}
      \AxiomC{$Γ ⟝ φ ∧ ⊥$}
      \UnaryInfC{$Γ ⟝ ⊥$}
    \end{bprooftree}
    \qquad
    \begin{bprooftree}
      \AxiomC{$Γ ⟝ φ ∧ ⊤$}
      \UnaryInfC{$Γ ⟝ φ$}
    \end{bprooftree}
    \qquad
    \begin{bprooftree}
      \AxiomC{$Γ ⟝ φ ∧ φ$}
      \UnaryInfC{$Γ ⟝ φ$}
    \end{bprooftree}
    \qquad
    \begin{bprooftree}
      \AxiomC{$Γ ⟝ φ ∧ ¬ φ$}
      \UnaryInfC{$Γ ⟝ ⊥$}
    \end{bprooftree}
  \]
  \caption{Theorems to remove inside of a formula by the
  \canonicalize rule.}
\label{fig:redundancies}
\end{figure}

The \canonicalize rule is mainly used by \Metis to introduce the negation of a
subgoal in its refutation proof. When the input formula is an axiom, definition
or hypothesis introduced in the \TPTP file of the problem, the \canonicalize
rule returns its negative normal form. In the remaining cases, as far as we
know, \Metis uses the \canonicalize rule to performs a sort of simplifications
to remove the tautologies listed in~Fig.~\ref{fig:redundancies} in the
conjunctive normal form of the input formula.

To reconstruct this rule, we guide its prensentation as follows. In
Lemma~\ref{lem:simplify-or} and Lemma~\ref{lem:simplify-and}, we show how to
simplify in disjunctions and conjunctions, respectively. The negative normal form
is stated in~Lemma~\ref{lem:nnf} and the conjunctive normal form in
Lemma~\ref{lem:cnf}. Finally, the reconstruction of the \canonicalize rule is
stated in Theorem~\ref{thm:canonicalize}.

% ----------------------------------------------------------------------------
% -- Simplify-Or --
% ----------------------------------------------------------------------------

\begin{notation}
In a disjunction, $φ ≡ φ₁ ∨ φ₂ ∨ \cdots ∨ φₙ$, we say $ψ ∈_{∨} φ$,
if there is some $i = 1, \cdots, n$ such that $ψ ≡ φᵢ$.
Note that $ψ ∈_{∨} φ$ is another representation for the equality
$ψ ≡ \freorder_{∨}~φ~ψ$.
We assume all disjunctions to be right-associative unless otherwise
stated.
\end{notation}

\begin{mainlemma}
  \label{lem:simplify-or}
  %Let $φ : \Prop$ be a right-associative formula.
  If $Γ~⟝~φ$ then $Γ~⟝~\fcanon_{∨}~φ$, where

\begin{equation}
\label{eq:canon-or}
\begin{aligned}
 &\hspace{.495mm}\fcanon_{∨} : \NProp \to \NProp\\
 &\begin{array}{lll}
   \fcanon_{∨} &(⊥ ∨ φ)     &= \fcanon_{∨}~φ \\
   \fcanon_{∨} &(φ ∨ ⊥)     &= \fcanon_{∨}~φ \\
   \fcanon_{∨} &(⊤ ∨ φ)     &= ⊤\\
   \fcanon_{∨} &(φ ∨ ⊤)     &= ⊤\\
   \fcanon_{∨} &(φ₁ ∨ φ₂)   &=
    \begin{cases}
     ⊤,                     & \text{ if } φ₁ ≡ ¬ ψ \text{ for some }ψ : \Prop \text{ and } ψ ∈_{∨} φ₂;\\
     ⊤,                     & \text{ if } (¬ φ₁) ∈_{∨} φ₂;\\
     \fcanon_{∨}~φ₂,        & \text{ if } φ₁ ∈_{∨} φ₂;\\
     ⊤,                     & \text{ if } \fcanon_{∨}~φ₂~ ≡ ⊤;\\
     φ₁,                    & \text{ if } \fcanon_{∨}~φ₂~ ≡ ⊥;\\
     φ₁ ∨ \fcanon_{∨}~φ₂,   & \text{ otherwise.}
    \end{cases}\\
   \fcanon_{∨} &φ           &= φ.
  \end{array}
\end{aligned}
\end{equation}
\end{mainlemma}

% ----------------------------------------------------------------------------
% -- Simplify-and --
% ----------------------------------------------------------------------------

Now, we have removed tautologies in disjunctions by applying the $\fcanon_{∨}$
function, we formulate in a similar way, the $\fcanon_{∧}$ function to simplify
tautologies about conjunctions.

\begin{notation}
In a conjunction, $φ ≡ φ₁ ∧ φ₂ ∧ \cdots ∧ φₙ$, we say
$ψ ∈_{∧} φ$, if there is some $i = 1, \cdots, n$ such that $ψ ≡ φᵢ$.
Note that $ψ ∈_{∧} φ$ is another representation of
the equality $ψ ≡ \fconjunct~φ~ψ$.
We assume all conjunctions to be right-associative unless otherwise
stated.
\end{notation}

\begin{mainlemma}
  \label{lem:simplify-and}
  %Let $φ : \Prop$ be a right-associative formula.
  If $Γ~⟝~φ$ then $Γ~⟝~\fcanon_{∧}~φ$, where

  \begin{equation}
   \label{eq:simplify-and}
    \begin{aligned}
     &\hspace{.495mm}\fcanon_{∧} : \NProp \to \NProp\\
      &\begin{array}{lll}
        \fcanon_{∧} &(⊥ ∧ φ)     &= ⊥  \\
        \fcanon_{∧} &(φ ∧ ⊥)     &= ⊥  \\
        \fcanon_{∧} &(⊤ ∧ φ)     &= \fcanon_{∧}~φ \\
        \fcanon_{∧} &(φ ∧ ⊤)     &= \fcanon_{∧}~φ \\
        \fcanon_{∧} &(φ₁ ∧ φ₂) &=
          \begin{cases}
            ⊥,                   & \text{ if } φ₁ ≡ ¬ ψ \text{ for some }ψ : \Prop \text{ and } ψ ∈_{∧} φ₂;\\
            ⊥,                   & \text{ if } (¬ φ₁) ∈_{∧} φ₂;\\
            \fcanon_{∧}~φ₂,      & \text{ if } φ₁ ∈_{∧} φ₂;\\
            φ₁,                  & \text{ if } \fcanon_{∧}~φ₂~≡ ⊤;\\
            ⊥,                   & \text{ if } \fcanon_{∧}~φ₂~≡ ⊥;\\
            φ₁ ∧ \fcanon_{∧}~φ₂, &\text{ otherwise.}
          \end{cases}\\
        \fcanon_{∧} &φ         &= φ.
       \end{array}
    \end{aligned}
    \end{equation}
\end{mainlemma}

Using the functions $\fcanon_{∨}$ and $\fcanon_{∧}$ we can define our function for
the negative normal form present in \Metis, we once again formulate this version
based on the \Metis source code to \emph{normalize} formulas. We use bounded
recursion to define the function $\fnnf_{1}$ in~\eqref{eq:nnf}.

% ----------------------------------------------------------------------------
% -- NNF --
% ----------------------------------------------------------------------------

\begin{equation}
\label{eq:nnf}
\begin{aligned}
&\hspace{.495mm}\fnnf_{1} : \Prop \to \Nat \to \Prop\\
&\begin{array}{llll}
\fnnf_{1} &(φ₁ ∧ φ₂)     &(\suc~n) &= \fcanon_{∧}~(\fassoc_{∧}~(\fnnf_{1}~φ₁~n ∧ \fnnf_{1}~φ₂~n))\\
\fnnf_{1} &(φ₁ ∨ φ₂)     &(\suc~n) &= \fcanon_{∨}~(\fassoc_{∨}~(\fnnf_{1}~φ₁~n ∨ \fnnf_{1}~φ₂~n))\\
\fnnf_{1} &(φ₁ ⇒ φ₂)     &(\suc~n) &= \fcanon_{∨}~(\fassoc_{∨}~(\fnnf_{1}~((¬ φ₁) ∨ φ₂)~n))\\
\fnnf_{1} &(¬ (φ₁ ∧ φ₂)) &(\suc~n) &= \fcanon_{∨}~(\fassoc_{∨}~(\fnnf_{1}~((¬ φ₁) ∨ (¬ φ₂))~n))\\
\fnnf_{1} &(¬ (φ₁ ∨ φ₂)) &(\suc~n) &= \fcanon_{∧}~(\fassoc_{∧}~(\fnnf_{1}~((¬ φ₁) ∧ (¬ φ₂))~n))\\
\fnnf_{1} &(¬ (φ₁ ⇒ φ₂)) &(\suc~n) &= \fcanon_{∧}~(\fassoc_{∧}~(\fnnf_{1}~((¬ φ₂) ∧ φ₁)~n))\\
\fnnf_{1} &(¬ (¬ φ))     &(\suc~n) &= \fnnf_{1}~φ₁~n\\
\fnnf_{1} &(¬ ⊤)         &(\suc~n) &= ⊥\\
\fnnf_{1} &(¬ ⊥)         &(\suc~n) &= ⊤\\
\fnnf_{1} &φ             &\zero    &= φ
\end{array}
\end{aligned}
\end{equation}

The function $\fnnf$ in~\eqref{eq:fnnf} is defined similarly as the \fstrip
function in~\eqref{eq:strip}, the complexity measure function $\fnnf_{cm}$ is
defined in Appendix in~\cite{Prieto-Cubides2017a}.

\begin{align}
  \label{eq:fnnf}
 \begin{split}
   &\fnnf : \NProp \to \NProp\\
   &\fnnf~φ = \fnnf_{1}~φ~(\fnnf_{cm} \varphi).
 \end{split}
\end{align}

\begin{mainlemma}
  \label{lem:nnf}
  If $Γ ⊢ φ$ then $Γ ⟝ \fnnf~φ$.
\end{mainlemma}

%%%
%%% TODO: 2018-06-25 17:14
%%%

To get the conjunctive normal form, we make sure the formula is a
conjunction of disjunctions. For such a purpose, we use distributive laws in
Lemma~\ref{lem:dist} to get that form after applying the $\fnnf$ function.

\begin{mainlemma}
  \label{lem:dist}
  $Γ ⟝ φ$ then $Γ ⟝ \fdist~φ$, where
  \begin{equation*}
  \begin{aligned}
  &\hspace{.495mm}\fdist : \NProp \to \NProp\\
  &\begin{array}{lll}
    \fdist &(φ₁ ∧ φ₂) &= \fdist~φ₁ ∧ \fdist~φ₂\\
    \fdist &(φ₁ ∨ φ₂) &= \fdist_{∨}~(\fdist~φ₁ ∨ \fdist~φ₂)\\
    \fdist &φ         &= φ
   \end{array}\\
  \text{and}\\
  &\hspace{.495mm}\fdist_{∨} : \NProp \to \NProp\\
  &\begin{array}{lll}
    \fdist_{∨}&((φ₁ ∧ φ₂) ∨ φ₃) &= \fdist_{∨}~(φ₁ ∨ φ₂) ∧ \fdist_{∨}~(φ₂ ∨ φ₃)\\
    \fdist_{∨}&(φ₁ ∨ (φ₂ ∧ φ₃)) &= \fdist_{∨}~(φ₁ ∨ φ₂) ∧ \fdist_{∨}~(φ₁ ∨ φ₃)\\
    \fdist_{∨}&φ &= φ.
    \end{array}
   \end{aligned}
  \end{equation*}
\end{mainlemma}

We get the conjunctive normal form by applying
the $\fnnf$ function follow by the $\fdist$ function.

\begin{mainlemma}
\label{lem:cnf}
  If $Γ ⊢ φ$ then $Γ ⊢ \fcnf~φ$, where
  \begin{align*}
    \begin{split}
    &\fcnf : \Prop \to \Prop\\
    &\fcnf~φ = \fdist~(\fnnf~φ).
    \end{split}
  \end{align*}
\end{mainlemma}


Since all the transformations in Lemma~\ref{lem:nnf} and
Lemma~\ref{lem:dist} came from logical equivalences in propositional
logic, we state the following lemmas used in the reconstruction of the \simplify rule in Lemma~\ref{lem:reduce-literal} and in
Theorem~\ref{thm:canonicalize} for the \canonicalize rule.
\begin{mainlemma}
\label{lem:nnf-inv}
  If $Γ ⊢ \fnnf~φ$ then $Γ ⊢ φ$.
\end{mainlemma}

\begin{mainlemma}
\label{lem:cnf-inv}
  If $Γ ⊢ \fcnf~φ$ then $Γ ⊢ φ$.
\end{mainlemma}

Now, we are ready to reconstruct the \fcanonicalize rule. This inference rule
defined in \eqref{eq:canonicalize} performs normalization for a proposition.
That is, depending on the role of the formula in the problem, it converts that
formula to its negative normal form or its conjunctive normal form. In both
cases, \canonicalize simplifies the formula by removing tautologies inside of
it as we widely described above for theorems in Fig.~\ref{fig:redundancies}.

Based on the \Metis source code, when the formula plays the axiom or
definition role, the \canonicalize rule transforms the source formula
to its negative normal form. Otherwise, this rule converts the
formula to a simplified conjunctive normal form.

Since this rule mostly consists of dealing with clauses, to
reconstruct this rule, our strategy mainly consists of checking the
equality of negative normal form between the source and the target
formula. If it fails, we try to reorder the conjunctive normal form
of the source formula to match with the conjunctive normal form of
the target formula. It definition is as follows.


\begin{mainth} % (fold)
  \label{thm:canonicalize}
   Let $ψ : \Target$. If $Γ ⊢ φ$ then $Γ~⊢~\fcanonicalize~φ~ψ$, where
  \begin{equation}
  \label{eq:canonicalize}
  \begin{aligned}
  &\hspace{.495mm}\fcanonicalize : \Source \to \Target \to \Prop\\
  &\fcanonicalize~φ~ψ = \begin{cases}
        ψ, &\text{ if  } ψ ≡ φ;\\
        ψ, &\text{ if  } ψ ≡ \fnnf~φ;\\
        ψ, &\text{ if  } \fcnf~ψ≡ \freorder_{∧∨}~(\fcnf~φ)~(\fcnf~ψ);\\
        φ, &\text{ otherwise. }
        \end{cases}
   \end{aligned}
  \end{equation}
\end{mainth}

\end{document}
