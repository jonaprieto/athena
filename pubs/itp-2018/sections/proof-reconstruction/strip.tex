\documentclass[../../main.tex]{subfiles}
\begin{document}

\subsubsection{Strip.}
\label{sssec:strip-a-goal}

To prove a goal, \Metis splits the goal into disjoint cases. This process
produces a list of new subgoals, the conjunction of these subgoals implies the
goal. Then, a proof of the goal becomes into a set of smaller proofs, one
refutation for each subgoal.

\begin{myexamplenum}
In Fig.~\ref{complete-metis-example}, we can see how
the subgoals associated to a goal are introduced
in the \TSTP derivation line \ref{line3} with the \strip inference rule.
The conjecture in line \ref{line2}, $(p ∨ q) ⇒ (p ∧ q)$,
is stripped into two subgoals (line~\ref{line4} and \ref{line5}):
$(p ∨ q) ⇒ p$ and $((p ∨ q) ∧ p) ⇒ q$.
\Metis proves each subgoal separately following the same order as
they appeared in the \TSTP derivation.
\end{myexamplenum}

\begin{remark}
\Metis does not make explicit in the \TSTP derivation the way it uses the
subgoals to prove the goal. We show this approach in the correctness of the
\strip inference rule in Theorem~\ref{thm:strip}. To show this theorem, we prove
Lemma~\ref{lem:inv-uh-lem} and  Lemma~\ref{lem:lem-inv-strip}. In the former, we
introduce the auxiliary function $\fuh_1$ to define the $\fuh$ function used in
in the latter in the definition of the $\fstrip$ function.
\end{remark}

\begin{mainlemma}
  \label{lem:inv-uh-lem}
Let be $\text{n} : \Nat$. If $Γ ⊢ \fuh₁~φ~n$ then $Γ ⊢ φ$ where

\begin{equation}
\label{eq:uh-structured}
\begin{aligned}
&\hspace{.495mm} \fuh_{1} : \Prop → \Nat → \Prop\\
&\begin{array}{llll}
\fuh_{1} &(φ₁ ⇒ (φ₂ ⇒ φ₃)) &(\suc~n) &= \fuh_{1}~((φ₁ ∧ φ₂) ⇒ φ₃)~n\\
\fuh_{1} &(φ₁ ⇒ (φ₂ ∧ φ₃)) &(\suc~n) &= \fuh_{1}~(φ₁ ⇒ φ₂)~n ∧ \fuh_{1}~(φ₁ ⇒ φ₃)~n\\
\fuh_{1} &φ &n &= φ.
\end{array}
\end{aligned}
\end{equation}
\end{mainlemma}

The second argument in the $\fuh_1$ function is a natural number that constrains
the number of recursive calls. To provide such a number given a formula
$\varphi$, we call the $\fuh_{cm}$ function defined in~\eqref{eq:uh-complexity}.
This function finds what we call the \emph{complexity measure} of the function
$\fuh_1$. For the convenience in the following descriptions, we have the
function $\fuh : \Prop → \Prop$ for $\varphi \mapsto \fuh_{1}~φ~(\fuh_{cm}~φ)$.
\begin{equation}
  \label{eq:uh-complexity}
  \begin{aligned}
    &\hspace{.495mm}\fuh_{cm} : \Prop → \Nat\\
    &\begin{array}{llll}
    \fuh_{cm}~(φ₁ ⇒ (φ₂ ⇒ φ₃)) &= \fuh_{cm}~φ₃ + 2\\
    \fuh_{cm}~(φ₁ ⇒ (φ₂ ∧ φ₃)) &= \fmax~(\fuh_{cm}~φ₂)~(\fuh_{cm}~φ₃) + 1\\
    \fuh_{cm}~φ                &= 0.
    \end{array}
  \end{aligned}
\end{equation}
Now, we define the $\fstrip$ function in \eqref{eq:strip} in a similar way as we
did above for the $\fuh$ function. We have a definition of the $\fstrip₁$ function
in~\eqref{eq:strip-fixed} and a complexity measure denoted by $\fstrip_{cm}$
and defined in~\cite{Prieto-Cubides2017a}.
% \vspace*{-5mm}
\begin{align}
  \label{eq:strip}
  \begin{split}
  &\fstrip : \Prop → \Prop\\
  &\fstrip~φ~ = \fstrip₁~φ~(\fstrip_{cm}~φ).
  \end{split}
\end{align}

The $\fstrip$ function yields the conjunction of subgoals that implies the goal
of the problem in the \Metis \TSTP derivations. Its definition comes mainly from
the reading of the \Metis source code and its \TSTP derivations.

\begin{equation}
\label{eq:strip-fixed}
\begin{aligned}
&\hspace{.495mm}\fstrip₁ : \Prop → \Nat → \Prop\\
&\begin{array}{llll}
\fstrip₁ &(φ₁ ∧ φ₂)     &(\suc~n) &= \fuh~(\fstrip₁~φ₁~n) ∧ \fuh~(φ₁ ⇒ \fstrip₁~φ₂~n)\\
\fstrip₁ &(φ₁ ∨ φ₂)     &(\suc~n) &= \fuh~((¬ φ₁) ⇒ \fstrip₁~φ₂~n)\\
\fstrip₁ &(φ₁ ⇒ φ₂)     &(\suc~n) &= \fuh~(φ₁ ⇒ \fstrip₁~φ₂~n)\\
\fstrip₁ &(¬ (φ₁ ∧ φ₂)) &(\suc~n) &= \fuh~(φ₁ ⇒ \fstrip₁~(¬ φ₂)~n)\\
\fstrip₁ &(¬ (φ₁ ∨ φ₂)) &(\suc~n) &= \fuh~(\fstrip₁~(¬ φ₁)~n) ∧ \fuh~((¬ φ₁) ⇒ \fstrip₁~(¬ φ₂)~n)\\
\fstrip₁ &(¬ (φ₁ ⇒ φ₂)) &(\suc~n) &= \fuh~(\fstrip₁~φ₁~n) ∧ \fuh~(φ₁ ⇒ \fstrip₁~(¬ φ₂)~n)\\
\fstrip₁ &(¬ (¬ φ₁))    &(\suc~n) &= \fuh~(\fstrip₁~φ₁~n)\\
\fstrip₁ &(¬ ⊥)         &(\suc~n) &= ⊤\\
\fstrip₁ &(¬ ⊤)         &(\suc~n) &= ⊥\\
\fstrip₁ &φ             &n        &= φ.
\end{array}
\end{aligned}
\end{equation}

\begin{mainlemma}
\label{lem:lem-inv-strip}
Let $n : \Nat$. If $Γ~⊢~\fstrip₁~φ~n$ then $Γ~⊢~φ$.
\end{mainlemma}

<<<<<<< HEAD
The following lemma is convenient to substitute in a theorem equals by equals in
the conclusion of the sequent. Recall the equality (≡) is symmetric and
transitive, and we use these properties without any mention.
=======
The following lemma is convenient to substitute equals by equals in a theorem.
Recall the equality (≡) is symmetric and transitive, and we use these properties
without any mention.
>>>>>>> a9fedb680d16b5a15e7397f9111524be5f53f071

\begin{mainlemma}[\fsubst]
  \label{lem:subst}
  If $Γ ⊢ φ$ and $ψ ≡ φ$ then $Γ ⊢ ψ$.
% \begin{equation*}
%   \label{eq:substitution-theorem}
%   \begin{bprooftree}
%   \AxiomC{$Γ ⊢ φ$}   \AxiomC{$ψ ≡ φ$}
%   \RightLabel{\fsubst}
%   \BinaryInfC{$Γ ⊢ ψ$}
%   \end{bprooftree}
% \end{equation*}
\end{mainlemma}

We can now formulate the result that justifies the stripping strategy of \Metis
to prove goals. For the sake of brevity, we state the following theorem for the
\strip function when the goal has two subgoals. In other cases, we extend the
theorem in the natural way.

\begin{mainth}
\label{thm:strip}
Let $s_1$ and $s_2$ be the subgoals of the goal $φ$, that is,
$\fstrip~\varphi≡~s_{1}∧~s_{2}$.
% $\fstrip₁~φ~n~≡~s₂∧~s₃$ where $\text{n} : \Nat$ is the complexity measure $\fstrip_{cm}$.
If $Γ ⊢ s₂$ and $Γ ⊢ s₃$ then $Γ ⊢ φ$.
\end{mainth}
\begin{proof}
\begin{equation*}
  \begin{bprooftree}
  \AxiomC{$ Γ ⊢ s_1 $}
  \AxiomC{$ Γ ⊢ s_2 $}
  \RightLabel{∧-intro}
  \BinaryInfC{$Γ ⊢ s_1\wedge s_2$}
  \AxiomC{$\fstrip~φ ≡ s_1\wedge s_2 $}
  \RightLabel{By def.~\eqref{eq:strip}}
  \UnaryInfC{$\fstrip₁~φ~(\fstrip_{cm}~\varphi) ≡ s_1\wedge s_2 $}
  \RightLabel{\fsubst}
  \BinaryInfC{$Γ ⊢ \fstrip₁~φ~n$}
  \RightLabel{Lemma~\ref{lem:lem-inv-strip}}
  \UnaryInfC{$Γ ⊢ φ$}
\end{bprooftree}
\end{equation*}
\end{proof}

% -------------------------------------------------------------------

\begin{remark}

The user of \Metis will probably note another rule in the derivations related
with these subgoals, the \negate rule. Since \Metis proves a conjecture by
<<<<<<< HEAD
refutation, to prove a subgoal, \Metis assumes its negation using the \negate
rule. This rule always ocurrs in the \TSTP derivation after the application of
the \strip rule that introduces the subgoal.
=======
refutation, to prove a subgoal, \Metis assumes its negation with the \negate
rule. This negation ocurrs always after the application of the \strip rule which
as we mentioned before it is the rule that introduces each subgoal.
>>>>>>> a9fedb680d16b5a15e7397f9111524be5f53f071

\end{remark}

\end{document}
