\documentclass[../../main.tex]{subfiles}
\begin{document}

\subsubsection{Conjunct.}
\label{sssec:splitting-a-conjunct}

The \conjunct rule extracts from a conjunction one of its conjuncts.
This rule is a generalization of the projection rules for the
conjunction connective as the following \TSTP excerpt shows.

\begin{myexamplenum}\hspace{10cm}

\begin{verbatim}
  fof(p₁, p ∧ q ∧ (r ∨ ¬ p), ...
  fof(p₂, q, inf(conjunct, p₁)).
  fof(p₃, r ∨ ¬ p, inf(conjunct, p₁)).
\end{verbatim}

In the first formula, $p ∧ q ∧ (r ∨ ¬ p)$, we find a left-associative
conjunction named \verb!p₁!. The \conjunct rule extracts $q$ from the
\verb!p₁! using a left projection (∧-proj₁) follow by a right
projection (∧-proj₂). After, the \conjunct rule extracts $r ∨ ¬ p$ by
using a right projection on \verb!p₁!.

\end{myexamplenum}

\begin{mainth}
  \label{thm:conjunct}
  Let $ψ : \Target$. If $Γ ⊢ φ$ then $Γ ⊢ \fconjunct~φ~ψ$, where
  \begin{equation*}
  % \label{eq:rank-definition}
  \begin{aligned}
  &\hspace{.495mm}\fconjunct : \Source \to \Target \to \Prop\\
  &\begin{array}{llll}
  \fconjunct~φ~ψ &=
        \begin{cases}
            ψ, &\text{ if }φ ≡ ψ;\\
            ψ, &\text{ if }φ ≡ φ₁ ∧ φ₂\text{ and }ψ ≡ \fconjunct~φ₁~ψ;\\
            ψ, &\text{ if }φ ≡ φ₁ ∧ φ₂\text{ and }ψ ≡ \fconjunct~φ₂~ψ;\\
            φ, &\text{ otherwise.}
          \end{cases}
  \end{array}
  \end{aligned}
  \end{equation*}
\end{mainth}

\end{document}
