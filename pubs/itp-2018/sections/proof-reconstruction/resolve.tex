\documentclass[../../main.tex]{subfiles}
\begin{document}

\subsubsection{Resolve.}
\label{sssec:resolve}

Logic equivalence between propositions is a major issue to justify
prover reasoning steps. Since we left out semantics to treat only the
syntax aspects of propositional logic, our approach shows logic
equivalence by converting propositions to their conjunctive normal
form, and reordering those and the inner disjunctions to match them.

\begin{mydefinition}

A \emph{literal} is an propositional variable (positive literal) or a
negation of an propositional variable (negative literal).

\end{mydefinition}

\begin{notation}
We use \Lit as synonym of \Prop type to refer literals.
\end{notation}

\begin{mydefinition}

The \emph{negative normal form} of a formula is the logical
equivalent version of it in which negations appear only in the
literals and the formula does not contain any implications.

\end{mydefinition}

\begin{mydefinition}

The \emph{conjunctive normal form} of a formula also called clausal
normal form is the logical equivalent version expressed as a
conjunction of clauses where a \emph{clause} is the disjunction of
zero or more literals.

\end{mydefinition}

\begin{myexamplenum}\hspace{10cm}
\label{ex:resolve-tstp}
\begin{verbatim}
  cnf(r₄, ¬ r ∨ p ∨ q, ...
  cnf(r₅, p ∨ q ∨ r, ...
  cnf(r₆, p ∨ q, inf(resolve, r, [r₅, r₄])).
\end{verbatim}

In the excerpt above, we apply resolution to the first two formulas,
$¬ r ∨ p ∨ q$ and $p ∨ q ∨ r$.
The last line tells us the literal used
for resolution is $r$. Syntactically speaking, we can not derive
neither the conclusion $p ∨ q$ in \verb!r₆! nor apply the resolution
theorem with \verb!r₄! and \verb!r₅! since the formulas do not fit
the pattern required.

If the scenario would have other like replacing \verb!r₅! by
\begin{verbatim}
  cnf(r₅, r ∨ p ∨ q, ...
\end{verbatim}
The \resolve rule have could derive $(p ∨ q) ∨ (p ∨ q)$, but again, that is not the expected result.
\end{myexamplenum}

Therefore, we perform a sequence of rearrangements inside the
involved formulas to match with the expected pattern by the
\emph{resolve} inference rule in Fig.~\ref{fig:metis-inferences}.

Using reordering after applying a customized version of the
resolution theorem defined in~\eqref{eq:rsol} we get the expected
result.

\begin{mainlemma}
  \label{lem:rsol}
  If $Γ ⊢ φ$ then $Γ ⊢ \frsol~φ$, where
  \begin{equation}
    \label{eq:rsol}
    \begin{aligned}
    &\frsol : \Prop \to \Prop\\
    &\begin{array}{ll}
      \frsol~φ &=
        \begin{cases}
          φ₂,      &\text{ if }φ ≡ (φ₁ ∨ φ₂) ∧ (¬ φ₁ ∨ φ₂);\\
          φ₂ ∨ φ₄, &\text{ if }φ ≡ (φ₁ ∨ φ₂) ∧ (¬ φ₁ ∨ φ₄);\\
          φ, &\text{ otherwise.}
        \end{cases}
      \end{array}
      \end{aligned}
\end{equation}
\end{mainlemma}

\begin{mainth}
  \label{thm:resolve}
  Let $ℓ$ be a literal, $ℓ : \Lit$, and $ψ : \Target$. If $Γ ⊢ φ₁$ and
  $Γ~⊢~φ₂$ then $Γ~⊢~\fresolve~φ₁~φ₂~ℓ~ψ$, where
  \begin{equation}
  \begin{split}
  \label{eq:resolve}
    &\fresolve : \Source \to \Source \to \Lit \to \Target \to \Prop\\
    &\fresolve~φ₁~φ₂~ℓ~ψ =
      \frsol~(\freorder_{∨}~φ₁~(ℓ ∨ ψ) ∧ \freorder_{∨}~φ₂~(¬ ℓ ∨ ψ)).
  \end{split}
  \end{equation}
\end{mainth}

\Needspace{3\baselineskip}
\begin{proof}
  \begin{equation*}
  \begin{bprooftree}
    \AxiomC{$Γ ⊢ φ₁$}
    \RightLabel{Lemma~\ref{lem:reorder-and-or}}
    \UnaryInfC{$Γ ⊢ \freorder_{∨}~φ₁~(ℓ ∨ ψ)$}
    \RightLabel{Lemma~\ref{lem:reorder-and-or}}
    \AxiomC{$Γ ⊢ φ₂$}
    \UnaryInfC{$Γ ⊢ \freorder_{∨}~φ₂~(¬ ℓ ∨ ψ)$}
    \RightLabel{∧-intro}
    \BinaryInfC{$Γ ⊢ \freorder_{∨}~φ₁~(ℓ ∨ ψ) ∧ \freorder_{∨}~φ₂~(¬ ℓ ∨ ψ)$}
    \RightLabel{Lemma~\ref{lem:rsol}}
    \UnaryInfC{$Γ ⊢ \frsol~(\freorder_{∨}~φ₁~(ℓ ∨ ψ) ∧
     \freorder_{∨}~φ₂~(¬ ℓ ∨ ψ))$}
    \RightLabel{by (\ref{eq:resolve})}
    \UnaryInfC{$Γ ⊢ \fresolve~φ₁~φ₂~ℓ~ψ$}
  \end{bprooftree}
  \end{equation*}
\end{proof}

\end{document}
