\documentclass[../../main.tex]{subfiles}
\begin{document}


% -- TODO algun texto aqui?

\begin{mydefinition}

A \emph{literal} is a propositional variable (positive literal) or a
negation of a propositional variable (negative literal).

\end{mydefinition}

\begin{notation}
\Lit is a synonym of \Prop to refer to literals.
\end{notation}

\begin{mydefinition}

The \emph{negative normal form} of a formula is the logical
equivalent version of it in which negations appear only in the
literals and the formula does not contain any implications.

\end{mydefinition}

\begin{mydefinition}

The \emph{conjunctive normal form} of a formula also called clausal
normal form is the logical equivalent version expressed as a
conjunction of clauses where a \emph{clause} is the disjunction of
zero or more literals.
\end{mydefinition}

% ----------------------------------------------------------------------------
% -- RESOLVE --
% ----------------------------------------------------------------------------

\subsubsection{Resolve.}
\label{sssec:resolve}

The \resolve inference rule in combination with the \simplify rule is  the
resolution theorem for propositional logic that we found in the \Metis \TSTP
derivations (see this rule in~Fig.~\ref{fig:metis-inferences}). In particular,
to reconstruct the \resolve rule following the pattern in
Example~\ref{ex:inference-rule-pattern}, it is necessary to have a mechanism to
test for logic equivalence between propositions. The challenge is because of we
left out the semantics to treat only the syntax aspects of the propositional
formulas, therefore we have to manipulate in somehow the formulas to see if they
are logically equivalent.

Our approach to address this issue mainly consists of checking for syntactical
equality between the formulas. Firstly by converting them to their conjunctive
normal and secondly, by performing reordering of the inner formulas of the
resulting normal forms in the first conversion when they were not equal.

In~\cite{Prieto-Cubides2017a}, we make a description of the complete machinery
we use for that purpose. For instance, we define functions of the type ($\Source
\to  \Target \to \Prop$) like $\freorder_{∨}$ to fix the order of a disjunction
(premise) given by the order in the literals of another disjunction (conclusion)
if it is possible. Following the same way, we define for conjunctions the
function $\freorder_{\wedge}$ and for conjunctive normal forms the function
$\freorder_{\wedge\vee}$.

Using reordering after applying a customized version of the
resolution theorem defined in~\eqref{eq:rsol} we get the expected
result.

\begin{mainlemma}
  \label{lem:rsol}
  If $Γ ⊢ φ$ then $Γ ⊢ \frsol~φ$, where
  \begin{equation}
    \label{eq:rsol}
    \begin{aligned}
    &\frsol : \Prop \to \Prop\\
    &\begin{array}{ll}
      \frsol~φ &=
        \begin{cases}
          φ₂,      &\text{ if }φ ≡ (φ₁ ∨ φ₂) ∧ (¬ φ₁ ∨ φ₂);\\
          φ₂ ∨ φ₄, &\text{ if }φ ≡ (φ₁ ∨ φ₂) ∧ (¬ φ₁ ∨ φ₄);\\
          φ, &\text{ otherwise.}
        \end{cases}
      \end{array}
      \end{aligned}
\end{equation}
\end{mainlemma}

\begin{mainth}
  \label{thm:resolve}
  Let $ℓ$ be a literal, $ℓ : \Lit$, and $ψ : \Target$. If $Γ ⊢ φ₁$ and
  $Γ~⊢~φ₂$ then $Γ~⊢~\fresolve~φ₁~φ₂~ℓ~ψ$, where
  \begin{equation}
  \begin{split}
  \label{eq:resolve}
    &\fresolve : \Source \to \Source \to \Lit \to \Target \to \Prop\\
    &\fresolve~φ₁~φ₂~ℓ~ψ =
      \frsol~(\freorder_{∨}~φ₁~(ℓ ∨ ψ) ∧ \freorder_{∨}~φ₂~(¬ ℓ ∨ ψ)).
  \end{split}
  \end{equation}
\end{mainth}

\end{document}
