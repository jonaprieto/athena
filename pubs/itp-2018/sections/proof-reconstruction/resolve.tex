\documentclass[../../main.tex]{subfiles}
\begin{document}

% ----------------------------------------------------------------------------
% -- RESOLVE --
% ----------------------------------------------------------------------------

\subsubsection{Resolve.}
\label{sssec:resolve}

The \resolve inference rule in combination with the \simplify rule are the
responsable rules to perform resolution. The \fresolve rules is a particular
version of the resolution theorem for propositional logic \TSTP derivations (see
this rule in~Fig.~\ref{fig:metis-inferences}).

\begin{example} In the \TSTP derivation in
Fig.~\ref{eq:complete-example-problem}, there are two ocurrences in
line~\ref{line14} and line~\ref{line16}~of the \resolve rule. This rule has two
inputs: the \emph{resolvent}, a literal denoted by ℓ in~Fig.\ref{fig:metis-inferences},
and a list of two formulas. \end{example}

In~\cite{Prieto-Cubides2017a}, we describe our first attempt to reconstruct this
rule that performs reordering of formulas jointly with the application of a
customized version of the resolution theorem to reconstruct the rule. We propose
an alternative definition formulated as a particular case
of the \fsimplify rule. We do this after seeing the behavior
of this rule in the \TSTP derivations from a more practical point of view for
the type-checking.

\begin{mainth}
  \label{thm:resolve}
  Let $ℓ$ be a literal, $ℓ : \Lit$, and $ψ : \Target$. If $Γ ⊢ φ₁$ and
  $Γ~⊢~φ₂$ then $Γ~⊢~\fresolve~φ₁~φ₂~ℓ~ψ$, where
  \begin{equation}
  \begin{split}
  \label{eq:resolve}
    &\fresolve : \Source \to \Source \to \Lit \to \Target \to \Prop\\
    &\fresolve~φ₁~φ₂~ℓ~ψ~=~\fsimplify~(φ₁ ∧ φ₂)~(¬ ℓ ∨ ℓ)~ψ
  \end{split}
  \end{equation}
\end{mainth}

%% maybe we want to show this easy proof.
% \begin{proof}
%   \begin{equation*}
%   \begin{bprooftree}
%     \AxiomC{$Γ ⊢ φ₁$}
%     \AxiomC{$Γ ⊢ φ₂$}
%     \RightLabel{∧-intro}
%     \BinaryInfC{$Γ ⊢~\varphi_1~\wedge~\varphi_2$}
%     \AxiomC{\abbre{PEM}~$\ell$}
%     \RightLabel{Theorem~\ref{thm:simplify}.}
%     \BinaryInfC{$\Gamma\vdash~\fsimplify~~(φ₁ ∧ φ₂)~(¬ ℓ ∨ ℓ)~ψ$}
%   \end{bprooftree}
%   \end{equation*}
% \end{proof}

\end{document}
