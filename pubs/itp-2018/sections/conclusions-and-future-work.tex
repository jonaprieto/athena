
\documentclass[../main.tex]{subfiles}
\begin{document}

% ===================================================================

\section{Conclusions}
\label{sec:conclusions}

We presented a proof-reconstruction approach in type theory for the
propositional fragment of the \Metis prover. We provided for each
\Metis inference rule a formal description in type theory following a
syntactical approach. These formalizations are mainly exposed in
Section~\ref{ssec:emulating-inferences}.

We built the \Athena translator tool written in \Haskell
that generates \Agda proof-terms of \Metis derivations.
\Agda files generated by this translator calls theorems from our
\Agda formalizations of the \Metis reasoning~\cite{AgdaProp,AgdaMetis}.

% This
% approach differs from our proof-reconstruction since we left out completely the
% use of propositions meanings towards a future work to support other logics where
% a syntactical approach plays an important role~(we refere the reader to
% \cite{Agudelo-Agudelo2017} for some examples)

The reconstruction approach in this study was designed to use
only syntactical aspects of the logic.
This decision was at the beginning a drawback
since it demands more detailed proofs, a
description of every transformation or deduction step performed by
the prover, which is rarely included in the output of these programs,
see Section~\ref{ssec:emulating-inferences}.
Nevertheless, we chose that syntactical treatment instead of using
semantics to extend this work towards the support of first-order
logic or other non-classical logics.
For first-order logic, recall satisfiability is undecidable and
its syntactical aspect plays an important role to reconstruct proofs.

This study increased the
trustworthiness of the automatic prover \Metis. Justifying a proof by
a theorem prover has a real significant impact for these automatic
tools.  The reverse engineering task to grasp the prover
reasoning can reveal important issues or bugs in many parts of these
systems (\eg, preprocessing, reasoning, or deduction modules). During
this research, we had the opportunity to contribute to \Metis by
reporting some bugs---see Issues No.~2, No.~4, and commit
\name{8a3f11e} in \Metis' official
repository.\footnote{\url{https://github.com/gilith/metis}.}
Fortunately, all these problems were fixed quickly by Hurd in
\Metis~2.3 (release~20170822).
\end{document}
